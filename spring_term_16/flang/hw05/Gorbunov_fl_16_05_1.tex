%!TEX program = xelatex

\documentclass[12pt, a4paper]{article}
\usepackage[utf8]{inputenc}
\usepackage[russian]{babel}
\usepackage{pscyr}
\usepackage{amssymb}
\usepackage{xifthen}
\usepackage{parskip}
\usepackage{hyperref}
\usepackage{setspace}

\usepackage{graphicx}
\usepackage{xcolor}
\usepackage{amsmath}
\usepackage{MnSymbol}
\usepackage{amsthm}
\usepackage{mathtools}
\usepackage{algorithm}
\usepackage[noend]{algpseudocode}
\usepackage[shortlabels]{enumitem}
                    \setlist[enumerate, 1]{1\textsuperscript{o}}
\usepackage{subfig}
\usepackage{tikz}
\usepackage{tikz,fullpage}
\usetikzlibrary{shapes,snakes}
\usetikzlibrary{arrows,%
                petri,%
                topaths}%
\usepackage{tkz-berge}
\usepackage[top=0.5in, bottom=0.7in, left=0.6in, right=0.6in]{geometry}
\linespread{1.3}

% \renewcommand\familydefault{\sfdefault}


% Stuff related to homework specific documents
\newcounter{MyTaskCounter}
\newcounter{MyTaskSectionCounter}
\newcommand{\tasksection}[1]{
	\stepcounter{MyTaskSectionCounter}
	\setcounter{MyTaskCounter}{0}
	\ifthenelse{\equal{#1}{}}{}{
	{\hfill\\[0.2in] \Large \textbf{\theMyTaskSectionCounter \enspace #1} \hfill\\[0.1in]}}
}

\newcommand{\task}[1]{
	\stepcounter{MyTaskCounter}
	\hfill\\[0.1in]
	\ifthenelse{\equal{\theMyTaskSectionCounter}{0}}{
	   \textbf{\large Задача №\theMyTaskCounter}
	}{
	   \textbf{\large Задача №\theMyTaskSectionCounter.\theMyTaskCounter}
	}
	\ifthenelse{\equal{#1}{}}{}{{\normalsize (#1)}}
	\hfill\\[0.05in]
}

% Math and algorithms

\makeatletter
\renewcommand{\ALG@name}{Алгоритм}
\renewcommand{\listalgorithmname}{Список алгроитмов}

\newenvironment{procedure}[1]
  {\renewcommand*{\ALG@name}{Процедура}
  \algorithm\renewcommand{\thealgorithm}{\thechapter.\arabic{algorithm} #1}}
  {\endalgorithm}

\makeatother

\algrenewcommand\algorithmicrequire{\textbf{Вход:}}
\algrenewcommand\algorithmicensure{\textbf{Выход:}}
\algnewcommand\True{\textbf{true}\space}
\algnewcommand\False{\textbf{false}\space}
\algnewcommand\And{\textbf{and}\space}

\newcommand{\xfor}[3]{#1 \textbf{from} #2 \textbf{to} #3}
\newcommand{\xassign}[2]{\State #1 $\leftarrow$ #2}
\newcommand{\xstate}[1]{\State #1}
\newcommand{\xreturn}[1]{\xstate{\textbf{return} #1}}

\DeclarePairedDelimiter\ceil{\lceil}{\rceil}
\DeclarePairedDelimiter\floor{\lfloor}{\rfloor}

\newcommand{\bigO}[1]{\mathcal{O}\left(#1\right)}

\newcommand{\xqed}{\hfill $\blacksquare$}

\newcommand{\code}[1]{\colorbox{gray!15}{\footnotesize\texttt{#1}}}

\title{Формальные языки. Домашнее задание на 16.05}
\author{Горбунов Егор Алексеевич}

\begin{document}
\maketitle

\begin{task}[1]
	Показать, что языки не являются контекстно свободными.
\end{task}
\begin{solution}
\begin{enumerate}
	\item Язык $a^ib^jc^k, i < j < k$. Будем показывать исходя из леммы о накачке. Для $n$. Рассмотрим слово $\omega = a^{n}b^{n+1}c^{n+2}$. Покажем, что какое бы разбиение не взяли $\omega = uvxyz$, что $vy \neq \varepsilon$ и $|vxy| \leq n$, то найдётся $k \geq 0$, что $uv^kxy^kz \notin L$. Т.к. $|vxy| \leq n$, то возможны только следующие случаи:
	\begin{itemize}
		\item $vxy = a^d$. Тогда мы всегда накачиваем $a$ ($v, y$ не пусты) и всегда можем найти такой $k$, что $v^kxy^k$ будет равен $a^D$, где $D > n + 1$.
		\item $vxy = a^db^q$. В этом случае при увеличении $k$ (накачивая) мы рано или поздно сделаем так, что либо степень $a$ будет больше степени $b$, либо степень $b$ будет больше степени $c$, т.к. степень $c$ остаётся неизменной.
		\item $vxy = b^d$. Аналогично случаю с $a^d$.
		\item $vxy = b^dc^q$. У нас в $v$ хотя бы один $b$ или в $y$ хотя бы один $c$. Положим $k = 0$, тогда в $uv^0xy^0z$ (накаченное $\omega$) будет хотя бы на 1 $b$ меньше чем в исходном $\omega$ (если в $u$ есть $b$), но тогда это слово не принадлежит языку, ибо степень $b$ в нём $\leq$ степени $a$. Аналогично если вдруг $u$ оказалось пустым, но тогда $y$ точно не пусто. (как я понял, по лемме важно, чтобы $vy$ было не пуста, а не каждый в отдельности)
		\item $vxy = c^d$. Тут берём $k = 0$ и аналогично предыдущему случаю.
	\end{itemize}
	\xqed

	\item Язык $a^nb^nc^i$, где $i \leq n$. Аналогично исходя из леммы на накачке. Для $n$. Рассмотрим слово $\omega = a^nb^nc^n$. $\omega = uvxyz$, $|vxy| \leq n$. Таким образом возможные случаи для $vxy$:
	\begin{itemize}
		\item $vxy = a^d$. Тут всё понятно. При накачке с любым $k$ мы потеряем равенство числа $a$-шек и $b$-шек
		\item $vxy = b^d$. Аналогично.
		\item $vxy = c^d$. Тут при накачке с любым $k > 0$ (достаточно $k=1$) получим, что число $c$-шек стало больше, чем число $a$-шек, а это уже вне языка!
		\item $vxy = a^db^q$. Возьмём тогда и положим $k = 0$. У нас тогда любо нарушится равенство числа $a$-шек и $b$-шек (это если в $v$ и $y$ их суммарно разное число), а если равенство и не нарушится, то т.к. $vy$ не пусто, то число $b$ и $a$ уменьшится хоть на 1, но тогда, т.к. число $c$-шек станет больше хотя бы на один в силу того, что во взятом слове число $a$ $b$ и $c$ одинаково.
		\item $vxy = b^dc^q$. Любая накачка приводит к нарушению либо равенства числа $a$ и $b$, либо к тому, что $c$ стало больше. (см. предыдущий пункт).
	\end{itemize}
	\xqed
	\item Язык $0^p$, где $p$ --- простое. Исходя из леммы о накачке. Для $n$. Рассмотрим слово подходящей длины:
	\[
		\omega = uvxyz = 0^{u} 0^{v} 0^{x} 0^{y} 0^{z}
	\]
	Накачаем его так:
	\begin{align*}
		& 0^{|u|} (0^{|v|})^{|u|+|x|+|z|} 0^{|x|} (0^{y})^{|u|+|x|+|z|} 0^{|z|} = 
		0^{|u|} 0^{|v|(|u|+|x|+|z|)} 0^{|x|} 0^{|y|(|u|+|x|+|z|)} 0^{z} = \\
		& = 0^{|u|+|x|+|z| + |v|(|u|+|x|+|z|) + |y|(|u|+|x|+|z|)} = 0^{(|u|+|x|+|z|)(1 + |v| + |y|)}
	\end{align*}
	Получили, что длинна накаченного слова составная! \xqed

	\item Язык $0^i1^j$, где $j = i^2$. Для $n$ достаточно рассмотреть слово $\omega = 0^n 1^{n^2}$.
	У нас $|vxy| \leq n$ (то, что в лемме о накачке). Таким образом суммарное число $0$ и $1$ в этой части слова равна $n$. Если при накачке число $0$ увеличивается на $1$, то число $1$ должно увеличиться на $2n+1$, чтобы получить слово из языка. Легко увидеть (правда =)), что такое невозможно.
	\xqed

	\item Язык $a^nb^nc^i$, где $n \leq i \leq 2n$. Для $n$ рассматриваем слово $\omega = a^n b^n c^{2n}$ и действуем аналогично пункту $(b)$.
\end{enumerate}
\end{solution}


\end{document}