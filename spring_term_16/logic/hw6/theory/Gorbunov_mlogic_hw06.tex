%!TEX program = xelatex

\documentclass[12pt, a4paper]{article}
\usepackage[utf8]{inputenc}
\usepackage[russian]{babel}
\usepackage{pscyr}
\usepackage{amssymb}
\usepackage{xifthen}
\usepackage{parskip}
\usepackage{hyperref}
\usepackage{setspace}

\usepackage{graphicx}
\usepackage{xcolor}
\usepackage{amsmath}
\usepackage{MnSymbol}
\usepackage{amsthm}
\usepackage{mathtools}
\usepackage{algorithm}
\usepackage[noend]{algpseudocode}
\usepackage[shortlabels]{enumitem}
                    \setlist[enumerate, 1]{1\textsuperscript{o}}
\usepackage{subfig}
\usepackage{tikz}
\usepackage{tikz,fullpage}
\usetikzlibrary{shapes,snakes}
\usetikzlibrary{arrows,%
                petri,%
                topaths}%
\usepackage{tkz-berge}
\usepackage[top=0.5in, bottom=0.7in, left=0.6in, right=0.6in]{geometry}
\linespread{1.3}

% \renewcommand\familydefault{\sfdefault}


% Stuff related to homework specific documents
\newcounter{MyTaskCounter}
\newcounter{MyTaskSectionCounter}
\newcommand{\tasksection}[1]{
	\stepcounter{MyTaskSectionCounter}
	\setcounter{MyTaskCounter}{0}
	\ifthenelse{\equal{#1}{}}{}{
	{\hfill\\[0.2in] \Large \textbf{\theMyTaskSectionCounter \enspace #1} \hfill\\[0.1in]}}
}

\newcommand{\task}[1]{
	\stepcounter{MyTaskCounter}
	\hfill\\[0.1in]
	\ifthenelse{\equal{\theMyTaskSectionCounter}{0}}{
	   \textbf{\large Задача №\theMyTaskCounter}
	}{
	   \textbf{\large Задача №\theMyTaskSectionCounter.\theMyTaskCounter}
	}
	\ifthenelse{\equal{#1}{}}{}{{\normalsize (#1)}}
	\hfill\\[0.05in]
}

% Math and algorithms

\makeatletter
\renewcommand{\ALG@name}{Алгоритм}
\renewcommand{\listalgorithmname}{Список алгроитмов}

\newenvironment{procedure}[1]
  {\renewcommand*{\ALG@name}{Процедура}
  \algorithm\renewcommand{\thealgorithm}{\thechapter.\arabic{algorithm} #1}}
  {\endalgorithm}

\makeatother

\algrenewcommand\algorithmicrequire{\textbf{Вход:}}
\algrenewcommand\algorithmicensure{\textbf{Выход:}}
\algnewcommand\True{\textbf{true}\space}
\algnewcommand\False{\textbf{false}\space}
\algnewcommand\And{\textbf{and}\space}

\newcommand{\xfor}[3]{#1 \textbf{from} #2 \textbf{to} #3}
\newcommand{\xassign}[2]{\State #1 $\leftarrow$ #2}
\newcommand{\xstate}[1]{\State #1}
\newcommand{\xreturn}[1]{\xstate{\textbf{return} #1}}

\DeclarePairedDelimiter\ceil{\lceil}{\rceil}
\DeclarePairedDelimiter\floor{\lfloor}{\rfloor}

\newcommand{\bigO}[1]{\mathcal{O}\left(#1\right)}

\newcommand{\xqed}{\hfill $\blacksquare$}

\newcommand{\code}[1]{\colorbox{gray!15}{\footnotesize\texttt{#1}}}

\title{Математическая логика. Домашнее задание №6}
\author{Горбунов Егор Алексеевич}

\begin{document}
\maketitle

\begin{task}[1]
	Закончите доказательство того, что интерпретация логики с $\land$ и $\lor$ в дистрибутивных решетках корректна.
	То есть нужно доказать, что если $\gamma \leq \varphi \lor \psi$, $\gamma \land \varphi \leq \chi$ и $\gamma \land \psi \leq \chi$, то $\gamma \leq \chi$.
\end{task}
\begin{solution}
Свойство дистрибутивности решётки:
\[
	x \land (y \lor z) = (x \land y) \lor (x \land z)
\]
Т.к. верно, что $\gamma \leq \varphi \lor \psi$, то $\gamma = \gamma \land (\varphi \lor \psi)$, далее по свойству дистрибутивности $\gamma = (\gamma \land \varphi) \lor (\gamma \land \psi)$. По определению инфимума:
\[
	\begin{array}{l}
		\gamma \land \varphi \leq \chi \\
		\gamma \land \psi \leq \chi
	\end{array}
	\Longrightarrow
	(\gamma \land \varphi) \lor (\gamma \land \psi) \leq \chi
\]
Но то, что слева от стрелочки, верно по условию, а тогда $\gamma \leq \chi$. \xqed
\end{solution}

\begin{task}[2]
	Докажите, что в любой решетке для любых элементов $x$, $\varphi$ и $\psi$ следующие утверждения эквивалентны:
	\begin{enumerate}
		\item \label{2:a} Для любого $\gamma$ если $\gamma \leq x$, то $\gamma \land \varphi \leq \psi$.
		\item \label{2:b} Для любого $\gamma$ если $\gamma \leq x$ и $\gamma \leq \varphi$, то $\gamma \leq \psi$.
		\item \label{2:c} $x \land \varphi \leq \psi$.
	\end{enumerate}
\end{task}
\begin{solution}
\begin{itemize}
	\item[\ref{2:a} $\Rightarrow$ \ref{2:b}] Имеем $\gamma \leq x$ и $\gamma \leq \varphi$, тогда по~\ref{2:a} верно, что $\gamma \land \varphi \leq \psi$, но $\gamma \leq \varphi \Leftrightarrow \gamma = \gamma \land \phi \leq \psi$, откуда $\gamma \leq \psi$ \xqed 
	\item[\ref{2:b} $\Rightarrow$ \ref{2:c}] Рассмотрим $\gamma = x \land \varphi$, тогда, т.к. \ref{2:b} верно, то сразу из того, что $x \land \varphi \leq x$ и $x \land \varphi \leq \varphi$ (по свойству инфимума) имеем $x \land \varphi \leq \psi$ \xqed
	\item[\ref{2:c} $\Rightarrow$ \ref{2:a}] Имеем $\gamma \leq x$ и $x \land \varphi \leq \psi$. Заметим, что:
	\begin{align*}
		\gamma \leq x &\Rightarrow \gamma = \gamma \land x \Rightarrow \gamma \land \varphi = (\gamma \land \varphi) \land x \\
					  &\Rightarrow (\gamma \land \varphi) = (\gamma \land \varphi) \land \varphi = (\gamma \land \varphi) \land (x \land \varphi)\\
					  &\Rightarrow \gamma \land \varphi \leq x \land \varphi
	\end{align*}
	Но т.к. по условию $x \land \varphi \leq \psi$, то $\gamma \land \varphi \leq \psi$ \xqed
\end{itemize}
\end{solution}

\begin{task}[3]
Пусть $\varphi$, $\psi$, $x$ и $x'$ -- элементы решетки. Допустим в ней верны следующие свойства:
	\begin{enumerate}
	\item \label{3:a} Для любого $\gamma$
	    \[ \gamma \leq x \Leftrightarrow \gamma \land \varphi \leq \psi \]
	\item \label{3:b} Для любого $\gamma$
	    \[ \gamma \leq x' \Leftrightarrow \gamma \land \varphi \leq \psi \]
	\end{enumerate}
Докажите, что тогда $x = x'$.
\end{task}
\begin{solution}
Рассмотрим $\gamma = x$, т.к. верно, что $x \leq x$, то по~\ref{3:a} имеем $x \land \varphi \leq \psi$, но тогда по стрелке в обратную сторону в~\ref{3:b} имеем $x \leq x'$. Теперь аналогичным образом рассматриваем $\gamma = x'$, и пользуясь сначала пунктом~\ref{3:b} вправо, а потом пунктом \ref{3:a} влево получаем, что $x' \leq x$. В силу антисимметричности $\leq$ получаем: $x = x'$. \xqed
\end{solution}

\begin{task}[4]
Покажите, что в любой алгебре Гейтинга $M$ верны следующие свойства:
	\begin{enumerate}
	\item В $M$ существует наибольший элемент, то есть элемент $\top$, удовлетворяющий условию, что $x \leq \top$ для любого $x$.
	\item Для любых $\varphi, \psi \in M$ верно $\varphi \leq \psi \Leftrightarrow (\varphi \to \psi) = \top$
	\end{enumerate}
\end{task}
\begin{solution}
\begin{enumerate}
\item В $M$ существует минимальный элемент $\bot$. Тогда введём $\top = \bot \rightarrow \bot$. Такой элемент существует по определению алгебры Гейтинга. Рассмотрим тогда любой $x \in M$. Очевидно, что $\gamma \land \bot = \bot$, т.к. $\bot$ --- минимальный. Т.е. $x \land \bot \leq \bot$, а значит в силу определения алгебры Гейтинга:
\[
	x \land \bot \leq \bot \Rightarrow x \leq (\bot \rightarrow \bot) = x \leq \top
\]
В силу того, что $x$ --- любое, то $\top$ --- максимальный элемент. \xqed
\item Пускай $\varphi \leq \psi$, тогда из определения максимального элемента получаем $\top \land \varphi = \varphi \leq \psi$, откуда ${\top \land \varphi \leq \psi}$. Тогда по определению алгебры Гейтинга, т.к. существует элемент $\varphi \rightarrow \psi$
\[
	\top \leq \varphi \rightarrow \psi \Rightarrow \varphi \rightarrow \psi = \top
\]
Теперь в обратную сторону: $\varphi \rightarrow \psi = \top \Rightarrow \varphi \rightarrow \psi \geq \top$, ну а тогда по свойству для стрелки (вправо) получаем, что $\varphi \rightarrow \psi = \top \Rightarrow \varphi \leq \psi$. \xqed
\end{enumerate}
\end{solution}

\begin{task}[5]
Докажите, что любая алгебра Гейтинга дистрибутивна.
\end{task}
\begin{solution}

\begin{itemize}
	\item Покажем, что $(x \land y) \lor (x \land z) \leq x \land (y \lor z)$. По свойствам супремума:
		\[
			\begin{array}{l}
				y \lor z \geq y \\
				y \lor z \geq z
			\end{array}
			\Rightarrow
			\begin{array}{l}
				x \land (y \lor z) \geq x \land y \\
				x \land (y \lor z) \geq x \land z
			\end{array}
			\Rightarrow
			x \land (y \lor z) \geq (x \land y) \lor (x \land z)
		\]
		Окей, показали.
	\item Покажем теперь обратное. Нужно туда-обратно применять свойство для импликации:
	\[
		\begin{array}{cl}
			x \land (y \lor z) \leq (x \land y) \lor (x \land z) & \\
			\Updownarrow & \text{(свойство импликации влево)}\\
			y \lor z \leq x \rightarrow (x \land y) \lor (x \land z) & \\
			\Updownarrow & \text{(свойство супремума)}\\
			\begin{cases}
				y \leq x \rightarrow (x \land y) \lor (x \land z)\\
				z \leq x \rightarrow (x \land y) \lor (x \land z)
			\end{cases} & \\
			\Updownarrow & \text{(свойство импликации вправо)}\\
			\begin{cases}
				x \land y \leq (x \land y) \lor (x \land z)\\
				x \land z \leq (x \land y) \lor (x \land z)
			\end{cases} & \text{(а это верно по свойству супремума)} \\
		\end{array}
	\]
	\xqed
\end{itemize}
\end{solution}

\end{document}
