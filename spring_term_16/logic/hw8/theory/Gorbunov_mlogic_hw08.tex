%!TEX program = xelatex

\documentclass[12pt, a4paper]{article}
\usepackage[utf8]{inputenc}
\usepackage[russian]{babel}
\usepackage{pscyr}

\usepackage{xifthen}
\usepackage{parskip}
\usepackage{hyperref}
\usepackage[top=0.7in, bottom=1in, left=0.6in, right=0.6in]{geometry}
\usepackage{setspace}

\usepackage{amsmath}
\usepackage{MnSymbol}
\usepackage{amsthm}
\usepackage{mathtools}

\usepackage{algorithm}
\usepackage[noend]{algpseudocode}



\linespread{1.2}
\setlength{\parskip}{0pt}

\renewcommand\familydefault{\sfdefault}


% Stuff related to homework specific documents
\newcounter{MyTaskCounter}
\newcounter{MyTaskSectionCounter}
\newcommand{\tasksection}[1]{
	\stepcounter{MyTaskSectionCounter}
	\setcounter{MyTaskCounter}{0}
	\ifthenelse{\equal{#1}{}}{}{
	{\hfill\\[0.2in] \Large \textbf{\theMyTaskSectionCounter \enspace #1} \hfill\\[0.1in]}}
}

\newcommand{\task}[1]{
	\stepcounter{MyTaskCounter}
	\hfill\\[0.1in]
	\ifthenelse{\equal{\theMyTaskSectionCounter}{0}}{
	   \textbf{\large Задача №\theMyTaskCounter}
	}{
	   \textbf{\large Задача №\theMyTaskSectionCounter.\theMyTaskCounter}
	}
	\ifthenelse{\equal{#1}{}}{}{{\normalsize (#1)}}
	\hfill\\[0.05in]
}

% Math and algorithms

\makeatletter
\renewcommand{\ALG@name}{Алгоритм}
\renewcommand{\listalgorithmname}{Список алгроитмов}

\newenvironment{procedure}[1]
  {\renewcommand*{\ALG@name}{Процедура}
  \algorithm\renewcommand{\thealgorithm}{\thechapter.\arabic{algorithm} #1}}
  {\endalgorithm}

\makeatother

\algrenewcommand\algorithmicrequire{\textbf{Вход:}}
\algrenewcommand\algorithmicensure{\textbf{Выход:}}
\algnewcommand\True{\textbf{true}\space}
\algnewcommand\False{\textbf{false}\space}
\algnewcommand\And{\textbf{and}\space}

\newcommand{\xfor}[3]{#1 \textbf{from} #2 \textbf{to} #3}
\newcommand{\xassign}[2]{\State #1 $\leftarrow$ #2}
\newcommand{\xstate}[1]{\State #1}
\newcommand{\xreturn}[1]{\xstate{\textbf{return} #1}}

\DeclarePairedDelimiter\ceil{\lceil}{\rceil}
\DeclarePairedDelimiter\floor{\lfloor}{\rfloor}

\newcommand{\bigO}[1]{\mathcal{O}\left(#1\right)}


\title{Математическая логика. Домашнее задание №8}
\author{Горбунов Егор Алексеевич}

\begin{document}
\maketitle

\begin{task}[1]
	Определите формулы, удовлетворяющие следующим описаниям. Для
	первых двух заданий мы предполагаем, что в сигнатуре есть преди-
	катный символ $\leq$.
	\begin{enumerate}
		\item $\exists x \leq t\ \psi$ (<<Существует $x$, меньше либо равный $t$, такой что верно $\psi$>>)
		\item $\forall x \leq t\ \psi$ (<<Для любого $x$, меньше либо равного $t$, верно $\psi$>>)
		\item <<Существует не менее двух элементов, удовлетворяющих $\varphi(x)$>>
		\item <<Существует ровно два элемента, удовлетворяющие $\varphi(x)$>>
		\item <<Существует по крайней мере один, но не более двух элементов, удовлетворяющих $\varphi(x)$>>
		\item <<Существует не более одного элемента, удовлетворяющего $\varphi(x)$>>
	\end{enumerate}
\end{task}
\begin{solution}
Сокращённо пишем вместо $\varphi[y/x] := \varphi(y)$, как я понимаю.
\begin{enumerate}
	\item $\exists x\ ((x \leq t) \land \psi)$
	\item $\forall x\ ((x \leq t) \to \psi)$
	\item $\exists x\ (\exists y\ ((x = y) \to \bot \land \varphi(x) \land \varphi(y)))$
	\item $\gamma :=$ $\exists x\ ((\exists y\ ((x = y) \to \bot \land \varphi(x) \land \varphi(y))) \land \forall z (\varphi(z) \to (x = z \lor y = z))$
	\item $\gamma \lor \exists x\ (\varphi(x) \land \forall y\ (\varphi(y) \to (x = y)))$
	\item $(\exists x\ (\varphi(x) \land \forall y\ (\varphi(y) \to (x = y))) \lor (\forall x\ (\varphi(x) \to \bot))$
\end{enumerate}
\end{solution}

\begin{task}[2]
	Напишите на хаскелле функцию, аналогичную конструкции $\bm{case}$ для пар, используя $fst$ и $snd$.
    Укажите ее тип (вам нужно будет использовать функции высшего порядка вместо расширения контекста).
    Реализуйте функции $fst'$ и $snd'$, эквивалентные обычным $fst$ и $snd$, через эту функцию.
\end{task}
\begin{solution}
\begin{lstlisting}
	caseI :: (a, b) -> (a -> b -> c) -> c 
	caseI p f = f (fst p) (snd p)

	fst' :: (a, b) -> a
	fst' p = caseI p const

	snd' :: (a, b) -> b
	snd' p = caseI p (flip const)
\end{lstlisting}
\end{solution}

\begin{task}[3]
	Пусть у нас есть несколько формул:
	\begin{enumerate}
		\item \label{it:no} $x \neq y$
		\item \label{it:e} $\exists x (x \neq y)$
		\item \label{it:a} $\forall x (x \neq y)$
		\item \label{it:ee} $\exists x \exists y (x \neq y)$
		\item \label{it:ea} $\exists x \forall y (x \neq y)$
		\item \label{it:ae} $\forall x \exists y (x \neq y)$
		\item \label{it:aa} $\forall x \forall y (x \neq y)$
	\end{enumerate}
	И несколько интерпретаций:
	\begin{align*}
	M_0 & = \varnothing \\
	M_1 & = \{ 7 \} \\
	M_2 & = \{ 13, 28 \}
	\end{align*}
	Какие из этих формул верны в каких моделях?
\end{task}
\begin{solution}
\begin{center}
\begin{tabular}{|c|c|c|c|}
	\hline
	            & $M_0$ & $M_1$ & $M_2$ \\
	\hline
	\ref{it:no} & \xtick  & \xcross & \xcross \\
	\ref{it:e}  & \xcross & \xcross & \xtick  \\
	\ref{it:a}  & \xtick  & \xcross & \xcross \\
	\ref{it:ee} & \xcross & \xcross & \xtick  \\
	\ref{it:ea} & \xcross & \xcross & \xcross \\
	\ref{it:ae} & \xtick  & \xcross & \xtick  \\
	\ref{it:aa} & \xtick  & \xcross & \xcross \\
	\hline
\end{tabular}
\end{center}
\end{solution}

\begin{task}[4]
Докажите, что формулы $\forall x \forall y (x \neq y)$ и $\neg \exists x\ \top$ эквивалентны,
написав лямбда терм типа $((\forall x \forall y (x \neq y)) \to \neg \exists x\ \top) \land ((\neg \exists x\ \top) \to \forall x \forall y (x \neq y))$.
\end{task}
\begin{solution}
\begin{enumerate}
	\item[$(\Rightarrow)$] Терм типа $(\forall x \forall y (x \neq y)) \to \neg \exists x\ \top)$:
	\[
		t_{\Rightarrow} := \lambda f.\xparen{\lambda p. f\ (fst\ p)\ (fst\ p)\ (refl\ (fst\ p))}
	\]
	Тут $f$ имеет тип $x \to y\ \to (x = y) \to \bot$, а $p$ имеет тип $(x, \bot \to \bot)$

	\item[$(\Leftarrow)$] Терм типа $(\neg \exists x\ \top) \to \forall x \forall y (x \neq y)$, тип можно переписать так: 
	\[
		((x, \bot \to \bot) \to \bot) \to (x \to y \to (x = y) \to \bot)
	\]
	Тогда искомый терм таков:
	\[
		t_{\Leftarrow} := \lambda g.\xparen{ \lambda a.\lambda b.\lambda e. g\ (a, leib(x, x=y, e, e))}
	\]
	У нас получается, что $a$ имеет тип $x$, $b$ имеет тип $y$, а $e$ имеет тип $x=y$, тогда, соответственно, можно считать, что $e$ имеет тип $x=y[x/x]$,
	а значит $leib(x, x=y, e, e)$ будет иметь тип $y=y$, что эквивалентно $\top = \bot \to \bot$
\end{enumerate}
Требуемый итоговый терм: $(t_{\Rightarrow}, t_{\Leftarrow})$ \xqed
\end{solution}

\end{document}
