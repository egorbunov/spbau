%!TEX program = xelatex

\documentclass[12pt, a4paper]{article}
\usepackage[utf8]{inputenc}
\usepackage[russian]{babel}
\usepackage{pscyr}
\usepackage{amssymb}
\usepackage{xifthen}
\usepackage{parskip}
\usepackage{hyperref}
\usepackage{setspace}

\usepackage{graphicx}
\usepackage{xcolor}
\usepackage{amsmath}
\usepackage{MnSymbol}
\usepackage{amsthm}
\usepackage{mathtools}
\usepackage{algorithm}
\usepackage[noend]{algpseudocode}
\usepackage[shortlabels]{enumitem}
                    \setlist[enumerate, 1]{1\textsuperscript{o}}
\usepackage{subfig}
\usepackage{tikz}
\usepackage{tikz,fullpage}
\usetikzlibrary{shapes,snakes}
\usetikzlibrary{arrows,%
                petri,%
                topaths}%
\usepackage{tkz-berge}
\usepackage[top=0.5in, bottom=0.7in, left=0.6in, right=0.6in]{geometry}
\linespread{1.3}

% \renewcommand\familydefault{\sfdefault}


% Stuff related to homework specific documents
\newcounter{MyTaskCounter}
\newcounter{MyTaskSectionCounter}
\newcommand{\tasksection}[1]{
	\stepcounter{MyTaskSectionCounter}
	\setcounter{MyTaskCounter}{0}
	\ifthenelse{\equal{#1}{}}{}{
	{\hfill\\[0.2in] \Large \textbf{\theMyTaskSectionCounter \enspace #1} \hfill\\[0.1in]}}
}

\newcommand{\task}[1]{
	\stepcounter{MyTaskCounter}
	\hfill\\[0.1in]
	\ifthenelse{\equal{\theMyTaskSectionCounter}{0}}{
	   \textbf{\large Задача №\theMyTaskCounter}
	}{
	   \textbf{\large Задача №\theMyTaskSectionCounter.\theMyTaskCounter}
	}
	\ifthenelse{\equal{#1}{}}{}{{\normalsize (#1)}}
	\hfill\\[0.05in]
}

% Math and algorithms

\makeatletter
\renewcommand{\ALG@name}{Алгоритм}
\renewcommand{\listalgorithmname}{Список алгроитмов}

\newenvironment{procedure}[1]
  {\renewcommand*{\ALG@name}{Процедура}
  \algorithm\renewcommand{\thealgorithm}{\thechapter.\arabic{algorithm} #1}}
  {\endalgorithm}

\makeatother

\algrenewcommand\algorithmicrequire{\textbf{Вход:}}
\algrenewcommand\algorithmicensure{\textbf{Выход:}}
\algnewcommand\True{\textbf{true}\space}
\algnewcommand\False{\textbf{false}\space}
\algnewcommand\And{\textbf{and}\space}

\newcommand{\xfor}[3]{#1 \textbf{from} #2 \textbf{to} #3}
\newcommand{\xassign}[2]{\State #1 $\leftarrow$ #2}
\newcommand{\xstate}[1]{\State #1}
\newcommand{\xreturn}[1]{\xstate{\textbf{return} #1}}

\DeclarePairedDelimiter\ceil{\lceil}{\rceil}
\DeclarePairedDelimiter\floor{\lfloor}{\rfloor}

\newcommand{\bigO}[1]{\mathcal{O}\left(#1\right)}

\newcommand{\xqed}{\hfill $\blacksquare$}

\newcommand{\code}[1]{\colorbox{gray!15}{\footnotesize\texttt{#1}}}

\title{Математическая логика. Домашнее задание №10}
\author{Горбунов Егор Алексеевич}

\begin{document}
\maketitle

\begin{task}[1]
Докажите, что $L_1^{lem}$ является консервативным расширением $L_0^{lem}$.
\end{task}
\begin{solution}
У нас в обеих логических системах есть исключённое третье. Мы показывали, что исключённое третье эквивалентно закону двойного отрицанию:
\[
	 \phi \lor \neg \phi \Leftrightarrow \neg \neg \phi \to \phi 
\]
Так же мы знаем, что в интуиционистской логике верно правило де Моргана:
\[
	\neg (\phi \land \psi) \leftrightarrow (\neg \phi \lor \neg \psi)
\]
Таким образом видим, что у нас при доказательстве появляется возможность использовать:
\begin{center}
\AxiomC{$\Gamma \vdash \neg \neg \phi$}
\UnaryInfC{$\Gamma \vdash \phi$}
\DisplayProof
\end{center}

Теперь пусть есть формула из $L_0^{lem}$, которая доказуема в $L_1^{lem}$. Тогда, чтобы получить её доказательство в $L_0^{lem}$ нужно сделать замены:
\begin{itemize}
	\item 
	\AxiomC{$\Gamma \vdash \varphi \land \psi$}
	\RightLabel{ $\land E_1$}
	\UnaryInfC{$\Gamma \vdash \varphi$}
	\DisplayProof
	\qquad 
	на
	\qquad
	\AxiomC{$\Gamma \vdash \neg (\neg \phi \lor \neg \psi)$}
	\UnaryInfC{$\Gamma \vdash \varphi$}
	\DisplayProof

	\item 
	\AxiomC{$\Gamma \vdash \varphi \land \psi$}
	\RightLabel{ $\land E_2$}
	\UnaryInfC{$\Gamma \vdash \psi$}
	\DisplayProof
	\qquad 
	на
	\qquad
	\AxiomC{$\Gamma \vdash \neg (\neg \phi \lor \neg \psi)$}
	\UnaryInfC{$\Gamma \vdash \psi$}
	\DisplayProof
\end{itemize}
Правило введения $\land$ заменится на 
\AxiomC{$\Gamma \vdash \varphi$}
\AxiomC{$\Gamma \vdash \psi$}
\RightLabel{ $\land I'$}
\BinaryInfC{$\Gamma \vdash \neg (\neg \phi \lor \neg \psi)$}
\DisplayProof\\
Получится доказательство в $L_0^{lem}$
\end{solution}

\begin{task}[2]
	Если в $PRCPA$ можно определять функции при помощи паттерн матчинга на нескольких аргументах сразу, то мы легко можем определить функции $min$ и $max$:
	\begin{align*}
	& min : \mathbb{N} \times \mathbb{N} \to \mathbb{N} \\
	& min(0,y) = 0 \\
	& min(S(x),0) = 0 \\
	& min(S(x),S(y)) = S(min(x,y)) \\
	& \\
	& max : \mathbb{N} \times \mathbb{N} \to \mathbb{N} \\
	& max(0,y) = y \\
	& max(S(x),S(y)) = S(max(x,y))
	\end{align*}
	\begin{enumerate}
	\item Определите их, используя только базовый вариант паттерн матчинга как в лекциях.
	\item Докажите, что $\forall x \forall y\ (min(x,y) \leq max(x,y))$, где $a \leq b$ означает $\exists c\ (a + c = b)$.
	\end{enumerate}
\end{task}
\begin{solution}
\begin{enumerate}
	\item Определяем:
		\begin{align*}
		    & P : \mathbb{N} \to \mathbb{N} \\
		    & P(0) = 0\\
		    & P(S(x)) = x\\
		    & \\
			& - : \mathbb{N} \times \mathbb{N} \to \mathbb{N}\\
			& 0 - y = 0\\
			& x - 0 = x\\
			& x - S(y) = P(x - y)\\
			& \\
		    & sign : \mathbb{N} \to \mathbb{N} \\
		    & sign(0) = 0\\
		    & sign(S(x)) = 1\\
		\end{align*}
		\begin{align*}
			& min : \mathbb{N} \times \mathbb{N} \to \mathbb{N} \\
			& min(0, y) = 0 \\
			& min(S(x), y) = min(x, y) + sign(y - x)\\
			& \\
			& max : \mathbb{N} \times \mathbb{N} \to \mathbb{N} \\
			& max(0,y) = y \\
			& max(S(x), y) = max(x, y) + sign(S(x) - y)
		\end{align*}
	\item Индукция по $x$. При $x = 0$: $min(0, y) = 0$, $max(0, y) = y$, знаем, что $0 + y = y$, т.е. нашли такой $b$, что $0 + b = y$, а значит $0 \leq y$. Пускай теперь утверждение верно для $x, y$, покажем для $S(x), y$:
	\begin{align*}
		min(S(x), y) &= min(x, y) + sign(y - x)\\
		max(S(x), y) &= max(x, y) + sign(S(x) - y) = min(x, y) + c + sign(S(x) - y)
	\end{align*}
	Нужно показать, что $sign(y - x) \leq c + sign(S(x) - y)$. Легко видеть, что если $S(x) > y$, то $sign(S(x) - y) = 1$, а $sign(y - x) = 0$, т.е. если взять $c' = c + sign(S(x) - y)$, то $max(S(x), y) = min(S(x), y) + c'$. 

	Теперь если $S(x) \leq y$ ($x < y$), то:
	\begin{align*}
		min(S(x), y) &= min(x, y) + sign(y - x) = x + 1\\
		max(S(x), y) &= max(x, y) + sign(S(x) - y) = y\\
		max(S(x), y) &= x + c \qquad (\text{по предположению})
	\end{align*}
	Таким образом $x + c = y$, но т.к. $x < y$, то $c > 0$, а значит $x + 1 \leq x + c$, т.е. $min(S(x), y) \leq max(S(x), y)$. \xqed
\end{enumerate}
\end{solution}

\begin{task}[3]
	Пусть в $PRCPA$ у нас есть функции $+$ и $+'$, определенные следующим образом:
	\begin{align*}
	& + : \mathbb{N} \times \mathbb{N} \to \mathbb{N} \\
	& 0 + y = y \\
	& S(x) + y = S(x + y) \\
	& \\
	& +' : \mathbb{N} \times \mathbb{N} \to \mathbb{N} \\
	& x +' 0 = x \\
	& x +' S(y) = S(x +' y)
	\end{align*}
	Докажите, что $\forall x \forall y\ (x + y = x +' y)$
\end{task}
\begin{solution}
Двойная индукция. Сначала по $x$, потом по $y$.
\begin{itemize}
\item Для $x = 0$.
	\begin{itemize}
		\item Для $y = 0$. $0 + 0 = 0$, $0 +' 0 = 0$, т.е. $0 + 0 = 0 +' 0$
		\item Пускай верно для $y$, тогда: $0 + S(y) = S(y)$, $0 +' S(y) = S(0 +' y) =^{\text{предположение}} S(0 + y) = S(y)$, т.е. $0 + S(y) = 0 +' S(y)$, показали.
	\end{itemize}
\item Пускай верно для $x$, тогда покажем для $S(x)$
	\begin{itemize}
		\item $y = 0$. $S(x) + 0 = S(x + 0) = S(x)$, $S(x) +' 0 = S(x)$, т.е. $S(x) + 0 = S(x) +' 0$.
		\item Пускай верно для $y$, покажем для $S(y)$: 
		\begin{align*}
			S(x) + S(y) &= S(x + S(y)) =^{\text{предположение для $x$}} = S(x +' S(y)) = S(S(x +' y))\\
			S(x) +' S(y) &= S(S(x) +' y) =^{\text{предположение для $y$}} = S(S(x) + y) = S(S(x + y))
 		\end{align*}
 		Но по предположению $x + y = x +' y$, т.е. $S(x)+S(y) = S(x) +' S(y)$ \xqed
	\end{itemize}
\end{itemize}
\end{solution}

\begin{task}[4]
	Докажите в $CPA$, что $\forall n\ (2^n \leq ack\,n\,n)$, где $a \leq b$ означает $\exists c\ (a + c = b)$ и функции $2^{(-)}$ и $ack$ определены следующим образом:
	\begin{align*}
	& 2^{(-)} : \mathbb{N} \to \mathbb{N} \\
	& 2^0 = S(0) \\
	& 2^{S(n)} = 2 \cdot 2^n \\
	& \\
	& ack : \mathbb{N} \to \mathbb{N} \to \mathbb{N} \\
	& ack\,0\,n = S(n) \\
	& ack\,(S\,m)\,0 = ack\,m\,(S\,0) \\
	& ack\,(S\,m)\,(S\,n) = ack\,m\,(ack\,(S\,m)\,n)
	\end{align*}
\end{task}
\begin{solution}
\end{solution}
\end{document}
