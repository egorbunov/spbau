%!TEX program = xelatex

\documentclass[12pt, a4paper]{article}
\usepackage[utf8]{inputenc}
\usepackage[russian]{babel}
\usepackage{pscyr}
\usepackage{amssymb}
\usepackage{xifthen}
\usepackage{parskip}
\usepackage{hyperref}
\usepackage{setspace}

\usepackage{graphicx}
\usepackage{xcolor}
\usepackage{amsmath}
\usepackage{MnSymbol}
\usepackage{amsthm}
\usepackage{mathtools}
\usepackage{algorithm}
\usepackage[noend]{algpseudocode}
\usepackage[shortlabels]{enumitem}
                    \setlist[enumerate, 1]{1\textsuperscript{o}}
\usepackage{subfig}
\usepackage{tikz}
\usepackage{tikz,fullpage}
\usetikzlibrary{shapes,snakes}
\usetikzlibrary{arrows,%
                petri,%
                topaths}%
\usepackage{tkz-berge}
\usepackage[top=0.5in, bottom=0.7in, left=0.6in, right=0.6in]{geometry}
\linespread{1.3}

% \renewcommand\familydefault{\sfdefault}


% Stuff related to homework specific documents
\newcounter{MyTaskCounter}
\newcounter{MyTaskSectionCounter}
\newcommand{\tasksection}[1]{
	\stepcounter{MyTaskSectionCounter}
	\setcounter{MyTaskCounter}{0}
	\ifthenelse{\equal{#1}{}}{}{
	{\hfill\\[0.2in] \Large \textbf{\theMyTaskSectionCounter \enspace #1} \hfill\\[0.1in]}}
}

\newcommand{\task}[1]{
	\stepcounter{MyTaskCounter}
	\hfill\\[0.1in]
	\ifthenelse{\equal{\theMyTaskSectionCounter}{0}}{
	   \textbf{\large Задача №\theMyTaskCounter}
	}{
	   \textbf{\large Задача №\theMyTaskSectionCounter.\theMyTaskCounter}
	}
	\ifthenelse{\equal{#1}{}}{}{{\normalsize (#1)}}
	\hfill\\[0.05in]
}

% Math and algorithms

\makeatletter
\renewcommand{\ALG@name}{Алгоритм}
\renewcommand{\listalgorithmname}{Список алгроитмов}

\newenvironment{procedure}[1]
  {\renewcommand*{\ALG@name}{Процедура}
  \algorithm\renewcommand{\thealgorithm}{\thechapter.\arabic{algorithm} #1}}
  {\endalgorithm}

\makeatother

\algrenewcommand\algorithmicrequire{\textbf{Вход:}}
\algrenewcommand\algorithmicensure{\textbf{Выход:}}
\algnewcommand\True{\textbf{true}\space}
\algnewcommand\False{\textbf{false}\space}
\algnewcommand\And{\textbf{and}\space}

\newcommand{\xfor}[3]{#1 \textbf{from} #2 \textbf{to} #3}
\newcommand{\xassign}[2]{\State #1 $\leftarrow$ #2}
\newcommand{\xstate}[1]{\State #1}
\newcommand{\xreturn}[1]{\xstate{\textbf{return} #1}}

\DeclarePairedDelimiter\ceil{\lceil}{\rceil}
\DeclarePairedDelimiter\floor{\lfloor}{\rfloor}

\newcommand{\bigO}[1]{\mathcal{O}\left(#1\right)}

\newcommand{\xqed}{\hfill $\blacksquare$}

\newcommand{\code}[1]{\colorbox{gray!15}{\footnotesize\texttt{#1}}}

\title{Математическая логика. Домашнее задание №7}
\author{Горбунов Егор Алексеевич}

\begin{document}
\maketitle

\begin{task}[1]
Докажите, что формула $((P \to Q) \to P) \to P$ невыводима, показав, что ее выводимость влечет выводимость исключенного третьего.
\end{task}
\begin{solution}
Обозначим:
\begin{center}
\AxiomC{}
\RightLabel{(x)}
\UnaryInfC{$\vdash ((P \to Q) \to P) \to P$}
\DisplayProof
\end{center}

Теперь:

\begin{center}

\AxiomC{}
\RightLabel{($x$)}
\UnaryInfC{$\vdash ((P \lor \neg P \to \bot) \to P \lor \neg P) \to (P \lor \neg P)$}
\AxiomC{см. ниже}
\UnaryInfC{$(P \lor \neg P \to \bot) \to P \lor \neg P$}
\RightLabel{($\to E$)}
\BinaryInfC{$\vdash P \lor \neg P $}
\DisplayProof

\vspace{20pt}

\AxiomC{}
\RightLabel{($var$)}
\UnaryInfC{$P, (P \lor \neg P \to \bot) \vdash P \lor \neg P \to \bot$}
\AxiomC{}
\RightLabel{($var$)}
\UnaryInfC{$P, (P \lor \neg P \to \bot) \vdash P$}
\RightLabel{($\lor I_1$)}
\UnaryInfC{$P, (P \lor \neg P \to \bot) \vdash P \lor \neg P$}
\RightLabel{($\to E$)}
\BinaryInfC{$P, (P \lor \neg P \to \bot) \vdash \bot$}
\RightLabel{($\to I$)}
\UnaryInfC{$(P \lor \neg P \to \bot) \vdash P \to \bot$}
\RightLabel{($\to I$ и $\lor I_2$)}
\UnaryInfC{$(P \lor \neg P \to \bot) \to P \lor \neg P$}
\DisplayProof
\end{center}
\vspace{20pt}
Видим, что у нас получилось вывести $\vdash P \lor \neg P $ воспользовавшись тем, что $\vdash ((P \to Q) \to P) \to P$ выводим (то, что мы обозначили за ($x$)). При использовании ($x$) мы совершили подстановку $\xbrack{(P \lor \neg P / P, \bot / Q)}$. 
\\\xqed
\end{solution}

\begin{task}[2]
	Докажите, что в интуиционистской логике следующие правила вывода эквивалентны:
	\begin{center}
	\AxiomC{}
	\RightLabel{, (lem)}
	\UnaryInfC{$\Gamma \vdash \varphi \lor \neg \varphi$}
	\DisplayProof
	\qquad
	\AxiomC{}
	\RightLabel{, (dne)}
	\UnaryInfC{$\Gamma \vdash \neg \neg \varphi \to \varphi$}
	\DisplayProof
	\end{center}
	Для этого достаточно показать, что имея правило (lem), заключение правила (dne) выводимо, и наоборот.
\end{task}

\begin{solution}

\begin{itemize}
\item $dne \Rightarrow lem$. Вот дерево вывода, доказывающее это (в интуиционистской логике):

\begin{center}
	\AxiomC{}
	\RightLabel{$(dne)$}
	\UnaryInfC{$\vdash \neg\neg (\phi \lor \neg \phi) \to (\phi \lor \neg \phi)$}
	\AxiomC{см. ниже}
	\UnaryInfC{$\vdash \neg\neg (\phi \lor \neg \phi)$}
	\RightLabel{$\to E$}
	\BinaryInfC{$\vdash \phi \lor \neg \phi $}
	\DisplayProof

	\vspace{20pt}

	\AxiomC{}
	\RightLabel{($var$)}
	\UnaryInfC{$\neg (\phi \lor \neg \phi) \vdash \neg (\phi \lor \neg \phi)$}
	\AxiomC{}
	\RightLabel{($var$)}
	\UnaryInfC{$\neg (\phi \lor \neg \phi), \phi \vdash \neg (\phi \lor \neg \phi)$}
	\AxiomC{}
	\RightLabel{($var$)}
	\UnaryInfC{$\Gamma, \phi \vdash \phi$}
	\RightLabel{$\lor I_1$}
	\UnaryInfC{$\neg (\phi \lor \neg \phi), \phi \vdash \phi \lor \neg \phi$}
	\RightLabel{$\to E$}
	\BinaryInfC{$\neg (\phi \lor \neg \phi), \phi \vdash \bot$}
	\RightLabel{$\to I$}
	\UnaryInfC{$\neg (\phi \lor \neg \phi) \vdash \neg \phi$}
	\RightLabel{$\lor I_2$}
	\UnaryInfC{$\neg (\phi \lor \neg \phi) \vdash \phi \lor \neg \phi$}
	\RightLabel{$\to E$}
	\BinaryInfC{$\neg (\phi \lor \neg \phi) \vdash \bot$}
	\RightLabel{$\to I$}
	\UnaryInfC{$\vdash \neg\neg (\phi \lor \neg \phi)$}
	\DisplayProof

\end{center}

\item $lem \Rightarrow dme$. Т.к. мы предполагаем исключённое третье, то можно работать в исчислении Генцена! А тогда:
\begin{center}
	\AxiomC{}
	\RightLabel{($lem$)}
	\UnaryInfC{$\vdash \neg \phi \lor \phi$}
	\UnaryInfC{$\vdash \neg \phi, \phi$}
	\AxiomC{}
	\RightLabel{$\bot L$}
	\UnaryInfC{$\bot \vdash \phi$}
	\RightLabel{$\to L$}
	\BinaryInfC{$\neg \phi \to \bot \vdash \phi$}
	\RightLabel{$\to R$}
	\UnaryInfC{$\vdash \neg \neg \phi \to \phi$}
	\DisplayProof
\end{center}
\end{itemize}
\xqed
\end{solution}

\begin{task}[3]
	Формула $\varphi \land \psi$ доказуема тогда и только тогда когда доказуемы $\varphi$ и $\psi$.
    Верно ли аналогичное утверждение для $\varphi \lor \psi$?
    То есть, правда ли что $\varphi \lor \psi$ доказуема тогда и только тогда когда доказуема либо $\varphi$, либо $\psi$?
    В классической логике это не верно.
    Действительно, $P \lor \neg P$ доказуема, но ни $P$, ни $\neg P$ не доказуемы.
    Докажите, что в интуиционистской логике это свойство выполнено.
    
    Hint: используйте в качестве доказательств формул лямбда термы и примените нормализацию.
\end{task}
\begin{solution}
\begin{itemize}
	\item[$\Rightarrow$] Тут нам нужно предоставить терм с типом $\varphi \lor \psi \rightarrow (\varphi \lor \psi  \rightarrow \varphi) \lor (\varphi \lor \psi  \rightarrow \psi)$. Вот он:
	\[
		\lambda e.\mathbf{case}\ e\ \mathbf{of} \{ Left(p) \to Left\ (\lambda x.p); Right(q) \to Right\ (\lambda x.q)\}
	\]

	\item[$\Leftarrow$] Тут нам нужно сконструировать термы с типом $\varphi \lor \psi$ имея $\varphi$ или $\psi$. Ну, это довольно очевдно, два случая:
	\begin{align*}
		\lambda p. Left\ p\\
		\lambda q. Right\ q\\
	\end{align*}

\end{itemize}
\xqed
\end{solution}

\begin{task}[4]
Докажите в классической логике формулу $\neg (\varphi \land \psi) \to \neg \varphi \lor \neg \psi$
Напишите лямбда терм, доказывающий эту формулу.
\end{task}
\begin{solution}

\begin{center}
\AxiomC{тут ещё применяем $\lor R$}
\UnaryInfC{$\varphi, \psi \vdash \varphi \land \psi, \bot$}
\AxiomC{}
\UnaryInfC{$\bot, \varphi, \psi \vdash \bot$}
\RightLabel{$\to L$}
\BinaryInfC{$\varphi \land \psi \to \bot, \varphi, \psi \vdash \bot$}
\RightLabel{дважды $\to R$}
\UnaryInfC{$\varphi \land \psi \to \bot \vdash \varphi \to \bot, \psi \to \bot$}
\RightLabel{$\lor R$}
\UnaryInfC{$\neg (\varphi \land \psi) \vdash \neg \varphi \lor \neg \psi$}
\RightLabel{$\to R$}
\UnaryInfC{$\vdash \neg (\varphi \land \psi) \to \neg \varphi \lor \neg \psi$}
\DisplayProof
\end{center}
\xqed
\end{solution}

\begin{task}[5]
Докажите в интуиционистской логике формулу $(((P \to Q) \lor (Q \to P)) \to P) \to P$.
    Напишите лямбда терм, доказывающий эту формулу.
\end{task}
\begin{solution}
\end{solution}

\end{document}
