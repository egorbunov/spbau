\documentclass[12pt, a4paper]{article}
\usepackage[utf8]{inputenc}
\usepackage[russian]{babel}
\usepackage{pscyr}
\usepackage{amssymb}
\usepackage{xifthen}
\usepackage{parskip}
\usepackage{hyperref}
\usepackage{setspace}

\usepackage{graphicx}
\usepackage{xcolor}
\usepackage{amsmath}
\usepackage{MnSymbol}
\usepackage{amsthm}
\usepackage{mathtools}
\usepackage{algorithm}
\usepackage[noend]{algpseudocode}
\usepackage[shortlabels]{enumitem}
                    \setlist[enumerate, 1]{1\textsuperscript{o}}
\usepackage{subfig}
\usepackage{tikz}
\usepackage{tikz,fullpage}
\usetikzlibrary{shapes,snakes}
\usetikzlibrary{arrows,%
                petri,%
                topaths}%
\usepackage{tkz-berge}
\usepackage[top=0.5in, bottom=0.7in, left=0.6in, right=0.6in]{geometry}
\linespread{1.3}

% \renewcommand\familydefault{\sfdefault}


% Stuff related to homework specific documents
\newcounter{MyTaskCounter}
\newcounter{MyTaskSectionCounter}
\newcommand{\tasksection}[1]{
	\stepcounter{MyTaskSectionCounter}
	\setcounter{MyTaskCounter}{0}
	\ifthenelse{\equal{#1}{}}{}{
	{\hfill\\[0.2in] \Large \textbf{\theMyTaskSectionCounter \enspace #1} \hfill\\[0.1in]}}
}

\newcommand{\task}[1]{
	\stepcounter{MyTaskCounter}
	\hfill\\[0.1in]
	\ifthenelse{\equal{\theMyTaskSectionCounter}{0}}{
	   \textbf{\large Задача №\theMyTaskCounter}
	}{
	   \textbf{\large Задача №\theMyTaskSectionCounter.\theMyTaskCounter}
	}
	\ifthenelse{\equal{#1}{}}{}{{\normalsize (#1)}}
	\hfill\\[0.05in]
}

% Math and algorithms

\makeatletter
\renewcommand{\ALG@name}{Алгоритм}
\renewcommand{\listalgorithmname}{Список алгроитмов}

\newenvironment{procedure}[1]
  {\renewcommand*{\ALG@name}{Процедура}
  \algorithm\renewcommand{\thealgorithm}{\thechapter.\arabic{algorithm} #1}}
  {\endalgorithm}

\makeatother

\algrenewcommand\algorithmicrequire{\textbf{Вход:}}
\algrenewcommand\algorithmicensure{\textbf{Выход:}}
\algnewcommand\True{\textbf{true}\space}
\algnewcommand\False{\textbf{false}\space}
\algnewcommand\And{\textbf{and}\space}

\newcommand{\xfor}[3]{#1 \textbf{from} #2 \textbf{to} #3}
\newcommand{\xassign}[2]{\State #1 $\leftarrow$ #2}
\newcommand{\xstate}[1]{\State #1}
\newcommand{\xreturn}[1]{\xstate{\textbf{return} #1}}

\DeclarePairedDelimiter\ceil{\lceil}{\rceil}
\DeclarePairedDelimiter\floor{\lfloor}{\rfloor}

\newcommand{\bigO}[1]{\mathcal{O}\left(#1\right)}

\newcommand{\xqed}{\hfill $\blacksquare$}

\newcommand{\code}[1]{\colorbox{gray!15}{\footnotesize\texttt{#1}}}
\title{Математическая логика. Домашнее задание №4}
\author{Горбунов Егор Алексеевич}

\begin{document}
\maketitle

\begin{task}[1]
Опишите 2-сортную сигнатуру и теорию коммутативных колец с единицей и модулей над ними (определение этих понятий легко найти в интернете).
\end{task} 
\begin{solution}
Алгебраическая сигнатура $(\mathcal{S}, \mathcal{F})$:
\[
	\mathcal{S} = \xbrace{R, M},\
	\mathcal{F} = 
		\xbrace{
			\begin{array}{l}
				  *_R: R\times R \xret R, \\
	              +:R\times R \xret R, \\
	              neg:R\xret R,\\
	              0:R,\\
	              1_R:R
	        \end{array}
	    }
	    \cup
		\xbrace{
			\begin{array}{l}
				  *_M: M\times M \xret M,\\
	              1_M:M,\\
	              inv:M\xret M
	        \end{array}
	    }
	    \cup
		\xbrace{
			\begin{array}{l}
				  *_{RM}: R\times M \xret M
	        \end{array}
	    }    
\]
Теория.
Коммутативное кольцо с 1:
\begin{equation*}
\begin{split}
	x + y &= y + x\\
	x + (y + z) &= (x + y) + z\\
	0 + x &= x\\
	neg(x) + x &= 0\\
	(x *_R y) *_R z &= x *_R (y *_R z)\\
	x *_R y &= y *_R x\\
	1_R *_R x &= x\\
	x *_R (y + z) &= x *_R y + x *_R z\\
\end{split}
\end{equation*}
Абелева группа:
\begin{equation*}
\begin{split}
	(x *_M y) *_M z &= x *_M (y *_M z)\\
	x *_M y &= y *_M x\\
	1_M *_M x &= x\\
	inv(x) *_M x &= 1_M\\
\end{split}
\end{equation*}
Модуль над кольцом:
\begin{equation*}
\begin{split}
	(x *_R y) *_{RM} m &= x *_{RM} (y *_{RM} m)\\
	x *_{RM} (a *_M b) &= (x *_{RM} a) *_M (x *_{RM} b)\\
	(x + y) *_{RM} m &= (x *_{RM} m) *_M (y *_{RM} m)
\end{split}
\end{equation*}
\xqed
\end{solution}

\begin{task}[2]
Рассмотрим сигнатуру $(\{N\}, \{ 0 : N, S : N \to N, + : N \times N \to N \})$.
    Рассмотрим следующую теорию:
\begin{align*}
0 + y & = y \\
S(x) + y & = S(x + y)
\end{align*}
Докажите, что следующие формулы невыводимы в этой теории
\begin{itemize}
\item $(x + y) + z = x + (y + z)$
\item $x + y = y + x$
\end{itemize}
\end{task}

\begin{solution}
Рассмотрим такую интерпретацию $M$ данной сигнатуры:
\[
	(\mathbb{R}, 0, id, /) = \xintrp{N} = \mathbb{R}, \xintrp{0} = 0, \xintrp{S} = id, \xintrp{+} = /
\]
Для этой интерпретации аксиомы выполняются, т.к. для $\forall x, y \in \mathbb{R}$ верно, что $0 + y = y$ и $id(x) / y = x / y = id(x / y)$, т.е. $M$ --- модель данной теории.

Теперь рассмотрим следующее означивание: $\rho(x) = 1, \rho(y) = 2, \rho(z) = 3$. Тогда:
\begin{align*}
	\xintrp{(x + y) + z}_{\rho} = (1 / 2) / 3 = \frac{1}{6} &\neq \frac{3}{2} = 1 / (2 / 3) = \xintrp{x + (y + z)}_{\rho}  \\
	\xintrp{x+y}_{\rho} = 1 / 2 = \frac{1}{2} &\neq 2 = 2 / 1 = \xintrp{y+x}_{\rho}
\end{align*}
Получили, что для выбранного означивания $\xintrp{(x + y) + z}_{\rho} \neq \xintrp{x + (y + z)}_{\rho}$ и $\xintrp{x+y}_{\rho} \neq \xintrp{y+x}_{\rho}$, а значит формулы невыводимы в теории. \xqed
\end{solution}
\newpage
\begin{task}[3]
Рассмотрим сигнатуру
\[ (\{D\}, \{ * : D \times D \to D, 1 : D, f : D \to D, g : D \to D, i_1 : D \to D, i_2 : D \to D \}) \]
Рассмотрим следующую теорию в ней:
\begin{align*}
(x * y) * z & = x * (y * z) \\
x * 1 & = x \\
1 * x & = x \\
f(f(x)) & = f(x) \\
g(g(x)) & = g(x) \\
f(g(x)) & = g(f(x)) \\
i_1(f(x)) * g(x) & = 1 \\
f(x) * i_2(g(x)) & = 1
\end{align*}
Какие из следующих утверждений являются теоремами этой теории? Докажите это.
\begin{enumerate}
\item $i_1(x) = i_2(x)$ \label{task3:t1}
\item $i_1(x) * x = 1$  \label{task3:t2}
\item $f(x) = g(x)$     \label{task3:t3}
\item $f(x) = x$        \label{task3:t4}
\end{enumerate}
При доказательстве выводимости можно опускать очевидные шаги, такие как применения ассоциативности и аксиом $1 * x = x$ и $x * 1 = x$.
\end{task}
\begin{solution}
Рассмотрим такую интерпретацию сигнатуры:
\[
	\xparen{ \mathbb{R},\ *,\ 1,\ f(x) = 1,\ g(x) = 1,\ i_1(x) = x,\ i_2(x) = x^2}
\]
Эта интерпретация --- модель, т.к. первые 3 аксиомы выполняются по свойству операции умножения над $\mathbb{R}$ и далее для любой означивающей функции будет верно:
\begin{align*}
	\xintrp{f(f(x))}  = 1 &= \xintrp{f(x)}  \\
	\xintrp{g(g(x))}  = 1 &= \xintrp{g(x)}  \\
	\xintrp{f(g(x))}  = 1 &= \xintrp{g(f(x))}  \\
	\xintrp{i_1(f(x)) * g(x)} = \xintrp{i_1}(1) * 1 = 1 * 1 & = \xintrp{1} \\
	\xintrp{f(x) * i_2(g(x))} = 1 * \xintrp{i_2}(1) = 1 * 1^2 &= \xintrp{1}
\end{align*}
Выберем означивающую функцию $\rho{(x)} = 2$, тогда:
\begin{align*}
	\xintrp{i_1(x)} = 2 &\neq 2^2 = \xintrp{i_2(x)} \\
	\xintrp{i_1(x) * x} = 2 * 2 &\neq 1 = \xintrp{1} \\
	\xintrp{f(x)} = 1 &\neq 2 = \xintrp{x}
\end{align*}
Таким образом утверждения~\ref{task3:t1}, \ref{task3:t2}, \ref{task3:t4} не являются теоремами в данной теории.


Осталось разобраться с утверждением~\ref{task3:t3}. Не вышло у меня.
\end{solution}

\begin{task}[4]
Рассмотрим сигнатуру 
\[ (\{D\}, \{ * : D \times D \to D, + : D \times D \to D, 1 : D, 0 : D, - : D \to D \}) \].
    Теория колец с единицей выглядит следующим образом:
\begin{align*}
(x + y) + z & = x + (y + z) \\
x + 0 & = x \\
0 + x & = x \\
x + y & = y + x \\
x + -x & = 0 \\
(x * y) * z & = x * (y * z) \\
x * 1 & = x \\
1 * x & = x \\
x * (y + z) & = (x * y) + (x * z) \\
(y + z) * x & = (y * x) + (z * x)
\end{align*}
Добавим к этой теории следующую аксиому:
\[ x * x = x \]
Докажите, что в этой расширенной теории выводимы следующие формулы:
\begin{enumerate}
\item $x * y = y * x$
\item $x + x = 0$
\end{enumerate}
\end{task}
\begin{solution}

\begin{enumerate}
	\item Будем считать, что уже доказали следующий пункт, т.е. $x + x = 0$, т.е. $x = -x$. Тогда:
	\begin{align*}
		(x + y) * (x + y) &=^{\text{дистрибутивность}} x*x + x*y + y*x + y*y = \\
 				          &= x + x*y + y*x + y
	\end{align*}
	С другой стороны $(x+y)*(x+y) = x+y$, т.е: 
	\begin{align*}
		x + y = x + y + x*y + y*x &\Rightarrow x + -x + y + -y = x + -x + y + -y +  x*y + y*x \\
							      &\Rightarrow 0 = x*y + y*x \Rightarrow x*y = -y*x
	\end{align*}
	И в силу того, что $x = -x$: $x*y = y*x$.
	\item 
	Покажем, что $0*x = 0$. $x + 0 = x \Rightarrow x*x + 0*x = x*x \Rightarrow x + 0*x = x$, прибавляем слева и справа $-x$ и полчаем, что $0*x = 0$.
	Теперь покажем то, что нужно:
	\begin{align*}
		x + -x = 0 &\Rightarrow x*x + -x*x = 0*x \Rightarrow x + -x*x = 0 \Rightarrow -x*x + -x*(-x*x) = 0 \Rightarrow \\
				   &\Rightarrow -x*(x + x) = 0 \Rightarrow x + x = 0
	\end{align*}
\end{enumerate}

\end{solution}


\end{document}
