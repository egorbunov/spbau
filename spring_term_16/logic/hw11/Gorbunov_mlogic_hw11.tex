%!TEX program = xelatex

\documentclass[12pt, a4paper]{article}
\usepackage[utf8]{inputenc}
\usepackage[russian]{babel}
\usepackage{pscyr}

\usepackage{xifthen}
\usepackage{parskip}
\usepackage{hyperref}
\usepackage[top=0.7in, bottom=1in, left=0.6in, right=0.6in]{geometry}
\usepackage{setspace}

\usepackage{amsmath}
\usepackage{MnSymbol}
\usepackage{amsthm}
\usepackage{mathtools}

\usepackage{algorithm}
\usepackage[noend]{algpseudocode}



\linespread{1.2}
\setlength{\parskip}{0pt}

\renewcommand\familydefault{\sfdefault}


% Stuff related to homework specific documents
\newcounter{MyTaskCounter}
\newcounter{MyTaskSectionCounter}
\newcommand{\tasksection}[1]{
	\stepcounter{MyTaskSectionCounter}
	\setcounter{MyTaskCounter}{0}
	\ifthenelse{\equal{#1}{}}{}{
	{\hfill\\[0.2in] \Large \textbf{\theMyTaskSectionCounter \enspace #1} \hfill\\[0.1in]}}
}

\newcommand{\task}[1]{
	\stepcounter{MyTaskCounter}
	\hfill\\[0.1in]
	\ifthenelse{\equal{\theMyTaskSectionCounter}{0}}{
	   \textbf{\large Задача №\theMyTaskCounter}
	}{
	   \textbf{\large Задача №\theMyTaskSectionCounter.\theMyTaskCounter}
	}
	\ifthenelse{\equal{#1}{}}{}{{\normalsize (#1)}}
	\hfill\\[0.05in]
}

% Math and algorithms

\makeatletter
\renewcommand{\ALG@name}{Алгоритм}
\renewcommand{\listalgorithmname}{Список алгроитмов}

\newenvironment{procedure}[1]
  {\renewcommand*{\ALG@name}{Процедура}
  \algorithm\renewcommand{\thealgorithm}{\thechapter.\arabic{algorithm} #1}}
  {\endalgorithm}

\makeatother

\algrenewcommand\algorithmicrequire{\textbf{Вход:}}
\algrenewcommand\algorithmicensure{\textbf{Выход:}}
\algnewcommand\True{\textbf{true}\space}
\algnewcommand\False{\textbf{false}\space}
\algnewcommand\And{\textbf{and}\space}

\newcommand{\xfor}[3]{#1 \textbf{from} #2 \textbf{to} #3}
\newcommand{\xassign}[2]{\State #1 $\leftarrow$ #2}
\newcommand{\xstate}[1]{\State #1}
\newcommand{\xreturn}[1]{\xstate{\textbf{return} #1}}

\DeclarePairedDelimiter\ceil{\lceil}{\rceil}
\DeclarePairedDelimiter\floor{\lfloor}{\rfloor}

\newcommand{\bigO}[1]{\mathcal{O}\left(#1\right)}


\title{Математическая логика. Домашнее задание №11}
\author{Горбунов Егор Алексеевич}

\begin{document}
\maketitle

\begin{task}[1]
Докажите, что отношение $\subseteq$ является частичным порядком. Какие
аксиомы при этом необходимо использовать?
\end{task}
\begin{solution}
\begin{itemize}
	\item Рефлексивность. Нам нужно показать, что $\forall x\ (x \subseteq x)$, т.е. $\forall z\ (z \in x \to z \in x)$. Это доказывается вообще без использования аксиом специфичных для $IZF$: достаточно один раз применить правило $\to I$, а после $(var)$.
	\item Антисимментричность сразу же следует из аксиомы экстенсиальности.
	\item Транзитивность. Нужно показать, что $\forall a \forall b \forall c\ (a \subseteq b \land b \subseteq c \to a \subseteq c)$.
	\[
		(\forall z\ (z \in a \to z \in b) \land \forall z\ (z \in b \to z \in c)) \to \forall z\ (z \in a \to z \in c)
	\] 
	Мы доказывали транзитивность импликации в интуиционистской логике (легко предоставить терм), так что никакие специфичные для $IZF$ аксиомы использовать не нужно.
\end{itemize}
\end{solution}

\begin{task}[2]
Докажите при помощи $\in$-индукции:
\begin{enumerate}
	\item Отношение $\in$ иррефлексивно.
	\item Не существует $x$ и $y$, таких что $x in y$ и $y in x$.
\end{enumerate}
\end{task}
\begin{solution}
\begin{enumerate}
	\item Для пустого множества по его определению иррефлексивность верна. Пускай для множества $x$ верно $ \forall y \in x\ y \notin y $. Пускай у нас $x \in x$, но тогда $\exists y \in x\ (y \in y)$, что не может быть верно по предположению индукции, а поэтому верно, что $\neg (x \in x)$.
	\item Индукция по $x$.
	\begin{itemize}
		\item $x = \emptyset$
			\begin{itemize}
				\item $y = \emptyset$. Понятно, что $x \notin y$ и $y \notin x$ по определению пустого множества.
				\item $y \neq \emptyset$. Тут верно, что $y \notin x$, т.к. $x$ пусто.
			\end{itemize}
		\item $x$ --- какое-то множество, не пустое.
			\begin{itemize}
				\item $y = \emptyset$. Верно, что $x \notin y$, а значит не верно, что $x \in y \land y \in x$.
				\item $y$ --- какое-то множество. По предположению индукции тут верно, что
				\[
					\forall a \forall b\ (a \in x \land b \in y \to \neg (a \in b \land b \in a))
				\]
				Если $x \in y \land y \in x$, то предположение индукции сразу же будет нарушено, т.к. найдутся такие $a = x$ и $b = y$, что $a \in b \land b \in a$.
			\end{itemize}
	\end{itemize}
\end{enumerate}
Я тут видимо напутал с тем, как использовать аксиому индукции, но сути это не меняет, т.к. видимо, достаточно просто последнего пункта док-ва.
\xqed
\end{solution}

\begin{task}[3]
Докажите следующие свойства натуральных чисел:
\begin{enumerate}
	\item $\forall x \in \mathbb{\mathbb{N}} (x = 0 \lor \exists y \in \mathbb{\mathbb{N}} (x = S(y)))$
	\item Используя предыдущий пункт и следующий принцип индукции
	\[
		(\forall x \in \mathbb{\mathbb{N}} (\forall y < x\ \varphi(y)) \to \varphi(x)) \to \forall x \in \mathbb{\mathbb{N}}\ \varphi(x)
	\]
	докажите обычный принцип индукции для натуральных чисел:
	\[
		\varphi(0) \land (\forall y \in \mathbb{\mathbb{N}} \varphi(y) \to \varphi(S(y))) \to \forall x \in \mathbb{\mathbb{N}}\ \varphi(x)
	\]
\end{enumerate}
\end{task}
\begin{solution}
Пусть $S'(n)$ --- это предикат в рамках теории множеств, а $S(n)$ --- succ.
\begin{enumerate}
	\item По аксиоме, характеризующей натуральные числа, если $n \in \mathbb{\mathbb{N}}$, то либо $n = \emptyset$, либо верен предикат $S'(n)$, а предикат $S'(n)$ в свою очередь утверждает, что $\exists k (n = \xbrace{k \cup \xbrace{k}})$ (неформально), т.е. $n = S(k) = \xbrace{k \cup \xbrace{k}}$. \xqed

	\item Покажем, что:
	\[
		x \vdash (\varphi(0) \land (\forall y \in \mathbb{\mathbb{N}} \varphi(y) \to \varphi(S(y)))) \to 
		((\forall y < x\ \varphi(y)) \to \varphi(x))
	\]
	По пункту а) $x$ может быть либо $0$ либо $S(z)$ для какого-то $y$. Если $x = 0$, то приведённая выше импликация примет вид:
	\[
		x \vdash \varphi(0) \to \top
	\]
	Т.к. не существует $y < 0$. Таким образом при $x = 0$ доказали.

	Если же $x = S(z)$. Тогда по левой части импликации верно, что $\varphi(z) \to \varphi(x)$. При этом по индукции верно, т.к. $z < x$, что $(\forall y < z\ \varphi(y)) \to \varphi(z)$, а тогда по транзитивности импликации получаем: $(x = S(z) \land (\forall y < z\ \varphi(y)) \to \varphi(z) \land (\varphi(z) \to \varphi(x))) \to \varphi(x)$, что эквивалентно тому, что хотим доказать.
	\xqed 
\end{enumerate}
\end{solution}

\begin{task}[4]
Докажите следующие свойства натуральных чисел (hint: каждый сле-
дующий пункт следует из предыдущего):
\begin{enumerate}
\item Если $S(x) < S(y)$, то $x < y$.
\item $x < y$ тогда и только тогда, когда $S(x) ≤ y$.
\item Отношение $\in$ транзитивно на элементах $\mathbb{N}$. Можно показать более
общее утверждение: если $x \in y \in z \in \mathbb{N}$, то $x \in z$.
\item Если $x \in y$ и $y \in \mathbb{N}$, то $x \in \mathbb{N}$.
\end{enumerate}
\end{task}

\begin{solution}
\begin{enumerate}
	\item $S(x) = \xbrace{x, \xbrace{x}}$, $S(y) = \xbrace{y, \xbrace{y}}$. Если $S(x) < S(y)$, то:
	\begin{align*}
		\xbrace{x, \xbrace{x}} \in y & \Rightarrow x \in \xbrace{x, \xbrace{x}} \in y \Rightarrow x < y \\
		\xbrace{x, \xbrace{x}} = y & \Rightarrow x \in \xbrace{x, \xbrace{x}} \subseteq y \Rightarrow x < y
	\end{align*}
\end{enumerate}
\end{solution}

\end{document}
