\documentclass[12pt, a4paper]{article}
\usepackage[utf8]{inputenc}
\usepackage[russian]{babel}
\usepackage{pscyr}

\usepackage{xifthen}
\usepackage{parskip}
\usepackage{hyperref}
\usepackage[top=0.7in, bottom=1in, left=0.6in, right=0.6in]{geometry}
\usepackage{setspace}

\usepackage{amsmath}
\usepackage{MnSymbol}
\usepackage{amsthm}
\usepackage{mathtools}

\usepackage{algorithm}
\usepackage[noend]{algpseudocode}



\linespread{1.2}
\setlength{\parskip}{0pt}

\renewcommand\familydefault{\sfdefault}


% Stuff related to homework specific documents
\newcounter{MyTaskCounter}
\newcounter{MyTaskSectionCounter}
\newcommand{\tasksection}[1]{
	\stepcounter{MyTaskSectionCounter}
	\setcounter{MyTaskCounter}{0}
	\ifthenelse{\equal{#1}{}}{}{
	{\hfill\\[0.2in] \Large \textbf{\theMyTaskSectionCounter \enspace #1} \hfill\\[0.1in]}}
}

\newcommand{\task}[1]{
	\stepcounter{MyTaskCounter}
	\hfill\\[0.1in]
	\ifthenelse{\equal{\theMyTaskSectionCounter}{0}}{
	   \textbf{\large Задача №\theMyTaskCounter}
	}{
	   \textbf{\large Задача №\theMyTaskSectionCounter.\theMyTaskCounter}
	}
	\ifthenelse{\equal{#1}{}}{}{{\normalsize (#1)}}
	\hfill\\[0.05in]
}

% Math and algorithms

\makeatletter
\renewcommand{\ALG@name}{Алгоритм}
\renewcommand{\listalgorithmname}{Список алгроитмов}

\newenvironment{procedure}[1]
  {\renewcommand*{\ALG@name}{Процедура}
  \algorithm\renewcommand{\thealgorithm}{\thechapter.\arabic{algorithm} #1}}
  {\endalgorithm}

\makeatother

\algrenewcommand\algorithmicrequire{\textbf{Вход:}}
\algrenewcommand\algorithmicensure{\textbf{Выход:}}
\algnewcommand\True{\textbf{true}\space}
\algnewcommand\False{\textbf{false}\space}
\algnewcommand\And{\textbf{and}\space}

\newcommand{\xfor}[3]{#1 \textbf{from} #2 \textbf{to} #3}
\newcommand{\xassign}[2]{\State #1 $\leftarrow$ #2}
\newcommand{\xstate}[1]{\State #1}
\newcommand{\xreturn}[1]{\xstate{\textbf{return} #1}}

\DeclarePairedDelimiter\ceil{\lceil}{\rceil}
\DeclarePairedDelimiter\floor{\lfloor}{\rfloor}

\newcommand{\bigO}[1]{\mathcal{O}\left(#1\right)}

\title{Домашнее задание №7 \\ Алгоритмы. 5 курс. Весенний семестр.}
\author{Горбунов Егор Алексеевич}

\begin{document}
\maketitle

\begin{task}[1]
Алёна отправила сообщение $m$, зашифрованное через $RSA$, двум людям. Для каждого
человека определено своё $e_i$ , но везде одинаковое $n = pq$, Оказалось, что $e_i$ взаимно
простые. Найдите сообщение Алёны за $\bigO{poly(\log n)}$.
\end{task}

\begin{solution}
Раз $e_1$ и $e_2$ взаимнопросты, то расширенный алгоритм Евклида даст нам такие $a$ и $b$, что:
\[
	ae_1 + be_2 = 1
\]
Окей, тогда можем записать:
\[
	m = m^1 = m^{ae_1+be_2} = (m^{e_1})^a (m^{e_2})^b
\]
Нам известны $m^{e_1}$ и $m^{e_2}$, а значит посчитать $(m^{e_1})^a (m^{e_2})^b$ мы сумеем используя арифметические операции по модулю $n$. Если $a$, например, оказалось отрицательным, то тоже не проблема,
т.к. $x^{-|a|} = (x^{-1})^{|a|}$, далее находим обратный по модулю и вперёд...

Ясно, что эти операции укладываются в $\bigO{poly(\log n)}$. \xqed
\end{solution}

\begin{task}[2]
\begin{enumerate}[label=(\alph*)]
	\item Пусть $n = pq$ и известно $\varphi{(n)}$, разложите n на множители. $\bigO{poly(\log n)}$.
	\item Пусть вам дано $RSA(n = pq, e, d)$. Пусть $e = 3$. Разложите $n$ на множители.
\end{enumerate}
\end{task}
\begin{solution}

\begin{enumerate}[label=(\alph*)]
	\item Имеем: $\varphi(n) = (p-1)(q-1) = pq - p - q - 1 = n - p - q - 1$, откуда:
	\[
		\begin{cases}
			p = n - q - \varphi(n) - 1 \\
			n = pq
		\end{cases}
		\Rightarrow
		\begin{cases}
			p = n - q - \varphi(n) - 1 \\
			q^2 - q(n - \varphi(n) - 1) + n = 0
		\end{cases}
	\]
	Решаем квадратное уравнение, разумеется, за $\bigO{poly(\log n)}$ и подставляем полученное $q$ в первое уравнение системы, чтобы получить $p$. \xqed

	\item По построению $RSA$ у нас $3d - 1 = 0 \mod{\varphi{n}}$, т.е. это значит, что $3d-1 = k\varphi{(n)}$, $\varphi{(n)} = \frac{3d-1}{k}$. Переберём $k$ и будем искать решение квадратного уравнения из предыдущего пункта, пока не найдём целочисленного решения. \xqed
\end{enumerate}
\end{solution}

\begin{task}[4]
Придумайте, как свести вычисление $FFT$ последовательности размера $pn$ к $p$ вычислениям $FFT$ от
последовательностей размера $n$ и $\bigO{p^2*n}$ дополнительных арифметических операций. Напишите
псевдокод.
\end{task}
\begin{solution}
Можно рассмотреть $p$ полиномов вида: $f_i(x) = \sum_{k=0}^{n-1}{a_{pk + i}x^k}$. Тогда полином, соответствующий исходной последовательности будет такой:
\[
	f(x) = \sum_{k = 0}^{p-1}{f_k(x^p)x^k}
\]
\end{solution}

\begin{lstlisting}
def ALG(a[n*p]):
	for i in [0, 1, 2, 3, ..., p-1]:
		fft[i] = FFT(a[i], a[i+p], a[i+2*p], ..., a[i+(n-1)*p])
	for i in [0, 1, 2, 3, ..., np-1]:
		for k in [0, 1, ..., p-1]:
			j = (-i * k) mod n*p
			answer[i] = answer[i] + (j-th power of n-th root of 1) * fft[k][i % n]
\end{lstlisting}
Как-то так...\xqed

\begin{task}[6]
 Заданы картинка $a$ и образец $p$ в виде матриц вещественных чисел из $[0, 1]$ размерами n × n и
 $k \times k$ соответственно $(n \geq k)$. Требуется найти позицию $(x, y), 0 \leq x \leq n - k, 0 \leq y \leq n - k$, для
 которой:
 \[
 	\sum_{i=0}^{k-1} { \sum_{j = 0}^{k - 1} {\xparen{p_{i,j} - a_{(y+i), (x+j))}}^2} }
 	\longrightarrow \min
 \]
 за время $\bigO{n^2 \log n}$
\end{task}
\begin{solution}
Раскрывая квадрат сумма разбивается на 3 слагаемых: сумма квадратов пикселей образца, сумма квадратов подматрицы картинки и скалярное произведение двух веторов длины $k^2$. Первое считаем в лоб, второе с использованием предподсчёта на <<префиксах>> матрицы, а третье сводим к вычислению всевозможных скалярных произведений на циклические сдвиги (emaxx говорит, что это делается за $\bigO{n^2 \log n}$)
\xqed
\end{solution}

\end{document}
