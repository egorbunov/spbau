\documentclass[12pt, a4paper]{article}
\usepackage[utf8]{inputenc}
\usepackage[russian]{babel}
\usepackage{pscyr}
\usepackage{amssymb}
\usepackage{xifthen}
\usepackage{parskip}
\usepackage{hyperref}
\usepackage{setspace}

\usepackage{graphicx}
\usepackage{xcolor}
\usepackage{amsmath}
\usepackage{MnSymbol}
\usepackage{amsthm}
\usepackage{mathtools}
\usepackage{algorithm}
\usepackage[noend]{algpseudocode}
\usepackage[shortlabels]{enumitem}
                    \setlist[enumerate, 1]{1\textsuperscript{o}}
\usepackage{subfig}
\usepackage{tikz}
\usepackage{tikz,fullpage}
\usetikzlibrary{shapes,snakes}
\usetikzlibrary{arrows,%
                petri,%
                topaths}%
\usepackage{tkz-berge}
\usepackage[top=0.5in, bottom=0.7in, left=0.6in, right=0.6in]{geometry}
\linespread{1.3}

% \renewcommand\familydefault{\sfdefault}


% Stuff related to homework specific documents
\newcounter{MyTaskCounter}
\newcounter{MyTaskSectionCounter}
\newcommand{\tasksection}[1]{
	\stepcounter{MyTaskSectionCounter}
	\setcounter{MyTaskCounter}{0}
	\ifthenelse{\equal{#1}{}}{}{
	{\hfill\\[0.2in] \Large \textbf{\theMyTaskSectionCounter \enspace #1} \hfill\\[0.1in]}}
}

\newcommand{\task}[1]{
	\stepcounter{MyTaskCounter}
	\hfill\\[0.1in]
	\ifthenelse{\equal{\theMyTaskSectionCounter}{0}}{
	   \textbf{\large Задача №\theMyTaskCounter}
	}{
	   \textbf{\large Задача №\theMyTaskSectionCounter.\theMyTaskCounter}
	}
	\ifthenelse{\equal{#1}{}}{}{{\normalsize (#1)}}
	\hfill\\[0.05in]
}

% Math and algorithms

\makeatletter
\renewcommand{\ALG@name}{Алгоритм}
\renewcommand{\listalgorithmname}{Список алгроитмов}

\newenvironment{procedure}[1]
  {\renewcommand*{\ALG@name}{Процедура}
  \algorithm\renewcommand{\thealgorithm}{\thechapter.\arabic{algorithm} #1}}
  {\endalgorithm}

\makeatother

\algrenewcommand\algorithmicrequire{\textbf{Вход:}}
\algrenewcommand\algorithmicensure{\textbf{Выход:}}
\algnewcommand\True{\textbf{true}\space}
\algnewcommand\False{\textbf{false}\space}
\algnewcommand\And{\textbf{and}\space}

\newcommand{\xfor}[3]{#1 \textbf{from} #2 \textbf{to} #3}
\newcommand{\xassign}[2]{\State #1 $\leftarrow$ #2}
\newcommand{\xstate}[1]{\State #1}
\newcommand{\xreturn}[1]{\xstate{\textbf{return} #1}}

\DeclarePairedDelimiter\ceil{\lceil}{\rceil}
\DeclarePairedDelimiter\floor{\lfloor}{\rfloor}

\newcommand{\bigO}[1]{\mathcal{O}\left(#1\right)}

\newcommand{\xqed}{\hfill $\blacksquare$}

\newcommand{\code}[1]{\colorbox{gray!15}{\footnotesize\texttt{#1}}}
\title{Домашнее задание №2 \\ Алгоритмы. 5 курс. Весенний семестр.}
\author{Горбунов Егор Алексеевич}

\begin{document}
\maketitle

\section{Мои решения}

\begin{task}[1]
	Дано дерево из одной вершины. Требуется уметь отвечать online за $\bigO{\log{n}}$ на запрос: подвесить вершину $u$ к вершине дерева $v$ и вернуть диаметр дерева. Диаметр дерева --- длина самого длинного простого пути в дереве.
\end{task}
\begin{solution}
Будем для дерева $T$ хранить концы максимального простого пути: $x$ и $y$. Ясно, что такой путь не один, мы будем хранить для какого-то одного. Так же для вершины будем хранить её глубину в дереве, для того, чтобы уметь находить расстояние между $2$-мя вершинами $a$ и $b$ через глубину $a$, $b$ и $lca(a, b)$. Следующая процедура решает исходную задачу:
\begin{lstlisting}[]
def hang(u, v, T):
	u.parent = v
	u.depth = v.depth + 1
	dist_x_u = T.x.depth + u.depth - 2 * lca(T.x, u).depth
	dist_y_u = T.y.depth + u.depth - 2 * lca(T.y, u).depth
	if dist_x_u > T.diametr:
		T.y = u
	elif dist_y_u > T.diametr:
		T.x = u
	T.diametr = max(dist_x_u, dist_y_u, T.diametr)
	return T.diametr
\end{lstlisting}
Мы умеем искать $lca(a,b)$ за $\bigO{\log{n}}$, а значит и сложность данной процедуры $\bigO{\log{n}}$. Осталось показать её корректность. Для этого покажем, что если у нас есть некоторый максимальных путь $x...y$ в дереве $T$, то,в дереве $T+v$, где $v$--- добавленный лист, максимальный путь будет $x...y$, $x...v$ или $y...v$. Действительно, если в $T+v$ диаметр не изменился в сравнении с $T$, то один из максимальных путей в $T+v$ --- это $x...y$. Пускай теперь диаметр $T+v$ на $1$ больше, чем в $T$ (больше он измениться не мог, т.к. добавили лишь $1$ вершину). Так же ясно, что диаметр мог увеличиться только засчёт того, что был продлён какой-то максимальный путь $a...b$. Любые $2$ максимальных пути в дереве пересекаются, т.к. иначе можно было бы построить путь длиннее в силу связности дерева. Пересекаться $2$ пути в дереве могут только по какому-то отрезку $c..d$, пусть он разбивает пути так (НУО): $x..c..d..y$, $va..c..d..b$.
Ясно, что длина $x..c$ равна $a..c$ и длина $d..y$ равна длине $d..b$, иначе, выбирая максимумы из каждой пары можно было бы построить пути длиннее $x..y$ в $T$, а он уже максимальный. Таким образом путь $va..c..d..y$ --- максимальный в $T+v$. Т.е. мы показали, что новый максимальный путь в $T+v$ стоит выбирать из: $x...y$, $x...v$ или $y...v$. А значит корректность показана. \xqed
\end{solution}

\begin{task}[2]
Дан ориентированный граф, в котором исходящая степень каждой вершины равна единице. Запросы online: из вершины $v$ сделать $k$ шагов вперёд.
\begin{enumerate}[label=(\alph*)]
	\item Предподсчёт: $\bigO{n\log{k_{max}}}$, время на запрос: $\bigO{\log{k}}$
	\item Предподсчёт: $\bigO{n\log{n}}$, время на запрос: $\bigO{\log{\min{(k,n)}}}$
\end{enumerate}
\end{task}
\begin{solution}
\begin{enumerate}[label=(\alph*)]
	\item Нам дана длина максимального прыжка $k_{max}$, предподсчитаем двоичные прыжки для каждой вершины (т.к. из каждой вершины можно пойти в одну единственную) длин $0, 2^1, 2^2, \ldots, 2^{\log{k_{max}}}$. Вершин всего $n$: $v_1, \ldots, v_n$
	\begin{algorithmic}
	\For{$v \in \lbrace v_1,\ldots, v_n \rbrace$}
		\xassign{$jump[v, 0]$}{$v$}
		\xassign{$jump[v, 1]$}{$v.next$}
	\EndFor
	\For{$k \in 1\ldots\log{k_{max}}$}
		\For{$v \in \lbrace v_1,\ldots, v_n \rbrace$}
			\xassign{$jump[v, 2^{k}]$}{$jump[jump[v, 2^{k-1}], 2^{k-1}]$}
		\EndFor
	\EndFor
	\end{algorithmic}
	Теперь будем отвечать на запрос сделать $k$ шагов из вершины $v$ так: прыгнем сначала в вершину $u = jump[v, 2^{\floor{\log{k}}}]$, а потом пойдём вперёд от полученной вершины $u$. Ясно, что нам осталось сделать не более $\log{k}$ переходов. Таким образом отвечаем на запрос за $\bigO{\log{k}}$ с предподсчётом $\bigO{n\log{k_{max}}}$.

	\item Не умаляя общности рассмотрим орграф, в котором лишь одна компонента слабой связности, т.е. слабо-связный орграф $G$ в котором $outdeg(v) = 1$. Такой граф может содержать лишь один цикл, иначе была бы вершины с $outdeg(v) > 1$. Мы легко за $\bigO{n}$ сможем найти этот цикл. Пометим все вершины, которые принадлежат циклу, это тоже делается за $\bigO{n}$ и запомним длину этого цикла --- $len$. Теперь, аналогично предыдущему пункту, предподсчитаем прыжки, но только для длин: $0, 2^1, 2^2, \ldots, 2^{\log{n}}$. Теперь, если длина прыжка в запросе $\leqslant n$, то мы аналогично прыжками сможем получить ответ, используя предподсчитанные прыжки. Но если $k > n$, то заметим, что прыгнув на $n = 2^{\log{n}}$ мы точно окажемся на цикле (если он есть), т.к. всего вершин $n$. Таким образом, мы находимся на цикле длины $len$ и нам осталось сделать $k - 2^{\log{n}}$ шагов. Но т.к. теперь мы будем шагать только по циклу, то имеет смысл прыгать на $(k - 2^{\log{n}}) \mod\ len$, а это число всяко $< n$. Таким образом нам хватило предподсчитанных прыжков $0, 2^1, 2^2, \ldots, 2^{\log{n}}$, чтобы покрыть запросы для любых $k$. Предподсчёт $\bigO{n\log{n}}$, запрос $\bigO{\log{\min(k,n)}}$.
	\end{enumerate}
\end{solution}

\begin{task}[3]
Дан массив чисел длины $n$. За $\bigO{\log{n}}$ в online обрабатывать запросы:
\begin{itemize}
	\item посчитать сумму кубов чисел на отрезке $[L,R]$
	\item прибавить $x$ ко всем числам на отрезке $[L,R]$
	\item получить значение $i$-го числа
\end{itemize}
\end{task}
\begin{solution}
Пускай $s_3 = x_1^3 + x_2^3 + \ldots + x_k^3$, $s_2 = x_1^2 + x_2^2 + \ldots + x_k^2$ и $s = x_1 + x_2 + \ldots + x_k$,
\begin{equation*}
\begin{split}
(x_1+a)^3 + (x_2+a)^3 + \ldots + (x_k+a)^3 &= x_1^3 + 3x_1^2a + 3x_1a^2 + a^3 + \ldots + x_k^3 + 3x_k^2a + 3x_ka^2 =\\ 
&= s_3 + 3a \cdot s_2 + 3a^2 \cdot s_1 + k \cdot a^3 
\end{split}
\end{equation*}
Т.о. построим на данном массиве дерево отрезков, но теперь будем поддерживать одновременно сумму, сумму квадратов и сумму кубов на отрезках. Пересчёт суммы кубов на отрезке теперь просто пересчитывается: к каждому за $\bigO{\log{n}}$ отрезков, разбивающих $[L,R]$, применяем полученную выше формулу. Получение значение $i$-го элемента остаётся таким же: спускаясь по дереву добавляем к ответу значения, которые когда-либо прибавлялись к отрезку, содержащему это число. \xqed
\end{solution}

\begin{task}[4]
Дана скобочная последовательность из круглых скобок длины $n$. Запросы: является ли отрезков $[L, R]$ правильной скобочной последовательностью; изменить $i$-ую скобку. $\bigO{\log{n}}$, online.
\end{task}
\begin{solution}
Рассмотрим следующий массив $P[0..n]$:
\begin{align*}
	P[0] &= 0 \\
	P[i] &= \text{\big[число <<(>> скобок} - \text{число <<)>> скобок\big] на отрезке } [1,i]
\end{align*}
Знаем, что вся скобочная последовательность длины $n$ является правильной, если $P[n] = 0$ и $\forall i \in [1..n]\ (P[i] \geqslant 0)$, а иначе: $\min_{i \in [1..n]}{P[i]} \geq 0$. Аналогично можно обобщить: подпоследовательность $[L,R]$ исходной скобочной последовательности является правильной, если:
\[
	P[R] = P[L - 1] \text{ и } \min_{i \in {L..R}}{P[i]} \geq P[L - 1] 
\]
Вышенаписанное правило верно в силу того, что число откр. скобок минус число закр. скобок на любом префиксе $[L..i]$ подстроки $[L..R]$ равно $P[i] - P[L - 1]$.

Построим над массивом $P[0..n]$ дерево отрезков. Мы умеем уже поддерживать операцию $\min$ на подотрезках, а значит умеем и отвечать на запрос о правильности скобочной подпоследовательности $[L..R]$ в силу описанного выше правила.

Заметим теперь, что изменение скобки на $i$-ой позиции исходной строки приводит к следующему:
\begin{align*}
\forall j \in [j..n]:\ P[j] \rightarrow P[j] - 2, \text{ если на } i \text{-ой позиции стояла <<(>>}\\
\forall j \in [j..n]:\ P[j] \rightarrow P[j] + 2, \text{ если на } i \text{-ой позиции стояла <<)>>}
\end{align*}
Таким образом нам просто нужно уметь за $\bigO{\log{n}}$ выполнять прибавление числа на отрезке, но это мы уже умеем делать используя дерево отрезков. 

Таким образом мы свели ответ на запрос о правильности скобочной подпоследовательности к поиску минимума на отрезке в массиве $P$, а запрос изменения скобки свели к прибавлению числа на отрезке в массиве $P$. Всё делаем за $\bigO{\log{n}}$. \xqed
\end{solution}

\section{Дорешивание}

\begin{task}[5]
Попробуем модифицировать идею \texttt{SparseTable} так, чтобы она работала для произвольных ассоциативных функций: предложите способ выделить $\bigO{n\log{n}}$ отрезков в массиве размера $n$ так, что любой отрезок $[L,R]$ можно было представить в виде объединения $\bigO{1}$ непересекающихся выделенных отрезков. Заметим, что дерево отрезков выделяет $\bigO{n}$ орезков и любой отрезок представляется как объединение $\bigO{\log{n}}$ из них.
\end{task}
\begin{solution}
Рассмотрим дерево отрезков на массиве. И на любых двух сосендних отрезках на одном уровне посчитаем суффиксные суммы на левом отрезке и префиксные суммы на правом отрезке.
\end{solution}

\begin{task}[6]
Дан массив из $n$ элементов. Запросы: $k-e$ по порядку среди различных чисел на отрезке $[L,R]$.
\begin{enumerate}[label=(\alph*)]
	\item offline за $\bigO{\log^3{n}}$
	\item online за $\bigO{\log^3{n}}$
\end{enumerate}
\end{task}
\begin{solution}
\begin{enumerate}[label=(\alph*)]
	\item Задача про порядковую статистику и бинпоиск (?)
	\item (?)
\end{enumerate}
\end{solution}

\section{Практика}
\begin{task}[1]
Поставить на плоскость точку и за $\bigO{\log^2{n}}$ отвечать на запрос, сколько точек в квадрате $[L..R x B..T]$.
\end{task}
\begin{solution}
hint: дерево отрезков, в вершине которого дерево отрезков.\\
Дерево поиска внутри дерева отрезков. Добавление --- вставка с $y$-коордиантой в качестве ключа в соответствующие конкретному $x$ деревья поиска.
Таким образом у нас изначально пустая сетка $F[1..n][1..m]$. Строим дерево отрезков размера $n$ (над массивом $F[1..n]$), а в каждом узле дерева храним пустое дерево поиска. Если нужно добавить точку с координатами $(i, j)$, то мы спусаемся по дереву отрезков к элементу $i$ и в каждую посещённую вершину дерева отрезков, как в дерево поиска, добавляем $j$.\\
Запрос о числе точек в $[L..R x B..T]$ соответственно будет таким: дерево отрезков даёт на $\bigO{\log{n}}$отрезков-деревьев, на которые разбит отрезок $[L..R]$. В каждом таком дереве быстро ищем число вершин с ключами в заданном отрезке $[B..T]$. \xqed
\end{solution}

\begin{task}[2]
Дано дерево с весами на рёбрах. Запрос online: минимум на пути из $a$ в $b$ за $\langle \bigO{n\log{n}}, \bigO{\log{n}} \rangle$, т.е. $\bigO{n\log{n}}$ --- предподсчёт и $\bigO{\log{n}}$ на запрос.
\end{task}
\begin{solution}
Предподсчитываем двоичные прыжки вместе с минимумом на прыжке, т.е.
$jump[v, 2^k]$ и $min[v, 2^k]$, где последнее --- это вес минимального ребра на пути вверх из $v$ в $jump[v, 2^k]$ \xqed
\end{solution}

\begin{task}[3]
Дерево с весами на рёбрах. Нужно отвечать online на запросы о длине простого пути между парой вершин. 
$\langle \bigO{n}, \bigO{1} \rangle$
\end{task}
\begin{solution}
Рассматриваем эйлеров обход дерева, но каждое ребро вниз входит с плюсом (его вес), а каждое ребро вверх идёт с минусом.
Далее на запрос о пути между $a$ и $b$ мы делаем запрос о сумме на отрезке между $a$ и $lca(a,b)$ (по модулю) и складываем с суммой на отрезке между $b$ и $lca(a,b)$.
\end{solution}

\begin{task}[4]
Дерево, на вершинах которого могут быть пометки. Запросы: пометить вершину, снять пометку с вершины, число помеченных вершин в поддереве. $\langle \bigO{n}, \bigO{\log{n}} \rangle$.
\end{task}
\begin{solution}
Эйлеров обход. Только нужно понимать, где границы поддерева, т.е. запоминать первое и последнее вхождения вершины.
\end{solution}


\begin{task}[8]
Дан лес подвешенных деревьев. Нужно отвечать на запросы:
\begin{itemize}
	\item Подвесить дерево с корнем $v$ к вершине $u$ другого дерева
	\item Отрезать поддерево с корнем в вершине $v$ от её дерева
	\item Проверить, в одном ли дереве леэат $u$ и $v$
\end{itemize}
Время: $\bigO{\log{n}}$.
\end{task}
\begin{solution}
Декартово дерево.
Красим вершины в эйлеровом обходе. Вставляем перед белым (?) вхождением.
Не ясно.
?????????????????????????????????
\end{solution}

\begin{task}[9]
Дан лес подвешенных деревьев. Нужно отвечать на запросы:
\begin{itemize}
	\item Подвесить дерево с корнем $v$ к вершине $u$ другого дерева
	\item \texttt{LCA} вершин $u$ и $v$
\end{itemize}
Время: $\bigO{\log{n}}$.
\end{task}
\begin{solution}
Та же структура, что и в $8$?
Белые вершины за +1, чёрные за -1 и минимум на префиксе.
\end{solution}

\end{document}
