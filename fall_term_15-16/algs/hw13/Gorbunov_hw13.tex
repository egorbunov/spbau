\documentclass[12pt, a4paper]{article}
\usepackage[utf8]{inputenc}
\usepackage[russian]{babel}
\usepackage{pscyr}

\usepackage{xifthen}
\usepackage{parskip}
\usepackage{hyperref}
\usepackage[top=0.7in, bottom=1in, left=0.6in, right=0.6in]{geometry}
\usepackage{setspace}

\usepackage{amsmath}
\usepackage{MnSymbol}
\usepackage{amsthm}
\usepackage{mathtools}

\usepackage{algorithm}
\usepackage[noend]{algpseudocode}



\linespread{1.2}
\setlength{\parskip}{0pt}

\renewcommand\familydefault{\sfdefault}


% Stuff related to homework specific documents
\newcounter{MyTaskCounter}
\newcounter{MyTaskSectionCounter}
\newcommand{\tasksection}[1]{
	\stepcounter{MyTaskSectionCounter}
	\setcounter{MyTaskCounter}{0}
	\ifthenelse{\equal{#1}{}}{}{
	{\hfill\\[0.2in] \Large \textbf{\theMyTaskSectionCounter \enspace #1} \hfill\\[0.1in]}}
}

\newcommand{\task}[1]{
	\stepcounter{MyTaskCounter}
	\hfill\\[0.1in]
	\ifthenelse{\equal{\theMyTaskSectionCounter}{0}}{
	   \textbf{\large Задача №\theMyTaskCounter}
	}{
	   \textbf{\large Задача №\theMyTaskSectionCounter.\theMyTaskCounter}
	}
	\ifthenelse{\equal{#1}{}}{}{{\normalsize (#1)}}
	\hfill\\[0.05in]
}

% Math and algorithms

\makeatletter
\renewcommand{\ALG@name}{Алгоритм}
\renewcommand{\listalgorithmname}{Список алгроитмов}

\newenvironment{procedure}[1]
  {\renewcommand*{\ALG@name}{Процедура}
  \algorithm\renewcommand{\thealgorithm}{\thechapter.\arabic{algorithm} #1}}
  {\endalgorithm}

\makeatother

\algrenewcommand\algorithmicrequire{\textbf{Вход:}}
\algrenewcommand\algorithmicensure{\textbf{Выход:}}
\algnewcommand\True{\textbf{true}\space}
\algnewcommand\False{\textbf{false}\space}
\algnewcommand\And{\textbf{and}\space}

\newcommand{\xfor}[3]{#1 \textbf{from} #2 \textbf{to} #3}
\newcommand{\xassign}[2]{\State #1 $\leftarrow$ #2}
\newcommand{\xstate}[1]{\State #1}
\newcommand{\xreturn}[1]{\xstate{\textbf{return} #1}}

\DeclarePairedDelimiter\ceil{\lceil}{\rceil}
\DeclarePairedDelimiter\floor{\lfloor}{\rfloor}

\newcommand{\bigO}[1]{\mathcal{O}\left(#1\right)}

\title{Алгоритмы. Домашнее задание №12}
\author{Горбунов Егор Алексеевич}

\begin{document}
\maketitle

\task{Вертикальные и горизонтальные прямые}
\textit{
	Условие: даны $n$ вертикальных и горизонтальных отрезков (всего $n$). Нужно посчитать число пересечений
	этих отрезков за $\bigO{n\log{n}}$.
}\\
Решение: пусть горизонтальные отрезки обозначены так: $(x_i, x'_i; y_i)$, где $x_i$ --- левая абсцисса горизонтального отрезка, $x'_i$ --- правая, а $y_i$ --- ордината, на которой отрезок находится. Вертикальные отрезки пусть заданы аналогично: $(y_j, y'_j; x_j)$. Разобъём каждый горизонтальный отрезок $(x_i, x'_i; y_i)$
на такие пары: $(x_i; y_i)$ и $(x'_i; y_i)$, где пара $(x_i; y_i)$ обозначает, что в точке $x_i$ по оси абсцисс
открывается отрезок на высоте $y_i$, а пара $(x'_i; y_i)$ обозначает, соответственно, что в точке $x'_i$ по оси абсцисс такой отрезок закрывается. Отсортируем все полученные пары и вертикальные отрезки по $x$-координате, т.е. $(y_j, y'_j; x_j) < (x_i; y_i) \iff x_j < x_i$. Теперь будем перебирать пары и вериткальные отрезки в отсортированном порядке, т.е. будем просто идти слева направо по оси абсцисс сканирующей прямой, перпендикулярной оси абсцисс. Обходя таким образом мы будем последовательно натыкаться либо на штуки вида $(x_i; y_i)$ либо $(x_i'; y_i)$ либо $(y_j, y'_j; x_j)$. Сейчас мы скажем, что будем делать в каждом случае и сразу увидим, что задача решена.
Заметим: 1) чтобы посчитать число всех пересечений нам нужно посчитать отдельно число пересечений каждого вертикального отрезка с горизонтальными 2) пускай идя сканирующей прямой мы наткнулись на вертикальный отрезок $(y_j, y'_j; x_j)$, тогда ясно, что число его пересечений с горизонтальными отрезками равно число открытых горизонтальных отрезков (тех отрезков, до концов которых сканирующая прямая ещё не дошла) на высоте от $y_j$ до $y'_j$. Научимся быстро отвечать на вопрос о таком количестве: будем поддерживать сбалансированное дерево 
$T$
поиска с операцией $numKeys(a, b)$ нахождения числа вершин (ключей) в диапазоне от $a$ до $b$, мы умеем это делать быстро, т.е. запрос о числе вершин в диапазоне занимает $\bigO{\log{n}}$.
Таким образом предлагается следующая тактика действий при обходе сканирующей прямой:
\begin{enumerate}
	\item Встретили $(x_i; y_i)$ (отрезок открывается), тогда выполняем $T.add(y_i)$
	\item Встретили $(y_j, y'_j; x_j)$, тогда выполняем $sum = sum + T.numKeys(y_j, y'_j)$
	\item Встретили $(x'_i; y_i)$ (отрезок закрывается), тогда выполняем $T.delete(y_i)$
\end{enumerate}
Видим, что число открытых в данный момент отрезков на высоте от $a$ до $b$ равно числу ключей в дереве $T$ в диапазоне от $a$ до $b$, т.к. ключи в дерево добавляются только, когда встречаем открытие отрезка, а удаляются сразу, как встречаем точку, закрывающую отрезок.
В конце концов (по окончании обхода сканирующей прямой) будем иметь ответ на вопрос задачи, записанный в переменной $sum$.\\
Сложность = время на сортировку + (число границ отрезков) * (время операции $T.add$ или $T.delete$ или $T.numKeys$) = $\bigO{n\log{n}}$ \xqed


\task{Число различных на отрезке}
Дан массив $A[1..n]$ нужно отвечать на запросы о количестве различных чисел на отрезке $[l..r]$
\begin{enumerate}[label=(\alph*)]
\item Отсортируем все запросы по правому концу $r$. Будем теперь двигаться по массиву слева направо параллельно
поддерживая следующий массив: $B[1..n]$: $B[i]~=~1$, если справа от $i$ нет элемента равного $A[i]$, иначе $B[i]~=~0$. Заметим, что при переходи от элемента $i$ к $i+1$ в массиве $B$ происходят следующие изменения: $B[i+1] = 1$ и, если $A[k], k < i + 1$ --- это последнее появление элемента равного $A[i+1]$ в $A[1..i]$, то $B[k]~=~0$. Таким образом, чтобы поддерживать массив $B$ нам нужно дополнительно хранить дерево $L$ (конечно, сбалансированное!), ключами в котором будут значения массива $A$, а значениями --- индекс последней встречи ключа к текущему моменту. Заметим следующее: пусть мы в данный момент прошли по $A$ префикс $A[1..i]$ и у нас
есть запрос на число различных элементов в подмассиве $A[l..i]$, тогда ответом на него будет \emph{число единиц в $B[l..i]$}. Это верно, т.к. чило различных элементов в $A[l..i]$ можно посчитать идя по $A[l..i]$ с конца и прибавлять $1$ к ответу каждый раз как встречаем новенькое число, легко видеть, что прибавлять единички мы будем именно в тех местах, где они стоят в $B[l..i]$.
Итого будем строить сбалансированное $T$ дерево по массиву $B$ последовательно, по мере обхода $A$. Ключами в этом дереве будет $i$, а значениями $B[i]$, соответственно $sum(B[l..i]) = T.sumPrefix(keyFrom = l)$. Последний запрос умеем выполнять за $\log{n}$.
Итого:
\begin{enumerate}
	\item Сортируем запросы по $r$
	\item $for\ i\ from\ 1\ to\ n$: --- $\bigO{n}$
	\item $B[i] = 1, T.add(key = i, value = 1)$ \\
	$k = L.getValByKey(A[i]),\ B[k] = 0,$\\
	$T.setValue(key=k, val=0), L.setValByKey(A[i], i)$\\
	--- $\bigO{\log{n}}$\\
	(последние 3 присваивания происходят только если такой $k$ нашёлся)
	\item запросы просматриваются вместе с проходом по $A$ (это не ломает линейной асимптотики обхода),
	если запрос $[l,r]$ таков, что $r == i$, то отвечаем на него: \\ 
	$T.sumPrefix(keyFrom = l)$ --- $\bigO{\log{n}}$.
	$sumValues$ в данном случае эквивалентно суммированию на подотрезке в массиве $B$ или подсчёт числа единиц :).
\end{enumerate}
Видно, что асимптотика на запрос получилась логарифмическое, т.е. $\bigO{\log{n}} = \bigO{\log^2{n}}$ \xqed
\item См. пункт $(a)$ \xqed
\item Для того, чтобы решить задачу $online$ прибегнем к персистентности. Построим за $\bigO{n\log{n}}$
$n$ сбалансированных деревьев $T_r$, что запросы вида $[l,r]$ будут обрабатываться так: $T_r.sumPrefix(l)$.
Т.е. мы, как и в пред. пунктах, пройдёмся по массиву на каждом шаге выделяя новое дерево. На предподсчёт потратится именно $\bigO{n\log{n}}$ в силу сбалансированности дерева. \xqed
\end{enumerate}


\task{Котики}
\textit{Дан набор из отрезков и котики...}
\begin{enumerate}[label=(\alph*)]
\item Будем хранить в сбалансированном дереве $T$ по ключу $i$ время, когда котик, сидящий на $i$-ой клетке ушёл.
У нас есть $n$ клеток и пусть вначале в этом дереве по всем ключам $1..n$ значения $-1$ --- кот не уходил.
\begin{enumerate}[1)]
	\item Пусть котики уходили с $1$ по $m$ моменты времени. 
	\\будем хранить $ANS[k]$ --- число отрезков, которые стали свободны после ухода котика в момент времени $k$
	\item Поддержим операцию $T.min(a, b)$ и $T.max(a,b)$, ведь мы умеем это делать. И заметим, что если в какой-то момент $T.min(a,b) = -1$, то на отрезке $[a..b]$ сидит котик. Если же это число не равно $-1$,
	то $T.max(a,b)=t'$ --- время, когда с отрезка ушёл последний котик
	\item Теперь для каждого отрезка $[l..r]$ найдём время $t'$ (при поищи $T$), в которое с отрезка ушёл последний котик и
	сделаем $ANS[t']++$ (если $t' \neq -1$)
\end{enumerate}
Таким образом видим, что в $ANS$ у нас записан ответ на вопрос задачи в каждый момент времени.
Первым этапом мы строим дерево $T$ по запросам: $\bigO{n\log{n}}$\\
Далее мы проходимся по отрезкам и выполняем с каждым действие за $\bigO{\log{n}}$

Итого время: $\bigO{n\log{n} + m\log{n}}$, где $m$ --- число отрезков.
Но время на запрос (обработку отрезка) равно $\bigO{\log{n}}$ \xqed
\end{enumerate}

\task{k-статистика (числа различны)}



\end{document}
