\documentclass[12pt, a4paper]{article}
\usepackage[utf8]{inputenc}
\usepackage[russian]{babel}
\usepackage{pscyr}

\usepackage{xifthen}
\usepackage{parskip}
\usepackage{hyperref}
\usepackage[top=0.7in, bottom=1in, left=0.6in, right=0.6in]{geometry}
\usepackage{setspace}

\usepackage{amsmath}
\usepackage{MnSymbol}
\usepackage{amsthm}
\usepackage{mathtools}

\usepackage{algorithm}
\usepackage[noend]{algpseudocode}



\linespread{1.2}
\setlength{\parskip}{0pt}

\renewcommand\familydefault{\sfdefault}


% Stuff related to homework specific documents
\newcounter{MyTaskCounter}
\newcounter{MyTaskSectionCounter}
\newcommand{\tasksection}[1]{
	\stepcounter{MyTaskSectionCounter}
	\setcounter{MyTaskCounter}{0}
	\ifthenelse{\equal{#1}{}}{}{
	{\hfill\\[0.2in] \Large \textbf{\theMyTaskSectionCounter \enspace #1} \hfill\\[0.1in]}}
}

\newcommand{\task}[1]{
	\stepcounter{MyTaskCounter}
	\hfill\\[0.1in]
	\ifthenelse{\equal{\theMyTaskSectionCounter}{0}}{
	   \textbf{\large Задача №\theMyTaskCounter}
	}{
	   \textbf{\large Задача №\theMyTaskSectionCounter.\theMyTaskCounter}
	}
	\ifthenelse{\equal{#1}{}}{}{{\normalsize (#1)}}
	\hfill\\[0.05in]
}

% Math and algorithms

\makeatletter
\renewcommand{\ALG@name}{Алгоритм}
\renewcommand{\listalgorithmname}{Список алгроитмов}

\newenvironment{procedure}[1]
  {\renewcommand*{\ALG@name}{Процедура}
  \algorithm\renewcommand{\thealgorithm}{\thechapter.\arabic{algorithm} #1}}
  {\endalgorithm}

\makeatother

\algrenewcommand\algorithmicrequire{\textbf{Вход:}}
\algrenewcommand\algorithmicensure{\textbf{Выход:}}
\algnewcommand\True{\textbf{true}\space}
\algnewcommand\False{\textbf{false}\space}
\algnewcommand\And{\textbf{and}\space}

\newcommand{\xfor}[3]{#1 \textbf{from} #2 \textbf{to} #3}
\newcommand{\xassign}[2]{\State #1 $\leftarrow$ #2}
\newcommand{\xstate}[1]{\State #1}
\newcommand{\xreturn}[1]{\xstate{\textbf{return} #1}}

\DeclarePairedDelimiter\ceil{\lceil}{\rceil}
\DeclarePairedDelimiter\floor{\lfloor}{\rfloor}

\newcommand{\bigO}[1]{\mathcal{O}\left(#1\right)}

\title{Алгоритмы. Домашнее задание №12}
\author{Горбунов Егор Алексеевич}

\begin{document}
\maketitle

\task{степень одиночества}
\textit{
$LR(T) = \frac{\text{Количество единственных детей в } T}{\text{Количество вершин в } T}$. Ребёнок единственный, если он один такой у своего родителя. Корень не является единственным ребёнком. 
	\begin{enumerate}[label=(\alph*)]
		\item \label{task1:a} Покажите, что в любом ненулевом $AVL$-дереве $T$ $LR(T)\leq \frac{1}{2}$
		\item Правда ли, что если $LR(T) \leq \frac{1}{2}$, то $h(T) \leq \log{size(T)}$?
		\item Пусть в дереве $T$ есть всего $\Theta(size(T))$ единственных детей и все они --- листья. Правда ли, что для любого такого дерева $h(T)~\leq~\log{size(T)}$?
	\end{enumerate}
}
\underline{Решение:}
Обозначим $LC(T)$ --- множество единственных детей в $T$.
\begin{enumerate}[label=(\alph*)]
% Пункт (а)
	\item В $AVL$-дереве на $1$ вершине $LR(T)=0 \leq \frac{1}{2}$. На $2$ вершинах $LR(T) = \frac{1}{2}$. Пускай утверждение верно для всех $AVL$-деревьев на $1,2,\ldots,n-1$ вершине. Рассмотрим $AVL$-дерево $T$ на $n > 2$ вершинах пусть $T_l$ и $T_r$ --- это левое и правое поддерево, подвешенные к корню $T$. Т.к. высоты $T_l$ и $T_r$ отличаются максимум на $1$, а $n~>~2$, то оба $T_l$ и $T_r$ --- непустые $AVL$-деревья. Пусть размер $T_l$ --- $n_l < n$, а размер $T_r$ --- $n_r < n$. Заметим, что индукционное предположение для $T_l$ и $T_r$ верно и $n = n_l + n_r + 1$.
	\begin{equation*}
	\begin{split}
		LR(T) &= \frac{|LC(T_l)| + |LC(T_r)|}{n} = \frac{n_ln_r}{n}\Big(\frac{|LC(T_l)|}{n_ln_r} + \frac{|LC(T_r)|}{n_ln_r}\Big) \leq^{по предположению} \frac{n_ln_r}{n}\Big(\frac{1}{2n_l} + \frac{1}{2n_r}\Big) =
		\\ &= \frac{1}{2}\frac{n_l+n_r}{n} = \frac{1}{2}\frac{n - 1}{n} \leq \frac{1}{2} 
	\end{split}
	\end{equation*}
	Итого по индукции показали, что утверждение \ref{task1:a} верно. \xqed
% Пункт (b)
	\item Нет, не правда. Рассмотрим дерево $T$ на $n$ вершинах (в данном случае $n$ чётное) изображённое на рисунке~\ref{fig:task1:b}. Высота этого дерева равна $\frac{n}{2}+1$, но при этом единственный ребёнок лишь один --- это лист $n$. Таким образом $LR(T) = \frac{1}{n} \leq {1}{2}$, но при этом высота $h(T) = \frac{n}{2} + 1 > \log{n}$ \xqed
	\begin{figure}[ht!]
	\centering
	\begin{tikzpicture}[scale=0.85]
	\GraphInit[vstyle=Classic]
 	\tikzstyle{every node}=[font=\scriptsize]
	\begin{scope}[VertexStyle/.append style = {minimum size = 5pt, inner sep = 0pt, color=black}]
        \Vertex[Math,x=0,y=0]{1}   
        \Vertex[Math,x=1,y=-1]{3}
        \Vertex[Math,x=3,y=-3]{n-3}
        \Vertex[Math,x=4,y=-4]{n-1}
		\Vertex[Math,x=-0.5,y=-1]{2}
		\Vertex[Math,x=0.5,y=-2]{4}
		\Vertex[Math,x=2.5,y=-4]{n-2}
	\end{scope}
	\begin{scope}[VertexStyle/.append style = {minimum size = 5pt, inner sep = 0pt, color=red}]
		\Vertex[Math,x=3.5,y=-5]{n}
	\end{scope}

	\Edge(1)(3)
	\Edge[style=dashed](3)(n-3)
	\Edge(n-3)(n-1)
	\Edge(1)(2)
	\Edge(3)(4)
	\Edge(n-3)(n-2)
	\Edge(n-1)(n)

	\end{tikzpicture}
	\caption{К задаче 1 (b)}
	\label{fig:task1:b}
	\end{figure}

% Пункт $(c)$
	\item Обозначим $n = size(T)$. Рассмотрим дерево изображённое на рисунке~\ref{fig:task1:c}. В этом дереве $n = 3m+1$ вершина. Высота этого дерева равна $m+2$ ($m$ <<веточек>> влево и ещё 2 уровня под корень и его сына $2$). Ясно, что в таком графе одиноких детей ровно $m$, они выделены красным на изображении и каждый из них является листом, т.е. дерево $T$ удовлетворяет условию задачи ($m = \floor{\frac{n}{3}}$). Но, как было сказано выше, высота этого дерева равна $m+2= \floor{\frac{n}{3}} + 2 \geq \log{n}$ начиная с некоторого $n$. Таким образом утверждение пункта задачи неверно! \xqed

	\begin{figure}[ht!]
	\centering
	\begin{tikzpicture}[scale=0.8]
	\GraphInit[vstyle=Classic]
 	\tikzstyle{every node}=[font=\scriptsize]
	\begin{scope}[VertexStyle/.append style = {minimum size = 5pt, inner sep = 0pt, color=black}]
        \Vertex[Math,x=0,y=0]{1}  
		\Vertex[Math,x=-0.5,y=-1]{2}
        \Vertex[Math,x=1,y=-1]{4}
		\Vertex[Math,x=0.5,y=-2]{5}
        \Vertex[Math,x=3,y=-3]{3k+1}
		\Vertex[Math,x=2.5,y=-4]{3k+2}
        \Vertex[Math,x=5,y=-5]{n-3}
		\Vertex[Math,x=4.5,y=-6]{n-2}
		\Vertex[Math,x=7,y=-7]{n}
	\end{scope}
		\begin{scope}[VertexStyle/.append style = {minimum size = 5pt, inner sep = 0pt, color=red}]
		\Vertex[Math,x=-1,y=-2]{3}
		\Vertex[Math,x=0,y=-3]{6}
		\Vertex[Math,x=2,y=-5]{3k+3}
		\Vertex[Math,x=4,y=-7]{n-1}
	\end{scope}

	\Edge(1)(2)
	\Edge(1)(4)
	\Edge(2)(3)
	\Edge(4)(5)
	\Edge(5)(6)
	\Edge[style=dashed](4)(3k+1)
	\Edge(3k+1)(3k+2)
	\Edge(3k+2)(3k+3)
	\Edge[style=dashed](3k+1)(n-3)
	\Edge(n-3)(n-2)
	\Edge(n-2)(n-1)
	\Edge(n-3)(n)
	\end{tikzpicture}
	\caption{К задаче 1 (с)}
	\label{fig:task1:c}
	\end{figure}
\end{enumerate}

\task{Слияния $AVL$-деревьев}
\textit{
Пусть даны $2$ $AVL$-дерева $T_1$ и $T_2$. Придумать, как построить $AVL$-дерево $T$, являющееся объединением деревьев $T_1$ и $T_2$ за время:
	\begin{enumerate}[label=(\alph*)]
		\item $\bigO{h(T_1)size(T_2)}$
		\item $\bigO{\max{\lbrace size(T_1), size(T_2) \rbrace}}$
	\end{enumerate}
}
\underline{Решение:}
\begin{enumerate}[label=(\alph*)]
	\item Не умаляя общности $size(T_1) \geq size(T_2)$. Обойдём дерево $T_2$ за $\bigO{size(T_2)}$ и в порядке обхода будем добавлять вершины из $T_2$ в $T_1$. Заметим, что размер дерева $T_2$ увеличится, во время выполнения такой процедуры, не более чем в $2$ раза, т.к. $size(T_1) \geq size(T_2)$. Но значит высота
	дерева может измениться лишь на константу, т.к. в $AVL$-дереве на $n$ вершинах равна $\Theta(\log{n})$, тогда высота дерева на $2n$ вершинах 
	$\Theta(\log{2}\log{n})=\Theta(\log{n})$. Таким образом на каждую операцию добавления в $T_1$ уйдёт $\bigO{\log{size(T_1)}} = \bigO{h(T_1)}$, а на всё объединение $\bigO{h(T_1)size(T_2)}$ \xqed

	\item
\end{enumerate}

\task{Порядковые статистики}
\textit{
	\begin{enumerate}[label=(\alph*)]
		\item Придумайте, как в $AVL$-дереве $T$ реализовать операцию получения $k$-ой порядковой статистики за $\bigO{h(T)}$
		\item Придумайте, как в $AVL$-дереве $T$ найти $index(k)$ --- позицию ключа $k$ в отсортированном массиве ключей за $\bigO{h(T)}$
		\item Придумайте, как в $AVL$-дереве $T$ найти количество ключей между $k_1$ и $k_2$ за $\bigO{h(T)}$
	\end{enumerate}
}
\underline{Решение:}
Будем в каждой вершине $v$ дерева $T$ дополнительно хранить размер поддерева $v.size$ с корнем в $v$, включая саму вершину $v$. Покажем, что размер поддерева легко поддерживать сохраняя асимптотику
операций на $AVL$-дереве. Действительно, в стандартной реализации $AVL$-дерева мы вынуждены поддерживать в каждой вершине высоту поддерева, аналогично ей будем поддерживать и размер поддерева. При добавлении вершины мы соответственно увеличиваем в каждой вершине, которая была затронута при добавлении (путь до места, куда добавляем), размер поддерева. Т.е. после этой части операции размеры корректны. Далее корректность может нарушаться при балансировке, но она легко чинится из $\bigO{1}$, т.к. при поворотах, которые используются в балансировке, корректность параметра размера может нарушиться лишь в паре вершин. Причём корректность легко восстановить посчитав размеры заново, исходя из размеров левых и правых поддеревьев вершин (нужно, собственно, их сложить и ещё 1 добавить), в которых она нарушилась. Таким образом мы умеем поддерживать в каждой вершине размер поддерева корнем в этой вершине.

\begin{enumerate}[label=(\alph*)]
	\item Найдём $k$-ую порядковую статистику. Т.е. найдём в дереве $T$ вершину $v$ с ключом $x$ таким, что в дереве ровно $k-1$ элемент меньше $x$. Следующая процедура это делает:
	\begin{algorithmic}
	\Procedure{$kStatistic$}{$T, k$}
		\xassign{$cur$}{$T.root$}
		\xassign{$numLess$}{$cur.l.size$}
		\While{$cur \neq nil$ AND $numLess \neq k-1$}
			\If{$numLess < k - 1$}
				\xassign{$cur$}{$cur.r$}
				\xassign{$numLess$}{$numLess + 1 + cur.l.size$}
			\EndIf
			\If{$numless > k - 1$}
				\xassign{$cur$}{$cur.l$}
				\xassign{$numLess$}{$numLess - 1 - cur.r.size$}
			\EndIf
		\EndWhile
		\xreturn{$cur.key$}
	\EndProcedure
	\end{algorithmic}
	На каждом шаге приведённой процедуры поддерживается инвариант $numLess$ --- число ключей в дереве, которые меньше, чем $cur.key$. Вначале мы стоим в корне $T.root$ и число ключей, меньших $T.root.key$ равно $size(T.root.l)$. Далее $numLess$ легко поддерживать. Пускай мы из $cur$ перешли в $cur.r$, тогда $numLess$ стало больше на число вершин, непосредственно подвешенных слева от $cur.r$, аналогично для перехода в $cur.l$. Ясно, что если в дереве $T$ есть $\geq k$ вершин, которые меньше $T.root.key$, что $k$-ая порядковая статистика где-то в левом поддереве от корня, иначе она находится в правом поддереве, таким образом оправданы переходы по условиям $numLess < k -1$ и $numLess > k-1$. Как только мы находим ключ такой, что в дереве строго $k-1$ ключ, меньше его, то значит, что мы нашли $k$-ую порядковую статистику.\\
	На каждом шаге цикла мы делаем один спуск на уровень ниже по дереву. Т.е. цикл отработает максимум за $h(T)$ операций, а значит время работы алгоритмы на $AVL$-дереве равно $\bigO{\log{size(T)}}$ \xqed

	\item Нужно найти $index(k)$. Будем искать переданный ключ $k$ в нашем дереве стандартной процедурой, но только будем поддерживать, аналогично предыдущему пункту, инвариант $numLess$ --- число ключей в дереве, строго меньших $cur.key$, где $cur$ --- вершина, рассматриваемая в качестве кандидата при поиске в данный момент. Вначале стоим в корне и $numLess$ равен числу ключей в дереве, меньших корня, т.е. $numLess$ есть индекс (если нумерация с 0) $T.root.key$ в отсортированном массиве. Пересчёт $numLess$ при переходах по дереву такой же, как и в предыдущем пункте.  Таким образом, как только стандартный алгоритм поиска натыкается на вершину с нужным ключом, так сразу возвращаем $numLess$. Алгоритм отработает за столько же, за сколько работает обычный поиск, т.е. за $\bigO{h(T)} = \bigO{\log{size(T)}}$ \xqed
	\item Ясно, что число элементов между ключом $k_1$ и $k_2$ равно $index(k_2)-index(k_1)-1$, но $index(k)$ мы умеем находить за $\bigO{\log{size(T)}}$ по предыдущему пункту. \xqed
\end{enumerate}
\end{document}
