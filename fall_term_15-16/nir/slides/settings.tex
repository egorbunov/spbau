% ALERT: Almost all of the settings are copy-paste from https://github.com/matze/mtheme

\documentclass{beamer}
\mode<presentation>

\usepackage[russian]{babel}
\usepackage[no-math]{fontspec}
\usepackage{xunicode}
\usepackage{xltxtra}

\usepackage{pdfpages}
\usepackage{tikz}
\usepackage{xcolor}
\usepackage{calc}

\usepackage{etoolbox}
\usepackage{pgfplots}
\usepackage{ifxetex, ifluatex}

\usepackage{pifont}% http://ctan.org/pkg/pifont

% ============ LENGTHS ============
\setlength{\parskip}{4pt plus 1pt minus 1pt}
\linespread{1.15}

% ============ ENVIRONMENTS ============


% ============ COLORS, FONTS AND FONTS SETTINGS ============

% colors
\definecolor{cTitle}{HTML}{23373B}
\definecolor{cBackground}{HTML}{FAFAFA}
\definecolor{cSeparator}{HTML}{EB811B}
\definecolor{cText}{HTML}{23373B}
\definecolor{cFrameBackground}{HTML}{23373B}

\definecolor{cGreen}{HTML}{007A29}
\definecolor{cRed}{HTML}{CC3300}

% new font families
\newfontfamily\ExtraLight{Fira Sans ExtraLight}
\newfontfamily\Light{Fira Sans Light}
\newfontfamily\Book{Fira Sans}

% setting spectial colors
\setbeamercolor{background canvas}{bg=cBackground}
\setbeamercolor{frametitle}{fg=cBackground,bg=cFrameBackground}
\setbeamercolor{title}{fg=cTitle}
\setbeamercolor{progress bar}{fg=cSeparator}
\setbeamercolor{title separator}{fg=cSeparator}
\setbeamercolor{section title}{fg=cTitle}
\setbeamercolor{normal text}{fg=cText, bg=cBackground}
\setbeamercolor{caption name}{fg=cText}
\setbeamercolor{block title}{fg=cText}
\setbeamercolor{block title alerted}{fg=cSeparator}
\setbeamercolor{alerted text}{fg=cSeparator}
\setbeamercolor{alerted text}{fg=cSeparator}
\setbeamercolor{item projected}{fg=cText}
\setbeamercolor{enumerate item}{fg=cText}
\setbeamercolor{enumerate subitem}{fg=cText}
\setbeamercolor{enumerate subsubitem}{fg=cText}
\setbeamercolor{itemize item}{fg=cText}
\setbeamercolor{itemize subitem}{fg=cText}
\setbeamercolor{itemize subsubitem}{fg=cText}

\setbeamercolor{palette primary}{
  use=normal text,
  fg=normal text.bg,
  bg=normal text.fg
}

\setbeamercolor{bibliography item}{parent=palette primary}
\setbeamercolor{bibliography entry author}{fg=cText}
\setbeamercolor{bibliography entry title}{fg=cText}
\setbeamercolor{bibliography entry location}{fg=cText}
\setbeamercolor{bibliography entry note}{fg=cText}

% setting spectial fonts
\defaultfontfeatures{Mapping=tex-text}
\setsansfont[BoldItalicFont={Fira Sans Italic},%
             ItalicFont={Fira Sans Light Italic},%
             BoldFont={Fira Sans}]{Fira Sans Light}
\setmonofont{Fira Mono}

\setbeamerfont{institute}{family=\ExtraLight, size=\footnotesize}
\setbeamerfont{title}{family=\Book, size=\Large}
\setbeamerfont{author}{family=\ExtraLight, size=\small}
\setbeamerfont{date}{family=\ExtraLight, size=\small}
\setbeamerfont{section title}{family=\Book, size=\Large}
\setbeamerfont{block title}{family=\Book, size=\normalsize}
\setbeamerfont{block title alerted}{family=\Book,size=\normalsize}
\setbeamerfont{subtitle}{family=\Light, size=\fontsize{12}{14}}
\setbeamerfont{frametitle}{family=\Book, size=\large}
\setbeamerfont{framenumber}{family=\ExtraLight, size=\scriptsize}
\setbeamerfont{caption}{size=\small}
\setbeamerfont{caption name}{family=\Book}
\setbeamerfont{description item}{family=\Book}
\setbeamerfont{page number in head/foot}{size=\scriptsize}
\setbeamerfont{bibliography entry author}{family=\Light, size=\normalsize}
\setbeamerfont{bibliography entry title}{family=\Book, size=\normalsize}
\setbeamerfont{bibliography entry location}{family=\Light, size=\normalsize}
\setbeamerfont{bibliography entry note}{family=\Light, size=\small}

% ============ NEW LENGTHS ============

\makeatletter
\newlength{\progressbar@percent}

\newlength{\theme@voffset}
\setlength{\theme@voffset}{2cm}
\makeatother

% ============ NEW COMMANDS ============
\newcommand{\additionalbegin}{
   \newcounter{finalframe}
   \setcounter{finalframe}{\value{framenumber}}
}
\newcommand{\additionalend}{
   \setcounter{framenumber}{\value{finalframe}}
}


\newcommand{\red}[1]{\textcolor{cRed}{#1}}
\newcommand{\green}[1]{\textcolor{cGreen}{#1}}

\newcommand{\iplus}{\item[\color{cGreen}$+$]} 
\newcommand{\iminus}{\item[\color{cRed}$-$]} 

\newcommand{\cmark}{\green{\ding{51}}}%
\newcommand{\xmark}{\red{\ding{55}}}%

\newcommand{\code}[1]{\colorbox{gray!15}{\footnotesize\texttt{#1}}}

% plain slide
\newcommand{\plain}[2][]{%
  {
    \setbeamercolor{background canvas}{bg=cText}
    \begingroup
      \begin{frame}{#1}
        \centering
        \vfill
        \vspace{1em}
        \usebeamercolor[fg]{palette primary}
        \usebeamerfont{section title}
        \scshape #2
        \vfill
      \end{frame}
    \endgroup
  }
}

% progress bar line
\makeatletter
\newcommand{\progressbar}[1]{%
  \setlength{\progressbar@percent}{%
    #1 * \ratio{\insertframenumber pt}{\inserttotalframenumber pt}%
  }%
  \begin{tikzpicture}
    \usebeamercolor{progress bar}
    \draw[bg, fill=bg] (0,0) rectangle (#1, 0.4pt);
    \draw[fg, fill=fg] (0,0) rectangle (\progressbar@percent, 0.4pt);
  \end{tikzpicture}%
}

\newif\if@notAddFrameNumberOnTitlePage
\@notAddFrameNumberOnTitlePagetrue

\def\maketitle{
  {
  \if@notAddFrameNumberOnTitlePage
    \setbeamertemplate{navigation symbols}{}
    \setbeamertemplate{footline}{}
  \fi
  \ifbeamer@inframe
    \titlepage
  \else
    \frame{\titlepage}
  \fi
  }
}
\makeatother

\newcommand{\insertsectionHEADaux}[3]{\scshape #3}
\newcommand{\insertsectionHEAD}{\expandafter\insertsectionHEADaux\insertsectionhead}

% PIE CHART
\definecolor{rosso}{RGB}{220,57,18}
\definecolor{giallo}{RGB}{255,153,0}
\definecolor{blu}{RGB}{102,140,217}
\definecolor{verde}{RGB}{16,150,24}
\definecolor{viola}{RGB}{153,0,153}

\makeatletter

\tikzstyle{chart}=[
    legend label/.style={font={\scriptsize},anchor=west,align=left},
    legend box/.style={rectangle, draw, minimum size=5pt},
    axis/.style={black,semithick,->},
    axis label/.style={anchor=east,font={\tiny}},
]

\tikzstyle{bar chart}=[
    chart,
    bar width/.code={
        \pgfmathparse{##1/2}
        \global\let\bar@w\pgfmathresult
    },
    bar/.style={very thick, draw=white},
    bar label/.style={font={\bf\small},anchor=north},
    bar value/.style={font={\footnotesize}},
    bar width=.75,
]

\tikzstyle{pie chart}=[
    chart,
    slice/.style={line cap=round, line join=round, very thick,draw=white},
    pie title/.style={font={\bf}},
    slice type/.style 2 args={
        ##1/.style={fill=##2},
        values of ##1/.style={}
    }
]

\pgfdeclarelayer{background}
\pgfdeclarelayer{foreground}
\pgfsetlayers{background,main,foreground}


\newcommand{\pie}[3][]{
    \begin{scope}[#1]
    \pgfmathsetmacro{\curA}{90}
    \pgfmathsetmacro{\r}{1}
    \def\c{(0,0)}
    \node[pie title] at (90:1.3) {#2};
    \foreach \v/\s in{#3}{
        \pgfmathsetmacro{\deltaA}{\v/100*360}
        \pgfmathsetmacro{\nextA}{\curA + \deltaA}
        \pgfmathsetmacro{\midA}{(\curA+\nextA)/2}

        \path[slice,\s] \c
            -- +(\curA:\r)
            arc (\curA:\nextA:\r)
            -- cycle;
        \pgfmathsetmacro{\d}{max((\deltaA * -(.5/50) + 1) , .5)}

        \begin{pgfonlayer}{foreground}
        \path \c -- node[pos=\d,pie values,values of \s]{$\v\%$} +(\midA:\r);
        \end{pgfonlayer}

        \global\let\curA\nextA
    }
    \end{scope}
}

\newcommand{\legend}[2][]{
    \begin{scope}[#1]
    \path
        \foreach \n/\s in {#2}
            {
                  ++(0,-10pt) node[\s,legend box] {} +(5pt,0) node[legend label] {\n}
            }
    ;
    \end{scope}
}

% ============ OVERRIDED BEAMER TEMPLATES ============
\makeatletter
\newcommand{\itemBullet}{∙}
\setbeamertemplate{itemize item}{\itemBullet}
\setbeamertemplate{itemize subitem}{\itemBullet}
\setbeamertemplate{itemize subsubitem}{\itemBullet}
\makeatother

% for progress bar better view (?)
\def\inserttotalframenumber{100} 

% section page template
\AtBeginSection[]
{
  \frame{\sectionpage}
}

\makeatletter
\setbeamertemplate{section page}
{
  \vspace{2em}
  \centering
  \begin{minipage}{22em}
    \usebeamercolor[fg]{section title}
    \usebeamerfont{section title}
    \insertsectionHEAD\\[-1ex]
    \progressbar{\textwidth}
  \end{minipage}
}
\makeatother

% frame title template
\makeatletter
\newif\if@useTitleProgressBar
\@useTitleProgressBartrue
% \@useTitleProgressBarfalse % uncomment to switch progress bar off on frame titles

\setbeamertemplate{frametitle}
{
  \nointerlineskip
  \begin{beamercolorbox}[wd=\paperwidth,leftskip=0.3cm,rightskip=0.3cm,ht=2.5ex,dp=1.5ex]{frametitle}
    \usebeamerfont{frametitle}%
    \scshape \protect\insertframetitle%
  \end{beamercolorbox}%

  \if@useTitleProgressBar
    \nointerlineskip
    \begin{beamercolorbox}[wd=\paperwidth,ht=0.4pt,dp=0pt]{frametitle}
      \progressbar{\paperwidth}
    \end{beamercolorbox}
  \fi
}
\makeatother


% title page template
\makeatletter
\setbeamertemplate{title page}
{
    \ifx\insertinstitute\@empty\else
    % \insertinstitute is nonempty
    \vspace*{0.5cm}
    \begin{minipage}[b][\paperheight]{\textwidth}
    \begin{minipage}[c]{\textwidth}
      {{
        \begin{center}
        \usebeamerfont{institute}%
        \usebeamercolor[fg]{institute}%
        \insertinstitute%
        \end{center}
      }}
    \end{minipage}
    \fi

  	\vspace*{1cm}

    \ifx\inserttitlegraphic\@empty\else
    {% \inserttitlegraphic is nonempty
      \vbox to 0pt
      {% display title graphic without changing the position of other elements
        \vspace*{2em}
        \usebeamercolor[fg]{titlegraphic}%
        \inserttitlegraphic%
      }%
      \nointerlineskip%
    }
    \fi

    \vfill%

	\ifx\inserttitle\@empty\else
    {{% \inserttitle is nonempty
      \raggedright%
      \linespread{1.0}%
      \usebeamerfont{title}%
      \usebeamercolor[fg]{title}%
      \begin{center}
      \scshape \inserttitle%
      \end{center}
    }}
    \fi

    \ifx\insertsubtitle\@empty\else
    {{% \insertsubtitle is nonempty
      \usebeamerfont{subtitle}%
      \usebeamercolor[fg]{subtitle}%
      \insertsubtitle%
      \vspace*{0.5em}%
    }}
    \fi

    \begin{tikzpicture}
      \usebeamercolor{title separator}
      \draw[fg] (0, 0) -- (\textwidth, 0);
    \end{tikzpicture}%
    \vspace*{1em}%

    \ifx\beamer@shortauthor\@empty\else
    {{% \insertauthor is always nonempty by beamer's definition, so we must
      % test another macro which is initialized by \author{...}
      % For details, see http://tex.stackexchange.com/questions/241306/
      \usebeamerfont{author}%
      \usebeamercolor[fg]{author}%
      \insertauthor%
      \par%
      \vspace*{0.25em}
    }}
    \fi

    \ifx\insertdate\@empty\else
    {{% \insertdate is nonempty
      \usebeamerfont{date}%
      \usebeamercolor[fg]{date}%
      \insertdate%
      \par%
    }}
    \fi

    \vfill
    \vspace*{2cm}

  \end{minipage}
}
\makeatother

% footline
\setbeamertemplate{footline}{
  \begin{minipage}[b]{0.99\textwidth}
    \flushright
    \raisebox{0.15cm}[0pt][0pt]{
      \usebeamerfont{framenumber}
      \usebeamercolor[fg]{title}

      \insertframenumber
      % \insertframenumber/\inserttotalframenumber
    }
    \end{minipage}
}

% switching off navigation sybmols
\makeatletter
\setbeamertemplate{navigation symbols}{}
\makeatother

% cpation
\setbeamertemplate{caption label separator}{: }
\setbeamertemplate{caption}[numbered]

% ============ PGFPLOTS ENVIRONMENTS ============
\definecolor{TolDarkPurple}{HTML}{332288}
\definecolor{TolDarkBlue}{HTML}{6699CC}
\definecolor{TolLightBlue}{HTML}{88CCEE}
\definecolor{TolLightGreen}{HTML}{44AA99}
\definecolor{TolDarkGreen}{HTML}{117733}
\definecolor{TolDarkBrown}{HTML}{999933}
\definecolor{TolLightBrown}{HTML}{DDCC77}
\definecolor{TolDarkRed}{HTML}{661100}
\definecolor{TolLightRed}{HTML}{CC6677}
\definecolor{TolLightPink}{HTML}{AA4466}
\definecolor{TolDarkPink}{HTML}{882255}
\definecolor{TolLightPurple}{HTML}{AA4499}
\pgfplotscreateplotcyclelist{mbarplot cycle}{%
  {draw=TolDarkBlue,    fill=TolDarkBlue!70},
  {draw=TolLightBrown,  fill=TolLightBrown!70},
  {draw=TolLightGreen,  fill=TolLightGreen!70},
  {draw=TolDarkPink,    fill=TolDarkPink!70},
  {draw=TolDarkPurple,  fill=TolDarkPurple!70},
  {draw=TolDarkRed,     fill=TolDarkRed!70},
  {draw=TolDarkBrown,   fill=TolDarkBrown!70},
  {draw=TolLightRed,    fill=TolLightRed!70},
  {draw=TolLightPink,   fill=TolLightPink!70},
  {draw=TolLightPurple, fill=TolLightPurple!70},
  {draw=TolLightBlue,   fill=TolLightBlue!70},
  {draw=TolDarkGreen,   fill=TolDarkGreen!70},
}
\pgfplotscreateplotcyclelist{mlineplot cycle}{%
  {TolDarkBlue, mark=*, mark size=1.5pt},
  {TolLightBrown, mark=square*, mark size=1.3pt},
  {TolLightGreen, mark=triangle*, mark size=1.5pt},
  {TolDarkBrown, mark=diamond*, mark size=1.5pt},
}
\pgfplotsset{
  compat=1.9,
  mlineplot/.style={
    mbaseplot,
    xmajorgrids=true,
    ymajorgrids=true,
    major grid style={dotted},
    axis x line=bottom,
    axis y line=left,
    legend style={
      cells={anchor=west},
      draw=none
    },
    cycle list name=mlineplot cycle,
  },
  mbarplot base/.style={
    mbaseplot,
    bar width=6pt,
    axis y line*=none,
  },
  mbarplot/.style={
    mbarplot base,
    ybar,
    xmajorgrids=false,
    ymajorgrids=true,
    area legend,
    legend image code/.code={%
      \draw[#1] (0cm,-0.1cm) rectangle (0.15cm,0.1cm);
    },
    cycle list name=mbarplot cycle,
  },
  horizontal mbarplot/.style={
    mbarplot base,
    xmajorgrids=true,
    ymajorgrids=false,
    xbar stacked,
    area legend,
    legend image code/.code={%
      \draw[#1] (0cm,-0.1cm) rectangle (0.15cm,0.1cm);
    },
    cycle list name=mbarplot cycle,
  },
  mbaseplot/.style={
    legend style={
      draw=none,
      fill=none,
      cells={anchor=west},
    },
    x tick label style={
      font=\footnotesize
    },
    y tick label style={
      font=\footnotesize
    },
    legend style={
      font=\footnotesize
    },
    major grid style={
      dotted,
    },
    axis x line*=bottom,
  },
  disable thousands separator/.style={
    /pgf/number format/.cd,
      1000 sep={}
  },
}
\endinput