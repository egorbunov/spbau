\documentclass[12pt, a4paper]{article}
\usepackage[utf8]{inputenc}
\usepackage[russian]{babel}
\usepackage{pscyr}

\usepackage{xifthen}
\usepackage{parskip}
\usepackage{hyperref}
\usepackage[top=0.7in, bottom=1in, left=0.6in, right=0.6in]{geometry}
\usepackage{setspace}

\usepackage{amsmath}
\usepackage{MnSymbol}
\usepackage{amsthm}
\usepackage{mathtools}

\usepackage{algorithm}
\usepackage[noend]{algpseudocode}



\linespread{1.2}
\setlength{\parskip}{0pt}

\renewcommand\familydefault{\sfdefault}


% Stuff related to homework specific documents
\newcounter{MyTaskCounter}
\newcounter{MyTaskSectionCounter}
\newcommand{\tasksection}[1]{
	\stepcounter{MyTaskSectionCounter}
	\setcounter{MyTaskCounter}{0}
	\ifthenelse{\equal{#1}{}}{}{
	{\hfill\\[0.2in] \Large \textbf{\theMyTaskSectionCounter \enspace #1} \hfill\\[0.1in]}}
}

\newcommand{\task}[1]{
	\stepcounter{MyTaskCounter}
	\hfill\\[0.1in]
	\ifthenelse{\equal{\theMyTaskSectionCounter}{0}}{
	   \textbf{\large Задача №\theMyTaskCounter}
	}{
	   \textbf{\large Задача №\theMyTaskSectionCounter.\theMyTaskCounter}
	}
	\ifthenelse{\equal{#1}{}}{}{{\normalsize (#1)}}
	\hfill\\[0.05in]
}

% Math and algorithms

\makeatletter
\renewcommand{\ALG@name}{Алгоритм}
\renewcommand{\listalgorithmname}{Список алгроитмов}

\newenvironment{procedure}[1]
  {\renewcommand*{\ALG@name}{Процедура}
  \algorithm\renewcommand{\thealgorithm}{\thechapter.\arabic{algorithm} #1}}
  {\endalgorithm}

\makeatother

\algrenewcommand\algorithmicrequire{\textbf{Вход:}}
\algrenewcommand\algorithmicensure{\textbf{Выход:}}
\algnewcommand\True{\textbf{true}\space}
\algnewcommand\False{\textbf{false}\space}
\algnewcommand\And{\textbf{and}\space}

\newcommand{\xfor}[3]{#1 \textbf{from} #2 \textbf{to} #3}
\newcommand{\xassign}[2]{\State #1 $\leftarrow$ #2}
\newcommand{\xstate}[1]{\State #1}
\newcommand{\xreturn}[1]{\xstate{\textbf{return} #1}}

\DeclarePairedDelimiter\ceil{\lceil}{\rceil}
\DeclarePairedDelimiter\floor{\lfloor}{\rfloor}

\newcommand{\bigO}[1]{\mathcal{O}\left(#1\right)}


\title{Haskell. Домашнее задание №1}
\author{Горбунов Егор Алексеевич}

\begin{document}
\maketitle

\tasksection{Долг с пары}
\task{функция $or$}
\[or\ x\ y = x\ true\ y\]
Если $x = true = \lambda ab.a$, то $or\ x\ y$ вернёт $true$ (т.е. $x$), иначе будет возвращён $y$, который и будет 
верным ответом, т.к. теперь всё от него и зависит (если $y = true$, то $or\ x\ y\ = true$ и с $false$ аналогично).

\tasksection{Примитивная рекурсия}
\[
prim\_rec\ b\ z\ n = snd\ (n\ (\lambda p.pair\ (succ\ (fst\ p))(p\ b))\ (pair\ \overline{0}\ z))
\]
Суть: применяем $b$ $n$ раз к $z$ (каждое следующее применение производится к результату предыдущего). Исходя их этого будет в следующих 
заданиях строить требуемые функции.
\task{факториал}
\[
fact\ n = prim\_rec\ (\lambda xy.mult\ (succ\ x)\ y)\ \overline{1}\ n
\]
Для $n = \overline{0}$ ($\overline{0}=zero$ --- число Чёрча): $fact\ 0 = snd\ (pair\ \overline{0}\ \overline{1})) = \overline{1}$ (уд. определению факториала).
Для $n = \overline{1}$: 
\begin{equation*}
\begin{split}
fact\ 1 &= snd\ (pair\ (succ\ (fst\ (pair\ \overline{0}\ \overline{1}))((pair\ \overline{0}\ \overline{1})\ (\lambda xy.mult\ (succ\ x)\ y)) \\
        &= snd\ (pair\ (succ\ \overline{0})\ (mult\ (succ\ \overline{0})\ \overline{1})) \\
        &= \overline{1}
\end{split}
\end{equation*}
Аналогично для $n > \overline{1}$. \hfill $\blacksquare$

{\color{red}\task{предыдущее число}}
На каждом шаге рекурсии во второй элемент пары будем класть текущее состояние счётчика. \\
При применении абстракции $b$ (параметр $prim\_rec$) счётчик передаётся первым параметром. Итого:
\[
pred\ n = prim\_rec\ (\lambda p.fst\ p)\ \overline{0}\ n
\]
$n=\overline{0}$: $pred\ \overline{0} = \overline{0} = snd\ (pair\ \overline{0}\ \overline{0}))$\\
$n=\overline{1}$: 
\begin{equation*}
\begin{split}
pred\ \overline{1} &= snd\ ((\lambda p.pair\ (succ\ (fst\ p))(p\ (\lambda p'.fst\ p')\ (pair\ \overline{0}\ \overline{0}))\\
			       &= snd\ (pair\ (succ\ (fst\ (pair\ \overline{0}\ \overline{0})))((pair\ \overline{0}\ \overline{0})\ (\lambda p'.fst\ p')))\\
			       &= snd\ (pair \overline{1}\ \overline{0}) \\
			       &= \overline{0}
\end{split}
\end{equation*}) 
Для $n=\overline{1}$ верно. Далее показывается так же просто.
\task{меньше или равно}
Пользуясь тем, что $pred\ \overline{0} = \overline{0}$ и определением чисел Чёрча получаем:
\[
leq\ a\ b = (ifZero\ (b\ pred\ a))\ true\ false
\]
Т.е. $b$ раз пытаемся вычесть $1$ из $a$. Если число стало в итоге $\overline{0}$, то $a-b \leq 0\ \Rightarrow\ a\ \leq\ b$ \hfill $\blacksquare$
\task{модуль разности}
Имея возможность узнать, какое из чисел больше, можем найти модуль разности вычитанием большего из меньшего:
\[
dist\ a\ b = (leq\ a\ b)\ (a\ pred\ b)\ (b\ pred\ a)
\]
\hfill $\blacksquare$
\tasksection{Списки}
\begin{align*}
	nil &= \lambda cx.x \\
	cons\ a\ as &= \lambda cx.c\ a\ (as\ c\ x)
\end{align*}
\task{$isEmpty$}
\[isEmpty\ l = l\ (\lambda ht.false)\ true\]
Проверим для пустого списка $nil$: $isEmpty\ nil = nil\ (\lambda ht.false)\ true = \lambda cx.x\ (\lambda ht.false)\ 
true =_\beta true$. Ок. Любой непустой список $l = cons\ a_1 (cons\ a_2 (\ldots(cons\ nil)\ldots))$ представляется в виде:
\begin{align}
l &= \lambda cx.c\ a_1\ (T\ c\ x)=\\
  &= \lambda cx.c\ a_1\ (c\ a_2\ (T'\ c\ x)))=\\
  &= \lambda cx.c\ a_1\ (c\ a_2\ (\ldots (c\ a_n\ (nil\ c\ x))\ldots))=\\
  &= \lambda cx.c\ a_1\ (c\ a_2(\ldots\ (c\ a_n\ x)\ldots) \label{eq:list}
\end{align}
Ясно, что если вместо $c$ подставить любую абстракцию с 2 связанными переменными, всегда возвращающую $false$, то $isEmpty\ l$ будет возвращать $false$. \hfill $\blacksquare$
\task{$head$}
\[ head\ l =  l\ true\ nil\]
Для $nil$: $head\ nil = nil\ true\ nil = (\lambda cx.x)\ true\ nil =_\beta nil$. Разумно.\\
Для непустого $l = cons\ a_1\ T$: 
\begin{align*}
head\ l &= (\lambda cx.c\ a_1\ (c\ a_2(\ldots\ (c\ a_n\ x)\ldots))\ true\ nil =_\beta \\
	    &=_\beta true\ a_1\ (\ldots) =_\beta \\
	    &=_\beta a_1
\end{align*}
\hfill $\blacksquare$
\task{$tail$}
Соберём хвост списка $l = [a_1,\ldots,a_{n-1},a_n]$ начиная с конца. Пускай:
\[c\ e\ p\ = pair\ (snd\ p)\ (cons\ e\ (snd\ p)))\]
Тогда, например $c\ a_n\ (pair\ nil\ nil) = pair\ nil\ l'$, где $l'$ --- лист из элемента $a_n$, т.е. в первом элементе пары хранится хвост списка $l'$ из одного элемента. Таким образом последовательные применения $c$ при построении списка (см. формулу \ref{eq:list}) приведут к тому, что после последнего применения $c$ получится пара $pair\ tail\ l$. Таким образом ответ на задачу:
\[tail\ l = fst\ (l\ (\lambda ep.pair\ (snd\ p)\ (cons\ e\ (snd\ p)))\ (pair\ nil\ nil))\]
\hfill $\blacksquare$
{ \color{red} \task{$append$} }
\[ append\ l\ a =  l\ (\lambda a'x.cons\ a'x)\ (\lambda cx.cax)\]
Вторым аргументом идёт лист из одного элемента --- элемента $a$, который мы приделываем к концу списка $l$.
Таким образом видно (легко это видеть из формулы \ref{eq:list}), что мы пересобираем список, запихивая элемент $a$ в конец.
\tasksection{$Y$-комбинатор}
\task{про $F:\ \forall\ M (FM=F)$}
Равенство $F=FM$ $\beta$-эквивалентно следующему: $F=(\lambda fm.f)F$. Видим, что $F$ --- неподвижная точка терма $\lambda fm.f$.
$Y$-комбинатор по определению таков, что: $\forall G:\ (YG=G(YG))$, но тогда $F=YG$, где $G=\lambda fm.f$, итого:
\[
	F = Y\ (\lambda fm.f)
\]
\hfill $\blacksquare$
\task{$fact$ через $Y$}
Определим факториал рекурсивно:
\[
	fact = \lambda n.(isZero\ n)\ 1\ (mult\ n\ (fact\ (pred\ n)))
\]
Теперь абстрагируемся:
\[
 	fact = \overbrace{(\lambda fn.(isZero\ n)\ 1\ (mult\ n\ (f\ (pred\ n))))}^{fact'}\ fact
\]
$fact'$ --- неподвижная точка терма $fact$, таким образом:
\[
	fact = Yfact'
\]
\hfill $\blacksquare$
\task{функция Аккермана}
Функция Аккремана:
\[
A(n, m) = 
\begin{cases}
n + 1 & m = 0 \\
A(m-1,1) & m > 0\ \land\ n = 0 \\
A(m-1,A(m,(n-1))) & m > 0\ \land\ n > 0
\end{cases}
\]
Зададим в терминах $\lambda$-исчисления (абстрагируемся сразу, как в предыдущем задании):
\[
A = \overbrace{(\lambda fnm.(isZero\ m)\ (succ\ n)\ ((isZero\ n)\ (f\ (pred\ m)\ \overline{1})\ (f\ (pred\ m)\ (f\ m\ (pred\ n)))))}^{A'}\ A
\]
Тут $\overline{1}$ --- число Чёрча, соответствующее 1.\\
Аналогично предыдущей задаче: $A = YA'$. \hfill $\blacksquare$
\end{document}
