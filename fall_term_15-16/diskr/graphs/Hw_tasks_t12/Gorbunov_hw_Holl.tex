\documentclass[12pt, a4paper]{article}
\usepackage[utf8]{inputenc}
\usepackage[russian]{babel}
\usepackage{pscyr}

\usepackage{xifthen}
\usepackage{parskip}
\usepackage{hyperref}
\usepackage[top=0.7in, bottom=1in, left=0.6in, right=0.6in]{geometry}
\usepackage{setspace}

\usepackage{amsmath}
\usepackage{MnSymbol}
\usepackage{amsthm}
\usepackage{mathtools}

\usepackage{algorithm}
\usepackage[noend]{algpseudocode}



\linespread{1.2}
\setlength{\parskip}{0pt}

\renewcommand\familydefault{\sfdefault}


% Stuff related to homework specific documents
\newcounter{MyTaskCounter}
\newcounter{MyTaskSectionCounter}
\newcommand{\tasksection}[1]{
	\stepcounter{MyTaskSectionCounter}
	\setcounter{MyTaskCounter}{0}
	\ifthenelse{\equal{#1}{}}{}{
	{\hfill\\[0.2in] \Large \textbf{\theMyTaskSectionCounter \enspace #1} \hfill\\[0.1in]}}
}

\newcommand{\task}[1]{
	\stepcounter{MyTaskCounter}
	\hfill\\[0.1in]
	\ifthenelse{\equal{\theMyTaskSectionCounter}{0}}{
	   \textbf{\large Задача №\theMyTaskCounter}
	}{
	   \textbf{\large Задача №\theMyTaskSectionCounter.\theMyTaskCounter}
	}
	\ifthenelse{\equal{#1}{}}{}{{\normalsize (#1)}}
	\hfill\\[0.05in]
}

% Math and algorithms

\makeatletter
\renewcommand{\ALG@name}{Алгоритм}
\renewcommand{\listalgorithmname}{Список алгроитмов}

\newenvironment{procedure}[1]
  {\renewcommand*{\ALG@name}{Процедура}
  \algorithm\renewcommand{\thealgorithm}{\thechapter.\arabic{algorithm} #1}}
  {\endalgorithm}

\makeatother

\algrenewcommand\algorithmicrequire{\textbf{Вход:}}
\algrenewcommand\algorithmicensure{\textbf{Выход:}}
\algnewcommand\True{\textbf{true}\space}
\algnewcommand\False{\textbf{false}\space}
\algnewcommand\And{\textbf{and}\space}

\newcommand{\xfor}[3]{#1 \textbf{from} #2 \textbf{to} #3}
\newcommand{\xassign}[2]{\State #1 $\leftarrow$ #2}
\newcommand{\xstate}[1]{\State #1}
\newcommand{\xreturn}[1]{\xstate{\textbf{return} #1}}

\DeclarePairedDelimiter\ceil{\lceil}{\rceil}
\DeclarePairedDelimiter\floor{\lfloor}{\rfloor}

\newcommand{\bigO}[1]{\mathcal{O}\left(#1\right)}

\title{Дискретная математика. Задачи про паросочетания.}
\author{Горбунов Егор Алексеевич}

\begin{document}
\maketitle

\subsection*{12.10}
\textit{\textbf{Условие:} пусть в двудольном графе $G=(X,Y)$ подмножество $S\subseteq X$ покрыто паросочетанием $M$, а подмножество $T \subseteq Y$ покрыто паросочетанием $M'$. Доказать, что в $G$ найдётся паросочетание, покрывающее как $S$, так и $T$.}\\
\textit{Решение:} Достаточно объединить паросочетания $M'$ и $M$ (повторившиеся рёбра просто удаляем). Полученное паросочетание покроет как $T$ так и $S$. \xqed
\subsection*{12.11}
\textit{\textbf{Условие:} пусть в двудольном графе $G=(X,Y)$ степень любой вершины блока $X$ больше или равна степени любой вершины $Y$. Доказать, что в этом графе существует $X$-насыщенное паросочетание.}\\
\textit{Решение:} Если граф $G$ не содержит рёбер и $\Delta(Y)=0$, то утверждение не верно, но положим, что $\Delta(Y)>0$. Тогда рассмотрим $S \subseteq X$. Если $|S| \leq \Delta(Y)$, то, т.е. $\forall v \in X\ (d(v) \leq \Delta(Y))$, то
$|N(S)| \geq |S|$. Пускай $|S| > \Delta(Y)$, рассмотрим $N(S)$. Предположим, что $|N(S)| < |S|$. Т.е. из каждой вершины $u \in S$ тянется как минимум $\Delta(Y)$ рёбер, а вершин в $S$ ровно $k = |S| > \Delta(Y)$, то по принципу Дирихле (распихивание $k\Delta(Y)$ <<кроликов>> по $< k$ клеткам) в какой-то вершине $u'\in N(S)$ будет $d(u') > \Delta(Y)$, что противоречит тому, что $\Delta(Y)$ --- максимальная степень вершины в $Y$. Итого, пришли к противоречию, а значит для любого $S\subseteq X\ (|S| \leq |N(S)|)$. \xqed
\subsection*{12.12}
\textit{\textbf{Условие:} пусть в двудольном графе $G=(X,Y)$ $|N(S)| > |S|$ для любого $S \neq \emptyset,\ S \subseteq X$. Доказать, что в таком графе любое ребро принадлежит какому-нибудь $X$-насыщенному паросочетанию. }\\
\textit{Решение:} Возьмём любое ребро $e = (u, v) \in E(G)$, где $u \in X, v \in Y$ (граф двудолен, все рёбра такие) и удалим из $G$ вершину $v$. Если в исходном графе $G$ было для любого $S \subseteq X$ верно, что $|S| < |N(S)|$, то после удаление $v$ (т.е. в графе $G\setminus v$) в $N(S)$ число вершин могло уменьшиться лишь на 1, т.е. $|S| \leq |N(S)|$. По теореме Холла в графе $G\setminus v$ есть $X$-насыщенное паросочетание $M$. Пускай $e' = (u,v')$ --- ребро, которое покрывает вершину $u$ в паросочетании $M$ (тут $u$ та же, что и в ребре $e = (u,v)$ правый конец которого удалили в начале). Рассмотрим паросочетание $M'$ в графе $G$ построенное так: $M' = (M\setminus e') \cup \lbrace e \rbrace$. Ясно, что это $X$-насыщенное паросочетание в исходном графе $G$, содержащее случайно выбранное ребро $e$. \xqed
\subsection*{12.13}
\textit{\textbf{Условие:} Доказать, что в двудольном графе $G$ совершенное паросочетание существует тогда и только тогда, когда для произвольного подмножества $X$ множества $V(G)$ вершин графа $G$ справедливо неравенство $|X|\leq |N(X)|$}\\
\textit{Решение:} Будем рассматривать двудольный граф $G=(X,Y)$.\\
$ \left[  \Rightarrow \right] $ Пусть в $G$ есть совершенное паросочетание. Тогда очевидно, что в $G$ есть $X$-насыщенное и $Y$-насыщенное паросочетание, а значит, что $\forall~S_x \subset X\ (|S_x| \leq |N(S_x)|)$ и $\forall~S_y \subset Y\ (|S_y| \leq |N(S_y)|)$. Но тогда для $S = S_x \cup S_y$, т.к. $S_x \cap S_y = \emptyset$ и $N(S_x) \cap N(S_y) = \emptyset$ (т.к. $N(S_x)\subseteq Y,\ N(S_y) \subseteq X$ верно:
\[
	|S_x\cup S_y| = |S_x| + |S_y| \leq |N(S_x)| + |N(S_y)| = |N(S_x)\cup N(S_y)| = |N(S_x \cup S_y)|
\]
Т.к. $S_x$ и $S_y$ любые и $\forall S \subseteq V(G)\ (S = S_x \cup S_y,\ S_x \subseteq X, S_y \subseteq Y)$, то $\forall S \subseteq V(G)\ (|S| \leq |N(S)|)$. Необходимость доказана.\\
$ \left[  \Leftarrow \right] $ Т.к. неравенство $|S| \leq |N(S)|$ верно для любого подмножества $V(G)$, то оно верно
и для $S=X$, а значит в $G$ существует $X$-насыщенное паросочетание, аналогично в $G$ существует $Y$-насыщенное паросочетание. Так же, $|X|=|Y|$. Но тогда по задаче $12.10$ в $G$ существует паросочетание покрывающее как $X$ так и $Y$, т.е. совершенное паросочетание. Достаточно показана. \xqed 
\end{document}
