%!TEX program = xelatex

\documentclass[12pt, a4paper]{article}
\usepackage[utf8]{inputenc}
\usepackage[russian]{babel}
\usepackage{pscyr}

\usepackage{xifthen}
\usepackage{parskip}
\usepackage{hyperref}
\usepackage[top=0.7in, bottom=1in, left=0.6in, right=0.6in]{geometry}
\usepackage{setspace}

\usepackage{amsmath}
\usepackage{MnSymbol}
\usepackage{amsthm}
\usepackage{mathtools}

\usepackage{algorithm}
\usepackage[noend]{algpseudocode}



\linespread{1.2}
\setlength{\parskip}{0pt}

\renewcommand\familydefault{\sfdefault}


% Stuff related to homework specific documents
\newcounter{MyTaskCounter}
\newcounter{MyTaskSectionCounter}
\newcommand{\tasksection}[1]{
	\stepcounter{MyTaskSectionCounter}
	\setcounter{MyTaskCounter}{0}
	\ifthenelse{\equal{#1}{}}{}{
	{\hfill\\[0.2in] \Large \textbf{\theMyTaskSectionCounter \enspace #1} \hfill\\[0.1in]}}
}

\newcommand{\task}[1]{
	\stepcounter{MyTaskCounter}
	\hfill\\[0.1in]
	\ifthenelse{\equal{\theMyTaskSectionCounter}{0}}{
	   \textbf{\large Задача №\theMyTaskCounter}
	}{
	   \textbf{\large Задача №\theMyTaskSectionCounter.\theMyTaskCounter}
	}
	\ifthenelse{\equal{#1}{}}{}{{\normalsize (#1)}}
	\hfill\\[0.05in]
}

% Math and algorithms

\makeatletter
\renewcommand{\ALG@name}{Алгоритм}
\renewcommand{\listalgorithmname}{Список алгроитмов}

\newenvironment{procedure}[1]
  {\renewcommand*{\ALG@name}{Процедура}
  \algorithm\renewcommand{\thealgorithm}{\thechapter.\arabic{algorithm} #1}}
  {\endalgorithm}

\makeatother

\algrenewcommand\algorithmicrequire{\textbf{Вход:}}
\algrenewcommand\algorithmicensure{\textbf{Выход:}}
\algnewcommand\True{\textbf{true}\space}
\algnewcommand\False{\textbf{false}\space}
\algnewcommand\And{\textbf{and}\space}

\newcommand{\xfor}[3]{#1 \textbf{from} #2 \textbf{to} #3}
\newcommand{\xassign}[2]{\State #1 $\leftarrow$ #2}
\newcommand{\xstate}[1]{\State #1}
\newcommand{\xreturn}[1]{\xstate{\textbf{return} #1}}

\DeclarePairedDelimiter\ceil{\lceil}{\rceil}
\DeclarePairedDelimiter\floor{\lfloor}{\rfloor}

\newcommand{\bigO}[1]{\mathcal{O}\left(#1\right)}

\title{Домашнее задание №12 \\ Алгоритмы. 5 курс. Весенний семестр.}
\author{Горбунов Егор Алексеевич}

\begin{document}
\maketitle

\begin{task}[1]
Найти подстроку в тексте. При сравнении строк можно делать циклический сдвиг алфавита в одной из них. $\bigO{n + m}$, алфавит --- не константа.
\end{task}
\begin{solution}
У нас алфавит конечный, пусть всем буквам сопоставлено целое число в соответствие. Теперь заметим, что если строка $c_1c_2\ldots c_l$ равна с точностью до сдвига алфавита строке $c'_1c'_2\ldots c'_l$, то это значит, что равны следующие массивы (строки):
\[
	(c_2 - c_1), (c_3 - c_2), \ldots, (c_k - c_{k-1}) \text{ и } (c'_2 - c'_1), (c'_3 - c'_2), \ldots, (c'_k - c'_{k-1})
\]
Разности соседних элементов не меняются при сдвиге алфавита.
Такие массивы по сути тоже строки, поэтому мы можем применять к ним тот же аппарат, что применяем к строкам.

Итак. Данные на вход строку и текст мы переделаем в массивы разностей, после чего будем использовать, например, алгоритм Кнута-Морриса-Пратта, что даст нам асимптотику в $\bigO{n + m}$. \xqed 
\end{solution}


\begin{task}[2]
Для каждого префикса строки найти количество его префиксов равных его суффиксу. $\bigO{n}$
\end{task}
\begin{solution}
Решается префикс-функцией. Рассмотрим какой-то префикс строки:
\[
    \overbracket[1pt]{s_0, s_1, s_2}, \ldots, \overbracket[1pt]{s_{k-3}, s_{k-2}, s_{k-1}}
\]
Выше обозначены максимальный префикс префикса равный его суффиксу, который мы получаем посчитав префикс-функцию.
Заметим, что все остальные длины префиксов префикса, равные его суффиксу, можно легко перебрать:
$next = p[p[k-1]]$. Соответственно для $k$-го префикса ответ считается так: $i = k-1$, пока $p[i] > 0$ прибавить к ответу $1$ и $i := p[i]$, а иначе закончить. \xqed
\end{solution}


\begin{task}[3]
Преобразовать $Z$-функцию в префикс-функцию без промежуточного восстановления строки за $\bigO{n}$
\end{task}
\begin{solution}
$z[i]$ --- размер максимального префикса подстроки $s[i..n-1]$ равный префиксу строки $s[0..n-1]$ ($Z$-функция).\\
$p[i]$ --- размер максимального суффикса подстроки $s[0..i]$ равный собственному префиксу этой подстроки.

Пускай у нас $z[i] > 0$, что обозначает, что $s[0..z[i]-1] = s[i..(i + z[i] - 1)]$. Заметим тогда, что для подстроки $s[0..(i + j)]$, где $j \in [0, z[i])$ точно верно, что её суффикс длины $j + 1$ равен префиксу.

Будем идти по массиву префикс-функции слева направо, т.е. в порядке убывания длины рассматриваемого суффикса строки. Для каждого рассматриваемого $z[i]$ будем перебирать $j$ в порядке убывания от $z[i] - 1$ до $0$ и если 
$p[i + j]$ ещё не проставлено, то присваивать $p[i + j] = j + 1$, а иначе переходить к следующему $i$.

Видим, что каждое значение $p[k]$ присваивается максимум один раз. Значит это работает за линейное время.

Но почему это вообще работает? Пускай в какой-то момент мы присвоили $p[i + j] = j + 1$. Пускай теперь по ходу алгоритма мы наткнулись на $i'$ и $j'$, что $i' > i$ (подругому и никак) и $i' + j' = i + j$. Но тогда очевидно, т.к. $i' > i$, то $j' < j$, а значит и $j' + 1 < j + 1$. Т.е. один раз проставив $p[i + j]$ его уже проставлять не нужно. Теперь заметим так же, что если в какой-то момент алгоритм проставил $p[i+j]$, то $p[i], p[i + 1],...p[i+j-1]$ тоже будут (были) проставлены (либо уже, либо будут на текущей итерации) в силу того, что $i$ перебирается в порядке возрастания. Оно работает. \xqed
\end{solution}


\begin{task}[5]
Даны бор $A$ и строка $s$. Нужно вернуть вершину бора $v$, от которой строку $s$ можно отложить вниз. Размер алфавита $\bigO{1}$. Время $\bigO{|A|+|s|)}$.
\end{task}
\begin{solution}
Давайте построим эйлеров обход бора по рёбрам. Будем хранить два таких эйлерова обхода $X$ и $Y$, но в $X$ будем держать буквы: если в обходе на позиции $i$ стоит ребро $(u, v)$, причём $u$ --- это родитель $v$ в боре, то в $X$ на позиции $i$ будет буква, соответствующая ребру $(u, v)$, а если же $(u, v)$ --- это возвратное ребро, то поставим на этой позиции в $X$ символ $\$$, а в $Y$ будем держать ссылки на вершины бора, соответствующие первой вершине ребра (т.е. если в обходе на позиции $i$ стоит ребро $(u, v)$, то в $Y$ на позиции $i$ будет ссылка на $u$). Идея в том, чтобы теперь искать $s$ в строке $X$ используя КМП, после чего, если мы нашли в $X$ такую позицию $u$ начиная с которой в ней лежит $s$, то ответом на запрос будет $Y[i]$. \xqed
\end{solution}

% \begin{task}[6]
% В словаре могут добавляться и удаляться слова. Необходимо в $online$ научиться отвечать на запрос $get(t)$ вида <<входит ли в текст $t$ хоть одно словарное слово>>. Амортизированное время работы $add(s)$ и $del(s)$: $\bigO{|s|\log{L}}$, время работы $get(t): \bigO{|t|\log{L}}$ ($L$ --- суммарная длина всего). Подсказка: заведите порядка $\log{L}$ боров.
% \end{task}


% \begin{task}[7]
% Дан набор слов. Придумать самую короткую строку, допускающую более одного разбиения на словарные слова или сообщить, что такой нет. Время работы --- полином от суммы длин слов и размера алфавита.
% \end{task}

\end{document}