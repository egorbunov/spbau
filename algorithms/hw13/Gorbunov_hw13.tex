%!TEX program = xelatex

\documentclass[12pt, a4paper]{article}
\usepackage[utf8]{inputenc}
\usepackage[russian]{babel}
\usepackage{pscyr}

\usepackage{xifthen}
\usepackage{parskip}
\usepackage{hyperref}
\usepackage[top=0.7in, bottom=1in, left=0.6in, right=0.6in]{geometry}
\usepackage{setspace}

\usepackage{amsmath}
\usepackage{MnSymbol}
\usepackage{amsthm}
\usepackage{mathtools}

\usepackage{algorithm}
\usepackage[noend]{algpseudocode}



\linespread{1.2}
\setlength{\parskip}{0pt}

\renewcommand\familydefault{\sfdefault}


% Stuff related to homework specific documents
\newcounter{MyTaskCounter}
\newcounter{MyTaskSectionCounter}
\newcommand{\tasksection}[1]{
	\stepcounter{MyTaskSectionCounter}
	\setcounter{MyTaskCounter}{0}
	\ifthenelse{\equal{#1}{}}{}{
	{\hfill\\[0.2in] \Large \textbf{\theMyTaskSectionCounter \enspace #1} \hfill\\[0.1in]}}
}

\newcommand{\task}[1]{
	\stepcounter{MyTaskCounter}
	\hfill\\[0.1in]
	\ifthenelse{\equal{\theMyTaskSectionCounter}{0}}{
	   \textbf{\large Задача №\theMyTaskCounter}
	}{
	   \textbf{\large Задача №\theMyTaskSectionCounter.\theMyTaskCounter}
	}
	\ifthenelse{\equal{#1}{}}{}{{\normalsize (#1)}}
	\hfill\\[0.05in]
}

% Math and algorithms

\makeatletter
\renewcommand{\ALG@name}{Алгоритм}
\renewcommand{\listalgorithmname}{Список алгроитмов}

\newenvironment{procedure}[1]
  {\renewcommand*{\ALG@name}{Процедура}
  \algorithm\renewcommand{\thealgorithm}{\thechapter.\arabic{algorithm} #1}}
  {\endalgorithm}

\makeatother

\algrenewcommand\algorithmicrequire{\textbf{Вход:}}
\algrenewcommand\algorithmicensure{\textbf{Выход:}}
\algnewcommand\True{\textbf{true}\space}
\algnewcommand\False{\textbf{false}\space}
\algnewcommand\And{\textbf{and}\space}

\newcommand{\xfor}[3]{#1 \textbf{from} #2 \textbf{to} #3}
\newcommand{\xassign}[2]{\State #1 $\leftarrow$ #2}
\newcommand{\xstate}[1]{\State #1}
\newcommand{\xreturn}[1]{\xstate{\textbf{return} #1}}

\DeclarePairedDelimiter\ceil{\lceil}{\rceil}
\DeclarePairedDelimiter\floor{\lfloor}{\rfloor}

\newcommand{\bigO}[1]{\mathcal{O}\left(#1\right)}

\title{Домашнее задание №13 \\ Алгоритмы. 5 курс. Весенний семестр.}
\author{Горбунов Егор Алексеевич}

\begin{document}
\maketitle

\begin{task}[1]
Найти среднее $lcp$ попарно всех суффиксов за $\bigO{n}$
\end{task}
\begin{solution}
\end{solution}

\begin{task}[3]
Найдите максимальный рефрен --- такую подстроку строки $s$, что количество
ее вхождений (возможно пересекающихся) помноженных на ее длину --- максимально. Решите
двумя способами: суффиксными деревом и массивом.
\end{task}
\begin{solution}
\begin{enumerate}
	\item Строим по строке $s$ суффиксное дерево $T$. Ответом на вопрос задачи будет какая-то позиция в этом дереве,
	т.к. любая позиция в $T$ отвечает некоторой подстроке. Заметим, что ответ всегда выгодно искать именно в вершинах, а не где-то по среди ребра, т.к. вдоль ребра число вхождений подстроки не меняется, а значит, можно пройти вних по ребру до ближайшего ветвления, что только увеличит длину подстроки (нас интересует макс. длина). Тогда предподсчитаем в вершинах $T$ длины подстрок $height(v)$, которые в них заканчиваются (глубины вершин в смысле числа символов), а так же размер поддерева в каждой вершине $size(v)$, т.е. число вершин, которые находятся ниже её самой и для которых она предок. Такие штуки считаются обходом дерева, а значит за линейное время от числа вершин дерева, а значит за $\bigO{|s|}$. Теперь осталось ещё раз обойти дерево, и среди вершин выбрать ту, у кторой $size(v) * height(v)$ максимально. \xqed
	\item Построим суффиксный массив $A$ по строке $s$, а так же массив $LCP$. Если какая-то подстрока встречается в $s$ $k$ раз, то есть $k$ суффиксов, которые идут подряд, будучи лексикографически отсортированы, а суффиксный массив как раз и задаёт такой порядок. Пойдём от обратного. Рассмотрим отрезок $[l, r)$: $\min_{i \in [l, r)}{LCP[i]}$ --- это длина подстроки, которая встречается $r - l$ раз. Таким образом нам нужно найти:
	\[
		(r - l) \cdot \min_{i \in [l, r)}{LCP[i]} \longrightarrow \max_{[l, r)}
	\]
	Заведём массив $B$. Будем перебирать $i$ в порядке убывания $LCP[i]$ и ставить $1$ в ячейку $B[i]$. Теперь найдём в массиве $B$ такую минимальную позицию $l \leq i$, что $\sum(B[l..i]) = i - l + 1$ (т.е. такое $l$, что всё от $l$ до $i$ в массиве $B$ заполнено $1$, т.е. такие $LCP$-шки были уже добавлены). И аналогично найдём такую границу $r \geq i$. И обновим ответ $ans = \max{(ans, (r - l + 1) \cdot LCP[i])}$. Сумму на отрезках $B$ можно считать деревом отрезков, а границы $l$ и $r$ подбирать бинарным поиском. Итого получится алгоритм за $\bigO{n\log^2(n)}$.
	Это быстрее чем $n^2$, а поэтому можем считать использование суфф. массива оправданным :) \xqed
\end{enumerate}
\end{solution}

\begin{task}[5]
Даны $k$ строк суммарной длины $n$. Найдите $p$-ю лексикографически общую их подстроку за $\bigO{n}$.
\end{task}
\begin{solution}
Будем решать суффиксным деревом. Рассмотрим строку $s_1 \$ s_2 \$ \ldots \$ s_k \$ \$ $. (два доллара на конце --- это важно). Её длина всё ещё $\bigO{n}$. Построим по такой строке суффиксное дерево $T$. Отбросим у этого суффиксного дерева ветку, которая идёт из корня по символу $\$$. Теперь нам нужно найти такую вершину дерева $T$, что в поддереве этой вершины есть $k$ листьев и при этом строка от корня до этой вершины $p$-ая в лексикографическом порядке. Для этого 1) нужно предподсчитать размеры поддеревьев и 2) обходить детей вершин в лексикографическом порядке при обходе в глубину. Всё. По-идее решение должно быть за $\bigO{n}$.
\end{solution}


\end{document}