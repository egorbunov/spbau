%!TEX program = xelatex

\documentclass[12pt, a4paper]{article}
\usepackage[utf8]{inputenc}
\usepackage[russian]{babel}
\usepackage{pscyr}

\usepackage{xifthen}
\usepackage{parskip}
\usepackage{hyperref}
\usepackage[top=0.7in, bottom=1in, left=0.6in, right=0.6in]{geometry}
\usepackage{setspace}

\usepackage{amsmath}
\usepackage{MnSymbol}
\usepackage{amsthm}
\usepackage{mathtools}

\usepackage{algorithm}
\usepackage[noend]{algpseudocode}



\linespread{1.2}
\setlength{\parskip}{0pt}

\renewcommand\familydefault{\sfdefault}


% Stuff related to homework specific documents
\newcounter{MyTaskCounter}
\newcounter{MyTaskSectionCounter}
\newcommand{\tasksection}[1]{
	\stepcounter{MyTaskSectionCounter}
	\setcounter{MyTaskCounter}{0}
	\ifthenelse{\equal{#1}{}}{}{
	{\hfill\\[0.2in] \Large \textbf{\theMyTaskSectionCounter \enspace #1} \hfill\\[0.1in]}}
}

\newcommand{\task}[1]{
	\stepcounter{MyTaskCounter}
	\hfill\\[0.1in]
	\ifthenelse{\equal{\theMyTaskSectionCounter}{0}}{
	   \textbf{\large Задача №\theMyTaskCounter}
	}{
	   \textbf{\large Задача №\theMyTaskSectionCounter.\theMyTaskCounter}
	}
	\ifthenelse{\equal{#1}{}}{}{{\normalsize (#1)}}
	\hfill\\[0.05in]
}

% Math and algorithms

\makeatletter
\renewcommand{\ALG@name}{Алгоритм}
\renewcommand{\listalgorithmname}{Список алгроитмов}

\newenvironment{procedure}[1]
  {\renewcommand*{\ALG@name}{Процедура}
  \algorithm\renewcommand{\thealgorithm}{\thechapter.\arabic{algorithm} #1}}
  {\endalgorithm}

\makeatother

\algrenewcommand\algorithmicrequire{\textbf{Вход:}}
\algrenewcommand\algorithmicensure{\textbf{Выход:}}
\algnewcommand\True{\textbf{true}\space}
\algnewcommand\False{\textbf{false}\space}
\algnewcommand\And{\textbf{and}\space}

\newcommand{\xfor}[3]{#1 \textbf{from} #2 \textbf{to} #3}
\newcommand{\xassign}[2]{\State #1 $\leftarrow$ #2}
\newcommand{\xstate}[1]{\State #1}
\newcommand{\xreturn}[1]{\xstate{\textbf{return} #1}}

\DeclarePairedDelimiter\ceil{\lceil}{\rceil}
\DeclarePairedDelimiter\floor{\lfloor}{\rfloor}

\newcommand{\bigO}[1]{\mathcal{O}\left(#1\right)}

\title{Домашнее задание №8 \\ Алгоритмы. 5 курс. Весенний семестр.}
\author{Горбунов Егор Алексеевич}

\begin{document}
\maketitle

\begin{task}[1]
Сведите к задаче линейного программирования задачу:
\[
	\min_{1 \leq i \leq p} \xbracket{\sum_{j = 1}^{n}{c_{ij}x_j}} \longrightarrow \max
\]
При ограничениях:
\[
	\begin{array}{c}
		\sum_{j = 1}^{n}{a_{ij}x_j} = b_i, i \in \xbracket{1\ldots m} \\
		x_i \geq 0, i \in \xbracket{1\ldots n}
	\end{array}
\]
\end{task}

\begin{solution}
Введём новую переменную $y$ и такие ограничения на неё:
\begin{equation}
\label{t1:1}
	y \leq \sum_{j = 1}^{n}{c_{ij}x_j}, i \in \xbracket{1\ldots p}
\end{equation}
Остальные ограничения оставим как есть и таким образом будем решать следующую задачу:
\[\begin{array}{l}
	y \longrightarrow \max \\
	y \leq \sum_{j = 1}^{n}{c_{ij}x_j}, i \in \xbracket{1\ldots p} \\
	\sum_{j = 1}^{n}{a_{ij}x_j} = b_i, i \in \xbracket{1\ldots m} \\
	x_i \geq 0, i \in \xbracket{1\ldots n} \\
\end{array}\]
Задачу приводим к стандартному виду (все неравенства на равенства и переменную $y$ превращаем в положительную) так, как проходили на лекции.

Так же заметим, что такая переформулировка верна в силу того, что хотя бы одно из неравенств~\ref{t1:1} превратится в равенство, иначе можно было бы увеличить $y$ не нарушив условия. \xqed
\end{solution}

\begin{task}[2]
Сведите к задаче линейного программирования задачу:
\[
	\sum_{i = 1}^{p}{ \left| \sum_{j = 1}^{n}{c_{ij}x_j} - d_i \right| } \longrightarrow \min
\]
При ограничениях:
\[
	\begin{array}{c}
		\sum_{j = 1}^{n}{a_{ij}x_j} = b_i, i \in \xbracket{1\ldots m} \\
		x_i \geq 0, i \in \xbracket{1\ldots n}
	\end{array}
\]
\end{task}
\begin{solution}
Введём переменные $y_1, y_2, \ldots, y_p$. И добавим такие ограничения:
\[
	\left| \sum_{j = 1}^{n}{c_{ij}x_j} - d_i \right| \leq y_i
	\Longleftrightarrow
	\begin{array}{c}
		\sum_{j = 1}^{n}{c_{ij}x_j} - d_i \leq y_i \\
		-\sum_{j = 1}^{n}{c_{ij}x_j} + d_i \leq y_i
	\end{array}
	, i \in \xbracket{1\ldots p}
\]
И тогда задача, которая будет решаться, такова:
\[\begin{array}{l}
	\sum_{i = 1}^{p}{y_i} \longrightarrow \min \\
	\sum_{j = 1}^{n}{c_{ij}x_j} - d_i \leq y_i, i \in \xbracket{1\ldots p}\\
	-\sum_{j = 1}^{n}{c_{ij}x_j} + d_i \leq y_i, i \in \xbracket{1\ldots p} \\
	x_i \geq 0, i \in \xbracket{1\ldots n}
\end{array}\]
Тут неравенства в равенства переводим как на лекции, опять же. Условия, наложенные на $y_i$ автоматически делают $y_i$ неотрицательными. \xqed
\end{solution}

\begin{task}[3]
Приведите пример несовместной задачи линейного программирования, двойственная к которой так же несовместна.
\end{task}
\begin{solution}
Рассмотрим задачу:
\[
	-x_1 - x_2 \longrightarrow \min
\]
С ограничениями:
\begin{align*}
	-x_1 &\geq 1 \\
	x_2 &\geq 1 \\
	x_1 &\geq 0\\
	x_2 &\geq 0\\
\end{align*}
Видно, что задача несовместна, т.к. $x_1 \geq 0$ и $x_1 \leq -1$ одновременно.\\
Двойственная к ней задача:
\[
	y_1 + y_2 \longrightarrow \max
\]
С ограничениями:
\begin{align*}
	-y_1 &\leq -1 \\
	y_2 &\leq -1 \\
	y_1 &\geq 0\\
	y_2 &\geq 0\\
\end{align*}
Система также несовместна, поскольку одновременно требуется $y_2 \geq 0$ и $y_2 \leq -1$.
\xqed
\end{solution}

\begin{task}[4]
Докажите, что если политоп линейной программы в канонической форме
($Ax = b, x \geq 0$) целый для любого вектора $b$, то матрица $A$ тотально унимодулярна.
\end{task}
\begin{solution}
\end{solution}

\end{document}
