\documentclass[12pt, a4paper]{article}
\usepackage[utf8]{inputenc}
\usepackage[russian]{babel}
\usepackage{pscyr}

\usepackage{xifthen}
\usepackage{parskip}
\usepackage{hyperref}
\usepackage[top=0.7in, bottom=1in, left=0.6in, right=0.6in]{geometry}
\usepackage{setspace}

\usepackage{amsmath}
\usepackage{MnSymbol}
\usepackage{amsthm}
\usepackage{mathtools}

\usepackage{algorithm}
\usepackage[noend]{algpseudocode}



\linespread{1.2}
\setlength{\parskip}{0pt}

\renewcommand\familydefault{\sfdefault}


% Stuff related to homework specific documents
\newcounter{MyTaskCounter}
\newcounter{MyTaskSectionCounter}
\newcommand{\tasksection}[1]{
	\stepcounter{MyTaskSectionCounter}
	\setcounter{MyTaskCounter}{0}
	\ifthenelse{\equal{#1}{}}{}{
	{\hfill\\[0.2in] \Large \textbf{\theMyTaskSectionCounter \enspace #1} \hfill\\[0.1in]}}
}

\newcommand{\task}[1]{
	\stepcounter{MyTaskCounter}
	\hfill\\[0.1in]
	\ifthenelse{\equal{\theMyTaskSectionCounter}{0}}{
	   \textbf{\large Задача №\theMyTaskCounter}
	}{
	   \textbf{\large Задача №\theMyTaskSectionCounter.\theMyTaskCounter}
	}
	\ifthenelse{\equal{#1}{}}{}{{\normalsize (#1)}}
	\hfill\\[0.05in]
}

% Math and algorithms

\makeatletter
\renewcommand{\ALG@name}{Алгоритм}
\renewcommand{\listalgorithmname}{Список алгроитмов}

\newenvironment{procedure}[1]
  {\renewcommand*{\ALG@name}{Процедура}
  \algorithm\renewcommand{\thealgorithm}{\thechapter.\arabic{algorithm} #1}}
  {\endalgorithm}

\makeatother

\algrenewcommand\algorithmicrequire{\textbf{Вход:}}
\algrenewcommand\algorithmicensure{\textbf{Выход:}}
\algnewcommand\True{\textbf{true}\space}
\algnewcommand\False{\textbf{false}\space}
\algnewcommand\And{\textbf{and}\space}

\newcommand{\xfor}[3]{#1 \textbf{from} #2 \textbf{to} #3}
\newcommand{\xassign}[2]{\State #1 $\leftarrow$ #2}
\newcommand{\xstate}[1]{\State #1}
\newcommand{\xreturn}[1]{\xstate{\textbf{return} #1}}

\DeclarePairedDelimiter\ceil{\lceil}{\rceil}
\DeclarePairedDelimiter\floor{\lfloor}{\rfloor}

\newcommand{\bigO}[1]{\mathcal{O}\left(#1\right)}

\title{Домашнее задание №4 \\ Алгоритмы. 5 курс. Весенний семестр.}
\author{Горбунов Егор Алексеевич}

\begin{document}
\maketitle

\section{Мои решения}

\begin{task}[1]
	\begin{enumerate}[label={(\alph*)}]
		\item Докажите, что любое 2-независимое семейство хэш-функций является универсальным.
		\item Докажите, что  любое $k+1$-независимое семейство хэш-функций является $k$-независимым.
	\end{enumerate}
\end{task}
\begin{solution}
	\begin{enumerate}[label={(\alph*)}]
		\item $Pr[h(x_1) = h(x_2)] = \sum_{y \in Y}{Pr[h(x_1) = y \wedge h(x_2) = y]} =^{\text{в силу 2-независимости}} =  \sum_{y \in Y}{\frac{1}{|Y|^2}} = |Y|\frac{1}{|Y|^2} = \frac{1}{|Y|}$
		Получили, что $Pr_{h \in \mathcal{H}}[h(x_1) = h(x_2)] \leqslant \frac{1}{|Y|}$, а значит 2-независимое семейство является универсальным. \xqed
		\item Заметим, что можно $h(x_{k+1})$ может принимать значения из $Y$ да и только, так что:
		\begin{align*}
			Pr\big[\bigwedge_{i=1}^{k}{h(x_i) = y_i}\big] &= \sum_{y \in Y}{Pr\big[\bigwedge_{i=1}^{k}{h(x_i) = y_i} \wedge h(x_{k+1}) = y\big]} =^{\text{по $k+1$-независимости}} = \sum_{|Y|}{\frac{1}{|Y|^{k+1}}} =\\
			&= |Y|\frac{1}{|Y|^{k+1}} = \frac{1}{|Y|^k}
		\end{align*}
		Получили, что хотели. \xqed
	\end{enumerate}
\end{solution}


\begin{task}[2]
Есть пара 2-независимых семейств хеш-функций $\mathcal{A} = \{ f:A\rightarrow \mathbb{F}_2^n \}$ и 
$\mathcal{B} = \{ g:B\rightarrow \mathbb{F}_2^n \}$. Постройте универсальное семейство хэш функций, которое:
\begin{enumerate}[label=(\alph*)]
	\item будет отправлять пары типа $(A, B)$ в $\mathbb{F}^n_2$
	\item будет отправлять мультимножества из элементов типа $A$ в $\mathbb{F}^n_2$
	\item будет отправлять множества из элементов типа $A$ в $\mathbb{F}^n_2$
	\item будет отправлять списки из элементов типа $A$ в $\mathbb{F}^n_2$
\end{enumerate}
\end{task}
\begin{solution}
\begin{enumerate}[label=(\alph*)]
	\item Рассмотрим такое семейство функций $\mathcal{C} = \{ h((a,b)) = f(a) + g(b) \}$ (сложение в данном поле --- это XOR). Вспомним свойства XOR, что если $c = a + d$, то $d = a + c$. Тогда:
	\begin{equation*}
	\begin{split}
		Pr[h((a_1, b_1)) &= h((a_2, b_2))] = Pr[f(a_1) + g(b_1) = f(a_2) + g(b_2)] =\\
						 &= \sum_{v \in \mathbb{F}^n_2}{P[f(a_1)+g(b_1) = v \wedge f(a_2) + g(b_2) = v]} = \\
						 &= \sum_{v \in \mathbb{F}^n_2}{\sum_{u \in \mathbb{F}^n_2}{P[f(a_1) = u \wedge g(b_1) = u+v \wedge f(a_2) + g(b_2) = v]}} =\\
						 &= \sum_{v \in \mathbb{F}^n_2}{\sum_{u \in \mathbb{F}^n_2}\sum_{e \in \mathbb{F}^n_2}{P[f(a_1) = u \wedge g(b_1) = u+v \wedge f(a_2) = e \wedge g(b_2) = e + v]}} = \\
						 &=^{\text{в силу независимости $f$ и $g$}} =\\
						 &=\sum_{v,u,e \in \mathbb{F}^n_2}{P[f(a_1) = u \wedge f(a_2) = e]P[g(b_1) = u+v \wedge g(b_2) = e + v]} =\\
						 &=^{\text{в силу 2-независимости $\mathcal{A}$ и $\mathcal{B}$}} =\\
						 &=(2^n)^3\cdot \frac{1}{2^{2n}2^{2n}} =\\
						 &=\frac{1}{2^n} = \frac{1}{| \mathbb{F}^n_2|}
	\end{split}
	\end{equation*}
	\xqed

	\item Заметим, что семейство функций должно быть бесконечным семейство хэш-функций. Действительно, если семейство конечно, то по принципу Дирихле в силу бесконечности числа мультимножеств получится, что хотя бы для 2-х элементов совпадают хэши (хэшом будет конкатенация применения всех функций из семейства к объекту), а значит условие универсальности для этих объектов не выполняется. 

	Рассматриваем мультимножество размера $k$ и выбираем независимо (и закрепляем) $f_1, f_2, \ldots, f_k \in \mathcal{A}$ и построим семейство с хэш-функциями $h(\{ a_1, a_2, \ldots, a_k \}) = \sum_{i=1}^k{f_i(a_i)}$, теперь нужно оценить вероятность:
	\begin{equation*}
	\begin{split}
		\Pr\left[ \sum{f_i(a_i)} = \sum{f_i(b_i)} \right] &= \sum_{y}{\Pr \left[ \sum{f_i(a_i)} = \sum{f_i(b_i)} = y \right] }
	\end{split}
	\end{equation*}
	Теперь там далее расписывается как в предыдущем пункте...
	Нужно ещё аккуратно разобрать случай, когда элементы совпадают, т.к. выше мы ей пользовались для 2-независимости, но эта проблема решается тем, что мы разбиваем множества...?
\end{enumerate}
\end{solution}

\begin{task}[3]
Используя задачу с практики построить семейство 2-независимых хэш-функций $\mathcal{H} = \{ h_i: \mathbb{F}_2^n \rightarrow \mathbb{F}^k_2\}_i$. Докажите, что семейство является 2-независимым.
\end{task}
\begin{solution}
Семейство выглядит так: $\mathcal{H} = \{ h(x) = Ax + b \}$. Параметрами семейства выступают матрица $A$ ранга $k$ и размера $k \times n$  над $\mathbb{F}_2$: 
\[
	\begin{pmatrix}
		A_1 \\
		A_2 \\
		\vdots\\
		A_k
	\end{pmatrix}
\]
И вектор $b$ над $\mathbb{F}_2$ длины $k$. Теперь будем показывать, что это семейство является 2-независимым, т.е. покажем, что $\forall x_1, x_2 \in \mathbb{F}_2^n,\ x_1 \neq x_2$ и $\forall y_1, y_2 \in \mathbb{F}_2^k$, что:
\[
	\Pr_{h \in \mathcal{H}}\big[ h(x_1) = y_1 \wedge h(x_2) = y_2 \big] = \frac{1}{|\mathbb{F}_2^n|^2} = \frac{1}{2^{2n}}
\]
Нам нужно найти вероятность такой матрицы $A$, что $y_1 = Ax_1 + b$ и $y_2 = Ax_2 + b$, т.е. $A(x_1 - x_2) = y_1 - y_2$ и вероятность вектора $b$. Пусть $y_1 - y_2 = (l_1, l_2, \ldots, l_k)^T$.
\begin{equation*}
\begin{split}
\Pr\big[ Az = y_1 - y_2 \big] &= \Pr\big[ A_1z = l_1 \big] \Pr\big[ A_2z = l_2 | A_1 \big] \cdot \ldots  \cdot \Pr\big[ A_kz = l_k | A_1, A_2, \ldots, A_{k-1}] \\
&=\text{в силу того, что у нас все элементы $A$ независимы}\\
&= \Pr\big[ A_1z = l_1 \big] \Pr\big[ A_kz = l_k \big] \cdot \ldots \cdot \Pr\big[ A_kz = l_k \big]
\end{split}
\end{equation*}
Заметим, что в каждой вероятности $A_iz$ равно либо $1$ либо $0$. Вероятности $0$ и $1$ в результате суммы и разности по модулю равны $\frac{1}{2}$, таким образом будет $\Pr[A_iz = 1 \wedge l_i = 1] + \Pr[A_iz = 0 \wedge l_i = 0] = \frac{1}{4}+\frac{1}{4} = \frac{1}{2}$ (тут в силу независимости раскрывается вероятность произведения в произведение вероятностей). Таким образом получили: 
\[
	\Pr\big[ Az = y_1 - y_2 \big] = \prod_{i = 1}^n{\frac{1}{2}} = \frac{1}{2^n}
\]
Вероятность же получить $b$ такой, что $b = Ax_1 - y_1$ у нас всегда $\frac{1}{2^n}$, т.к. все элементы вектора независимы, а таким образом $\Pr_{h \in \mathcal{H}}\big[ h(x_1) = y_1 \wedge h(x_2) = y_2 \big]$ получается произведением полученных вероятностей для $b$ и $A$:
\[
 \Pr_{h \in \mathcal{H}}\big[ h(x_1) = y_1 \wedge h(x_2) = y_2 \big] = \frac{1}{2^n}\frac{1}{2^n} = \frac{1}{2^{2n}}
\]
\xqed
\end{solution}

\begin{task}[4]
Есть хэш-таблица размера $n$ с хэш-функцией $h: K \rightarrow [n]$. $h$ равновероятно отправляет ключи в корзины. На вход поступает $n$ ключей. Коллизии разрешаются цепочками. Показать:
\begin{enumerate}[label=(\alph*)]
	\item Зафиксируем $x \in [n]$. Доказать:
	\[ Q_k = \Pr\big[ \text{$k$ ключей имеют хэш $x$} \big] = {n \choose k} \cdot \left(\frac{1}{n}\right)^k \cdot \left(1 - \frac{1}{n}\right)^{n-k}\] 
	\item Пусть $P_k$ --- вероятность максимальной цепочки иметь длину $k$. Доказать, что $P_k \leqslant nQ_k$
	\item Показать, что $Q_k < (\frac{e}{k})^k$
	\item Показать, что для некоторого $c > 1$ верно $Q_k \leqslant \frac{1}{n^3}$ при $l \geqslant c \frac{\log{n}}{\log{\log{n}}}$
	\item Доказать, что матожидание длины максимальной цепочки не превосходит
	\[ \Pr\left[ M > c \frac{\log{n}}{\log{\log{n}}} \right]n + \Pr\left[ M \leqslant c \frac{\log{n}}{\log{\log{n}}} \right]c \frac{\log{n}}{\log{\log{n}}} \]
	где $M$ --- максимальная длина цепочки. $M$ --- случайная величина. Вывести оценку сверху на это матожидание: $\bigO{\frac{\log{n}}{\log{\log{n}}}}$
\end{enumerate}
\end{task}
\begin{solution}
\begin{enumerate}[label=(\alph*)]
\item У нас есть $n$ элементов. Если мы зафиксируем какие-то $k$ элементов, то ясно, что вероятность, что именно они будут иметь хэш $x$ равна $\left(\frac{1}{n}\right)^k\left(1-\frac{1}{n}\right)^{n-k}$, т.к. вероятность, что у элемента хэш равен $x$ равна $\frac{1}{n}$, а вероятность того, что не равен, соответственна $1 - \frac{1}{n}$. А $k$ элементов мы можем зафиксировать ${n \choose k}$ способами. Все такие выборы независимы, а значит нужно просуммировать ${n \choose k}$ раз $\left(\frac{1}{n}\right)^k\left(1-\frac{1}{n}\right)^{n-k}$ чтобы получить $Q_k$. \xqed

\item $P_k = \sum_{x\in [n]}{P[\text{максимальная цепь имеет длину $k$ и все элементы из неё имеют хэш $x$}]}$. Ясно, что вероятность под суммой всяко меньше или равна $Q_k$ (т.к. в $Q_k$ не требуется максимальность), а значит: $P_k \leqslant \sum_{x \in [n]}{Q_k} = nQ_k$ \xqed

\item 
\begin{equation*}
\begin{split}
	Q_k &= \frac{n!(n-1)^{n-k}}{k!(n-k)!n^kn^{n-k}} < \frac{n!(n)^{n-k}}{k!(n-k)!n^kn^{n-k}} = \frac{n!}{k!(n-k)!n^k} < \frac{n^k}{k!(n-k)!n^k} =\\
	    &= \frac{1}{k!(n-k)!} < \frac{1}{k!} \approx^{\text{Стирлинг}} \left(\frac{e}{k}\right)^k\frac{1}{\sqrt{\pi k}} < \left(\frac{e}{k}\right)^k 
\end{split}
\end{equation*}
\xqed

\item Воспользуемся предыдущим пунктом. 
\begin{equation*}
\begin{split}
	Q_k &<  \left(\frac{e}{k}\right)^k = \left(\frac{e}{c \frac{\log{n}}{\log{\log{n}}}}\right)^{c \frac{\log{n}}{\log{\log{n}}}} = \left(e\frac{\log{\log{n}}}{c\log{n}}\right)^{c \frac{\log{n}}{\log{\log{n}}}} =\\
	    &= \exp\left( \frac{c\log{n}}{\log{\log{n}}} + \frac{c\log{n}}{\log{\log{n}}} \left( \log{\log{\log{n}}} - \log{c} - \log{\log{n}} \right)\right) = \\
	    &= \exp\left( -c\log{n}\left(1 - \frac{\log{\log\log{n}} - \log{c} + 1}{\log{\log{n}}}\right)\right)
\end{split}
\end{equation*}
Ясно, что найдётся такое $c$, т.к. то, что внутри скобки в экспоненте стремиться к $1$ при стремлении $n$ к бесконечности, такое, что: $Q_k < e^{-3\log{n}} = \frac{1}{n^3}$ \xqed

\item Распишем матожидание длины максимальной цепочки:
\begin{equation*}
\begin{split}
	E &= \sum_{k=1}^n{kP_k} = \left( P_1 + P_2 2 + \ldots + P_{c \frac{\log{n}}{\log{\log{n}}}}{c \frac{\log{n}}{\log{\log{n}}}} \right) + \left( P_{c \frac{\log{n}}{\log{\log{n}}} + 1}\left(c \frac{\log{n}}{\log{\log{n}}}+1\right)  + \ldots P_nn\right) \leqslant \\
	  &\leqslant  c\frac{\log{n}}{\log{\log{n}}} \left(P_1 + \ldots P_{c \frac{\log{n}}{\log{\log{n}}}} \right) + n\left( P_{c \frac{\log{n}}{\log{\log{n}}} + 1} + \ldots P_n\right) =\\
	  &= \text{а это и есть то, что нужно, т.к. длина макс. цепи перебирается} \\
	  &= Pr\left[ M > c \frac{\log{n}}{\log{\log{n}}} \right]n + \Pr\left[ M \leqslant c \frac{\log{n}}{\log{\log{n}}} \right]c \frac{\log{n}}{\log{\log{n}}} 
	  \end{split}
\end{equation*}
Для доказательство оценки сверху нужно просто воспользоваться пунктом c, d и (первый переход был получен выше):
\begin{equation*}
\begin{split}
 	E &= \left( P_1 + P_2 2 + \ldots + P_{c \frac{\log{n}}{\log{\log{n}}}}{c \frac{\log{n}}{\log{\log{n}}}} \right) + \left( P_{c \frac{\log{n}}{\log{\log{n}}} + 1}\left(c \frac{\log{n}}{\log{\log{n}}}+1\right)  + \ldots P_nn\right) \leqslant \\
 	  &\leqslant \left( P_1 + P_2 2 + \ldots + P_{c \frac{\log{n}}{\log{\log{n}}}}{c \frac{\log{n}}{\log{\log{n}}}} \right) + n\left( Q_{c \frac{\log{n}}{\log{\log{n}}} + 1}\left(c \frac{\log{n}}{\log{\log{n}}}+1\right)  + \ldots Q_nn\right) \leqslant \\
 	  &\leqslant  \left( P_1 + P_2 2 + \ldots + P_{c \frac{\log{n}}{\log{\log{n}}}}{c \frac{\log{n}}{\log{\log{n}}}} \right) + n\left( \frac{1}{n^3}\left(c \frac{\log{n}}{\log{\log{n}}}+1\right)  + \ldots \frac{1}{n^3}n\right) = \\
 	  &= \bigO{\Pr\left[ M \leqslant c \frac{\log{n}}{\log{\log{n}}} \right]c \frac{\log{n}}{\log{\log{n}}}} =^{\text{Вероятность $\leqslant 1$}} = \bigO{\frac{\log{n}}{\log{\log{n}}}}
\end{split}
\end{equation*}
\xqed
\end{enumerate}
\end{solution}

\begin{task}[7]
Про восстановление $k$ по входному числе $2^k$ за $\bigO{1}$.
\end{task}
\begin{solution}
Нужно найти такое число, которое даёт различные остатки от деления на все $2^k$, где $k$ пробегает от $0$ до $63$. Это число будет небольшим (вроде как подойдёт $67$). В силу того, что числа маленькие и $k < 64$ всё поместится в 100 байт. \xqed
\end{solution}

\begin{task}[8]
Про изоморфизм деревьев.
\end{task}
\begin{solution}
\begin{enumerate}[label=(\alph*)]
\item Если порядок важен, то можно делать так: для дерева $T$ найти все вершины, высота поддеревьев которых равна высоте $t$. Далее для каждой такой вершины построить строчку так: разметить все вершины в порядке обхода в глубину, а потом сделать эйлеров обход и получить список чисел. Так же сделать для дерева $t$. Теперь просто хэшами сравниваем эти списки.
\item Если порядок не важен, то так же отбираем по нужной высоте вершины, как и в прошлой задаче, а потом в каждом поддереве каждой вершины начинаем рассматривать вершины с самых нижних по уровням. Только теперь нужно сравнивать каждый уровень с соответ. уровнем $t$ хэшами не на списках, а на множествах :)
\end{enumerate}
\end{solution}

\begin{task}[9]
Про изоморфизм деревьев без хэшей
\end{task}
\begin{solution}
См. предыдущую, только сравниваем без хэшей, а за линию.
\end{solution}
\end{document}
