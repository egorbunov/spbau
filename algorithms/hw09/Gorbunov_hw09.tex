%!TEX program = xelatex

\documentclass[12pt, a4paper]{article}
\usepackage[utf8]{inputenc}
\usepackage[russian]{babel}
\usepackage{pscyr}

\usepackage{xifthen}
\usepackage{parskip}
\usepackage{hyperref}
\usepackage[top=0.7in, bottom=1in, left=0.6in, right=0.6in]{geometry}
\usepackage{setspace}

\usepackage{amsmath}
\usepackage{MnSymbol}
\usepackage{amsthm}
\usepackage{mathtools}

\usepackage{algorithm}
\usepackage[noend]{algpseudocode}



\linespread{1.2}
\setlength{\parskip}{0pt}

\renewcommand\familydefault{\sfdefault}


% Stuff related to homework specific documents
\newcounter{MyTaskCounter}
\newcounter{MyTaskSectionCounter}
\newcommand{\tasksection}[1]{
	\stepcounter{MyTaskSectionCounter}
	\setcounter{MyTaskCounter}{0}
	\ifthenelse{\equal{#1}{}}{}{
	{\hfill\\[0.2in] \Large \textbf{\theMyTaskSectionCounter \enspace #1} \hfill\\[0.1in]}}
}

\newcommand{\task}[1]{
	\stepcounter{MyTaskCounter}
	\hfill\\[0.1in]
	\ifthenelse{\equal{\theMyTaskSectionCounter}{0}}{
	   \textbf{\large Задача №\theMyTaskCounter}
	}{
	   \textbf{\large Задача №\theMyTaskSectionCounter.\theMyTaskCounter}
	}
	\ifthenelse{\equal{#1}{}}{}{{\normalsize (#1)}}
	\hfill\\[0.05in]
}

% Math and algorithms

\makeatletter
\renewcommand{\ALG@name}{Алгоритм}
\renewcommand{\listalgorithmname}{Список алгроитмов}

\newenvironment{procedure}[1]
  {\renewcommand*{\ALG@name}{Процедура}
  \algorithm\renewcommand{\thealgorithm}{\thechapter.\arabic{algorithm} #1}}
  {\endalgorithm}

\makeatother

\algrenewcommand\algorithmicrequire{\textbf{Вход:}}
\algrenewcommand\algorithmicensure{\textbf{Выход:}}
\algnewcommand\True{\textbf{true}\space}
\algnewcommand\False{\textbf{false}\space}
\algnewcommand\And{\textbf{and}\space}

\newcommand{\xfor}[3]{#1 \textbf{from} #2 \textbf{to} #3}
\newcommand{\xassign}[2]{\State #1 $\leftarrow$ #2}
\newcommand{\xstate}[1]{\State #1}
\newcommand{\xreturn}[1]{\xstate{\textbf{return} #1}}

\DeclarePairedDelimiter\ceil{\lceil}{\rceil}
\DeclarePairedDelimiter\floor{\lfloor}{\rfloor}

\newcommand{\bigO}[1]{\mathcal{O}\left(#1\right)}

\title{Домашнее задание №8 \\ Алгоритмы. 5 курс. Весенний семестр.}
\author{Горбунов Егор Алексеевич}

\begin{document}
\maketitle
\begin{task}[1]
Дан граф и выделенные вершины $s$ и $t$. Нужно проверить, правда ли существует единственный минимальный $s-t$ разрез.
\begin{enumerate}
	\item $\bigO{poly(V, E)}$
	\item $\bigO{E}$ при условии, что нам уже известен максимальный поток (с доказательством).
\end{enumerate}
\end{task}
\begin{solution}
\begin{enumerate}
	\item Найдём какой-нибудь минимальный $s-t$ резрез $(S, T)$. Допустим, что он не единственный и существует ещё один минимальный разрез $(S', T')$. Тогда точно есть такое ребро $e \notin (S', T')$, что $e \in (S, T)$, а это значит, что если в исходном графе вес ребра $e$ увеличить, то вес минимального разреза не увеличится. Будем считать, не умаляя общности, что у каждого ребра $e$ исходного графа есть вес $w(e)$. Отсюда получаем алгоритм:
	\begin{algorithmic}
	\xassign{$C$}{$MinCut(G)$}
	\xassign{$minW$}{$w(C)$}
	\For{$e \in C$}
		\xassign{$w(e)$}{$w(e) + 1$}
		\If{$minW = w(MinCut(G))$}
			\xreturn{минимальный разрез не единственный}
		\EndIf
		\xassign{$w(e)$}{$w(e) - 1$}
	\EndFor
	\xreturn{минимальный разрез единственный}
	\end{algorithmic}
	Минимальный разрез мы умеем находить за $\bigO{poly(V, E)}$  (например, найдя максимальный поток Эдмонсом-Карпом),
	а значит и весь алгоритм работает за $\bigO{poly(V, E)}$ \xqed

	\item Рассматриваем исходный граф как $s-t$ сеть (пропускные способности рёбер --- это их веса). Пускай нам известен максимальный поток $f$ в этой сети, а значит нам известен и какой-то минимальный разрез $(S, T)$, $s \in S, t \in T$, полученный обходом остаточной сети $G^f$. Пускай в этой остаточной сети есть подмножество $T_x \subset T, T_x \neq \emptyset$ такое, что в $G^f$ ни одно ребро не выходит из $T_x$ и $t \notin T_x$. Рассмотрим тогда разрез $(S \cup T_x, T \setminus T_x)$ и его пропускную способность:
	\[
		\sum_{e \in (S \cup T_x, T \setminus T_x)}{c_e} = \sum_{e \in (S, T)}{c_e} - \sum_{e \in (S, T_x)}{c_e} + \sum_{e \in (T_x, T)}{c_e}
	\]
	Поскольку $(S, T)$ --- минимальный разреза, т.е. в остаточной сети его рёбра отсутствуют, а значит поток $f$ по ним равен их пропускным способностям. Аналогично, в силу выбора $T_x$, поток по рёбрам из $T_x$ равен их пропускным способностям:
	\[
		\sum_{e \in (S \cup T_x, T \setminus T_x)}{c_e} = \sum_{e \in (S, T)}{c_e} - \sum_{e \in (S, T_x)}{f_e} + \sum_{e \in (T_x, T)}{f_e}
	\]
	Теперь замети, что единственный поток, который входит в $T_x$, идёт из $S$. Если бы это было не так, и было бы какое-то ребро $(v, u)$$ \in G$, $u \in T_x, v \in T \setminus T_x$, что $f_{(v,u)} > 0$, то в по определению, в остаточной сети $G^f$ было бы ребро $(u, v)$ с пропускной способностью $f_{(v, u)}$, но это ребро тянется из $T_x$, а по построению $T_x$ это невозможно. Тогда по закону сохранения потока:
	\[
		\sum_{e \in (S \cup T_x, T \setminus T_x)}{c_e} = \sum_{e \in (S, T)}{c_e} - \sum_{e \in (S, T_x)}{f_e} + \sum_{e \in (S, T_x)}{f_e} = \sum_{e \in (S, T)}{c_e}
	\]
	Таким образом, если такое множество $T_x$ можно выбрать, то мы найдём ещё один минимальный разрез, отличный от $(S, T)$.

	Выберем $T_x$ таким образом: $T_x = T \setminus X$, гдe:
	\[
		X = \xbrace{\text{вершины, достижимые по обратным рёбрам в } G^f \text{ из } t}
	\]
	Такое $X$ находится за $\bigO{E + V}$ поиском в глубину. Это не совсем $\bigO{E}$, конечно... \xqed
\end{enumerate}
\end{solution}

\begin{task}[2]
В неориентированном графе без кратных рёбер необходимо удалить минимальное число рёбер так, чтобы увеличилось количество компонент связности за $\bigO{V\cdot Flow}$. Оцените время работы того же алгоритма более точно как $\bigO{E^2}$.
\end{task}
\begin{solution}
Каждому ребру исходного графа сообщим вес $1$. Теперь задача сводится к тому, чтобы найти в этом графе минимальный разрез $(S, T)$, вес которого равен минимальному числу рёбер, удаление которых приводит к увеличению компонент связности. Зафиксируем вершину $v \in V(G)$, в минимальном разрезе она попадёт либо в $S$, либо в $T$, а значит, т.к. граф неориентированный, есть такая вершина $u$, что $v-u$ разрез минимален. Т.е. мы можем перебирать $u \in V$ и искать максимальный поток (а значит и минимальный разрез) в сети, с истоком $v$ и стоком $u$. В качестве ответа выдавать минимум из этих разрезов. Таким образом асимптотика алгоритма $\bigO{V\cdot Flow}$.

Будем искать максимальный поток алгоритмом Форда-Фалкерсона, сложность которого $\bigO{E\cdot f}$, где $f$ --- величина максимального потока. В нашем случае пропускные способности всех рёбер равны $1$, а значит максимальный поток ограничен $deg(sink)$, где $sink$ --- это перебираемый нами в алгоритме сток. Таким образом сложность всего алгоритма:
\[
	\bigO{\sum_{u \in V}{E\cdot deg(u)}} = \bigO{E\cdot2E} = \bigO{E^2}
\]
\xqed
\end{solution}

\begin{task}[3]
Есть ориентированный граф с начальной и конечной вершинами. В начальной вершине есть $K$ грузовиков. Грузовикам нужно попасть в конечную вершину. Время дискретно. За единицу времени каждый грузовик или стоит на месте, или перемещается в одну из соседних вершин. В любой вершине могут одновременно стоять несколько грузовиков. По любому из рёбер в каждый момент времени должен ехать не более чем один грузовик. Минимизируйте время, когда грузовики окажутся в конечной вершине.
\begin{enumerate}
	\item $\bigO{poly(V, E, K)}$
	\item $\bigO{K(V+K)E}$
\end{enumerate}
\end{task}
\begin{solution}
\end{solution}

\begin{task}[4]
Есть $n$ рабочих и $m$ работ. И есть матрица умения: <<какой рабочий какие работы умеет делать>>. Нужно максимально равномерно распределить работы между рабочими. То есть, каждой работе сопоставить рабочего, который умеет делать эту работу, а кроме того минимизировать $\max_{i=1..n}{k_i}$, где $k_i$ --- количество работ, выданных $i$-ому рабочему.
\end{task}
\begin{solution}
Построим сеть: $s$ --- исток, соединён с $m$ вершинами-работами ($w_i$) рёбрами с пропускной способностью $1$, каждая вершина работа $w_i$ соединена рёбрами пропускной способности $1$ c вершинами-рабочими $h_i$, а каждая вершина-рабочий $h_i$ соединена со стоком $t$ ребром с пропускной способностью $k$. Тогда в такой сети, если величина максимального потока равна $m$ (все работы выполнены), то каждый рабочий получил не более $k$ работ. Это $k$ нам нужно минимизировать. Подобно задаче с практики мы можем: 
	\begin{enumerate}
		\item Устроить бинарный поиск по $k$
		\item Последовательно искать максимальный поток в такой сети сначала с $k=1$, потом, не пересчитывая полученного потока, начинать искать его в сети с $k=2$ и так далее, пока не наткнёмся на $k$, при котором поток равен $m$
	\end{enumerate}
\xqed
\end{solution}
\end{document}