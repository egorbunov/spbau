\documentclass[12pt, a4paper]{article}

\usepackage[utf8]{inputenc}
\usepackage[russian]{babel}
\usepackage{pscyr}
\usepackage{amssymb}
\usepackage{xifthen}
\usepackage{parskip}
\usepackage{hyperref}
\usepackage{setspace}
\usepackage{graphicx}
\usepackage{xcolor}
\usepackage{amsmath}
\usepackage{MnSymbol}
\usepackage{amsthm}
\usepackage{mathtools}
\usepackage{algorithm}
\usepackage[noend]{algpseudocode}
\usepackage[shortlabels]{enumitem}
\usepackage{listings}
\usepackage{subfig}
\usepackage{tikz}
\usepackage{tkz-berge}
\usepackage[top=0.5in, bottom=0.7in, left=0.6in, right=0.6in]{geometry}
\usepackage{bm}


\usetikzlibrary{shapes,snakes}
\usetikzlibrary{arrows, petri, topaths}
\linespread{1.3}
\setlist[enumerate, 1]{1\textsuperscript{o}}

% Homework structure

\newenvironment{task}[1][]
{
	\par\medskip
	\noindent \textbf{Задание~№#1}
	\rmfamily
}
{
}

\newenvironment{solution}[1][]
{
	\par
	\noindent \textbf{\textit{Решение:}}
	\rmfamily
}
{
	\medskip
}

% Math and algorithms
\makeatletter
\renewcommand{\ALG@name}{Алгоритм}
\renewcommand{\listalgorithmname}{Список алгроитмов}
\newenvironment{procedure}[1]
  {\renewcommand*{\ALG@name}{Процедура}
  \algorithm\renewcommand{\thealgorithm}{\thechapter.\arabic{algorithm} #1}}
  {\endalgorithm}
\makeatother

\algrenewcommand\algorithmicrequire{\textbf{Вход:}}
\algrenewcommand\algorithmicensure{\textbf{Выход:}}
\algnewcommand\True{\textbf{true}\space}
\algnewcommand\False{\textbf{false}\space}
\algnewcommand\And{\textbf{and}\space}
\newcommand{\xfor}[3]{#1 \textbf{from} #2 \textbf{to} #3}
\newcommand{\xassign}[2]{\State #1 $\leftarrow$ #2}
\newcommand{\xstate}[1]{\State #1}
\newcommand{\xreturn}[1]{\xstate{\textbf{return} #1}}
\DeclarePairedDelimiter\ceil{\lceil}{\rceil}
\DeclarePairedDelimiter\floor{\lfloor}{\rfloor}
\newcommand{\bigO}[1]{\mathcal{O}\left(#1\right)}
\newcommand{\xqed}{\hfill $\blacksquare$}
\newcommand{\code}[1]{\colorbox{gray!15}{\footnotesize\texttt{#1}}}

% lst 

\lstset{frame=tb,
  language=Python,
  aboveskip=2mm,
  belowskip=2mm,
  showstringspaces=false,
  columns=fullflexible,
  basicstyle={\scriptsize\ttfamily},
  numbers=left,
  numberstyle=\tiny\color{gray},
  keywordstyle=\color{blue},
  commentstyle=\color{dkgreen},
  stringstyle=\color{mauve},
  breaklines=true,
  breakatwhitespace=true,
  tabsize=4,
  backgroundcolor=\color{gray!5},
  showspaces=false,
  showtabs=false,
  escapeinside={(*@}{@*)},
}

\newcommand{\xparen}[1]{\left( #1 \right)}
\newcommand{\xangel}[1]{\left\langle #1 \right\rangle}
\newcommand{\xbrack}[1]{\left[ #1 \right]}
\newcommand{\xbrace}[1]{\left\lbrace #1 \right\rbrace}
\newcommand{\xret}[0]{\rightarrow}
\newcommand{\xintrp}[1]{\left\llbracket #1 \right\rrbracket}
