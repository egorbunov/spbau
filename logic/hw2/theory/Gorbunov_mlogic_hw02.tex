\documentclass[12pt, a4paper]{article}
\usepackage[utf8]{inputenc}
\usepackage[russian]{babel}
\usepackage{pscyr}

\usepackage{xifthen}
\usepackage{parskip}
\usepackage{hyperref}
\usepackage[top=0.7in, bottom=1in, left=0.6in, right=0.6in]{geometry}
\usepackage{setspace}

\usepackage{amsmath}
\usepackage{MnSymbol}
\usepackage{amsthm}
\usepackage{mathtools}

\usepackage{algorithm}
\usepackage[noend]{algpseudocode}



\linespread{1.2}
\setlength{\parskip}{0pt}

\renewcommand\familydefault{\sfdefault}


% Stuff related to homework specific documents
\newcounter{MyTaskCounter}
\newcounter{MyTaskSectionCounter}
\newcommand{\tasksection}[1]{
	\stepcounter{MyTaskSectionCounter}
	\setcounter{MyTaskCounter}{0}
	\ifthenelse{\equal{#1}{}}{}{
	{\hfill\\[0.2in] \Large \textbf{\theMyTaskSectionCounter \enspace #1} \hfill\\[0.1in]}}
}

\newcommand{\task}[1]{
	\stepcounter{MyTaskCounter}
	\hfill\\[0.1in]
	\ifthenelse{\equal{\theMyTaskSectionCounter}{0}}{
	   \textbf{\large Задача №\theMyTaskCounter}
	}{
	   \textbf{\large Задача №\theMyTaskSectionCounter.\theMyTaskCounter}
	}
	\ifthenelse{\equal{#1}{}}{}{{\normalsize (#1)}}
	\hfill\\[0.05in]
}

% Math and algorithms

\makeatletter
\renewcommand{\ALG@name}{Алгоритм}
\renewcommand{\listalgorithmname}{Список алгроитмов}

\newenvironment{procedure}[1]
  {\renewcommand*{\ALG@name}{Процедура}
  \algorithm\renewcommand{\thealgorithm}{\thechapter.\arabic{algorithm} #1}}
  {\endalgorithm}

\makeatother

\algrenewcommand\algorithmicrequire{\textbf{Вход:}}
\algrenewcommand\algorithmicensure{\textbf{Выход:}}
\algnewcommand\True{\textbf{true}\space}
\algnewcommand\False{\textbf{false}\space}
\algnewcommand\And{\textbf{and}\space}

\newcommand{\xfor}[3]{#1 \textbf{from} #2 \textbf{to} #3}
\newcommand{\xassign}[2]{\State #1 $\leftarrow$ #2}
\newcommand{\xstate}[1]{\State #1}
\newcommand{\xreturn}[1]{\xstate{\textbf{return} #1}}

\DeclarePairedDelimiter\ceil{\lceil}{\rceil}
\DeclarePairedDelimiter\floor{\lfloor}{\rfloor}

\newcommand{\bigO}[1]{\mathcal{O}\left(#1\right)}

\title{Математическая логика. Домашнее задание №2}
\author{Горбунов Егор Алексеевич}

\begin{document}
\maketitle

\begin{task}[1]
Докажите, что $\mathbb{N}$ и $\{0,1\}^*$ разрешимо равномощны, где второе множество -- это множество последовательностей из 0 и 1.
\end{task}
\begin{solution}
Будем показывать по теореме Кантора-Берштейна:
\[
	\begin{array}{l}
		f:\mathbb{N} \rightarrow \{0,1\}^* \\
		f(n) = \texttt{Integer.toBinaryString(n)}
	\end{array}
	\begin{array}{l}
		g:\{0,1\}^* \rightarrow \mathbb{N} \\
		g(bs) = 2^{length(bs)}\cdot 3^{\texttt{Integer.parseInt(bs, 2)}}
	\end{array}
\]
Тут подразумевается, что под \texttt{Inteter...} подставится результат выполнения соответствующих функций.

Ясно, что функции \texttt{fpi} и \texttt{gpi} (как в \texttt{cb.hs}) пишутся на хаскеле, первая будет возвращать \texttt{Nothing}, если строка содержит лидирующие нули (и длина более 1), а иначе будет интерпретировать строку как двочичную запись натурального числа; \texttt{gpi} же будет делить до упора на 2, чтобы получить длину, потом делить до упора на 3, чтобы получить число, которое будет переводить в двоичную систему счисления и добавлять нужное число нулей, в зависимости от длины (если на каком-то из шагов что-то не сходится, то \texttt{Nothing}).

А так же можно заметить, что:
\[
		\forall a \in \mathbb{N}:  g(f(a)) > a
\]
А значит \texttt{dfc = const False}. \xqed
\end{solution}

\begin{task}[2]
Докажите, что $\{0,1\}^*$ и $\mathbb{N}_2$ разрешимо равномощны, где второе множество -- это множество двоичных натуральных чисел,
    то есть последовательностей 0 и 1 без ведущих нулей (кроме случая, когда последовательность состоит из одной цифры).
\end{task}
\begin{solution}
Не вижу разницы с предыдущим заданием, так что: см. решение задания 1 (там просто не нужно будет делать лишних вызовав в духе <<найти двоичное представление числа>>. \xqed
\end{solution}

\begin{task}[3]
Докажите, что $\{0,1\}^*$ и множество корректных программ на каком-либо (любом) языке программирования разрешимо равномощны.
\end{task}
\begin{solution}
Обозначим множество корректных программ на \texttt{python} за $\mathbb{CP}$. Опять показываем по теореме Кантора-Берштейна вводя вложения туда и обратно.
\begin{align*}
	f&: \{0,1\}^* \rightarrow CP\\
	f&(bs)=
		\begin{rcases*}
			\texttt{print("Hello!")} \\
			\texttt{print("Hello!")} \\
			\ldots \\
			\texttt{print("Hello!")}
		\end{rcases*}
		length(bs) + 1 \text{ раз}
\end{align*}
\begin{align*}
	g&: CP \rightarrow \{0,1\}^*\\
	g&(s) = \texttt{unicode\_codes\_str(s)}
\end{align*}
Тут \texttt{unicode\_codes\_str(s)} возвращает строку, в которой каждый символ исходной строки программы $s$ был переведён в двухбайтовый код (т.е. строку из $\{0,1\}^{16}$).

Эти функции реализуются на хаскеле достаточно просто, достаточно знать коды символов, чтобы написать $g$. Функции \texttt{fpi} и \texttt{gpi} тоже реализуемы на хаскеле, а ещё видно, что для любой строки $bs$ из $\{0,1\}^*$ в программе $f(bs)$ будет $15(length(bs) + 1)$ символов (не считая переводов строки), а значит в строке $g(f(bs))$ будет $16\cdot 15 \cdot(length(bs) + 1)$ символов, что точно больше $length(bs)$, а значит \texttt{dfc = const False}. \xqed
\end{solution}

\begin{task}[4]
Определите множество простых чисел.
\end{task}
\begin{solution}
$Prime = \{ p \in \mathbb{N}\ |\ (p \neq 1) \wedge (\forall n \in \{2,\ldots,p-1\} : p \mod n \neq 0) \} $ 
\xqed
\end{solution}

\begin{task}[5]
Определите следующие функции над $\mathbb{Q}$ и докажите их корректность:
\begin{enumerate}[label=(\alph*)]
	\item Функция $neg : \mathbb{Q} \to \mathbb{Q}$, возвращающая обратное по сложению число.
	\item Функция $inv : \mathbb{Q}_{\neq 0} \to \mathbb{Q}_{\neq 0}$, возвращающая обратное по умножению число.
	\item Функция $plus : \mathbb{Q} \times \mathbb{Q} \to \mathbb{Q}$, возвращающая сумму двух чисел.
\end{enumerate}
\end{task}
\begin{solution}
\begin{enumerate}[label=(\alph*)]
	\item $neg([(x,y)]_{\sim}) = -(x / gcd(x,y) )/(y / gcd(x,y))$
	\item $inv([(x,y)]_{\sim}) = (y / gcd(x,y) )/(x / gcd(x,y))$
	\item $plus([(a_x,a_y)]_{\sim}, [(b_x,b_y)]_{\sim}) = \frac{(a_xb_y + b_xa_y) / gcd(a_xb_y + b_xa_y,b_xb_y)}{b_xb_y / gcd(a_xb_y + b_xa_y,b_xb_y)}$
\end{enumerate}
\end{solution}
\end{document}
