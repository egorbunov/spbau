\documentclass[12pt, a4paper]{article}
\usepackage[utf8]{inputenc}
\usepackage[russian]{babel}
\usepackage{pscyr}

\usepackage{xifthen}
\usepackage{parskip}
\usepackage{hyperref}
\usepackage[top=0.7in, bottom=1in, left=0.6in, right=0.6in]{geometry}
\usepackage{setspace}

\usepackage{amsmath}
\usepackage{MnSymbol}
\usepackage{amsthm}
\usepackage{mathtools}

\usepackage{algorithm}
\usepackage[noend]{algpseudocode}



\linespread{1.2}
\setlength{\parskip}{0pt}

\renewcommand\familydefault{\sfdefault}


% Stuff related to homework specific documents
\newcounter{MyTaskCounter}
\newcounter{MyTaskSectionCounter}
\newcommand{\tasksection}[1]{
	\stepcounter{MyTaskSectionCounter}
	\setcounter{MyTaskCounter}{0}
	\ifthenelse{\equal{#1}{}}{}{
	{\hfill\\[0.2in] \Large \textbf{\theMyTaskSectionCounter \enspace #1} \hfill\\[0.1in]}}
}

\newcommand{\task}[1]{
	\stepcounter{MyTaskCounter}
	\hfill\\[0.1in]
	\ifthenelse{\equal{\theMyTaskSectionCounter}{0}}{
	   \textbf{\large Задача №\theMyTaskCounter}
	}{
	   \textbf{\large Задача №\theMyTaskSectionCounter.\theMyTaskCounter}
	}
	\ifthenelse{\equal{#1}{}}{}{{\normalsize (#1)}}
	\hfill\\[0.05in]
}

% Math and algorithms

\makeatletter
\renewcommand{\ALG@name}{Алгоритм}
\renewcommand{\listalgorithmname}{Список алгроитмов}

\newenvironment{procedure}[1]
  {\renewcommand*{\ALG@name}{Процедура}
  \algorithm\renewcommand{\thealgorithm}{\thechapter.\arabic{algorithm} #1}}
  {\endalgorithm}

\makeatother

\algrenewcommand\algorithmicrequire{\textbf{Вход:}}
\algrenewcommand\algorithmicensure{\textbf{Выход:}}
\algnewcommand\True{\textbf{true}\space}
\algnewcommand\False{\textbf{false}\space}
\algnewcommand\And{\textbf{and}\space}

\newcommand{\xfor}[3]{#1 \textbf{from} #2 \textbf{to} #3}
\newcommand{\xassign}[2]{\State #1 $\leftarrow$ #2}
\newcommand{\xstate}[1]{\State #1}
\newcommand{\xreturn}[1]{\xstate{\textbf{return} #1}}

\DeclarePairedDelimiter\ceil{\lceil}{\rceil}
\DeclarePairedDelimiter\floor{\lfloor}{\rfloor}

\newcommand{\bigO}[1]{\mathcal{O}\left(#1\right)}

\title{Математическая логика. Домашнее задание №5}
\author{Горбунов Егор Алексеевич}

\begin{document}
\maketitle

\begin{task}[1]
	Между правилами вывода логики и конструкциями в лямбда исчислении существует естественная биекция.
    Например, $\to\!\!I$ соответствует абстракции, а $\to\!\!E$ соответствует аппликации.
    Запишите эту биекцию для остальных правил и конструкций.
\end{task}

\begin{solution}
\begin{enumerate}

\item 
\AxiomC{}
\RightLabel{, $(x : \varphi) \in \Gamma$}
\UnaryInfC{$\Gamma \vdash x : \varphi$}
\DisplayProof
\qquad
$\eqsim$
\qquad
\AxiomC{}
\RightLabel{(var)}
\UnaryInfC{$\phi_1, \ldots, \phi_n \vdash \phi_i$}
\DisplayProof

\item 
\AxiomC{$\Gamma \vdash b : \bot$}
\UnaryInfC{$\Gamma \vdash absurd_\varphi\,b : \varphi$}
\DisplayProof
\qquad
$\eqsim$
\qquad
\AxiomC{$\Gamma \vdash \bot$}
\RightLabel{ $\bot E$}
\UnaryInfC{$\Gamma \vdash \varphi$}
\DisplayProof

\item
\AxiomC{$\Gamma, x : \varphi \vdash b : \psi$}
\UnaryInfC{$\Gamma \vdash \lambda x.\,b : \varphi \to \psi$}
\DisplayProof
\qquad
$\eqsim$
\qquad
\AxiomC{$\Gamma, \varphi \vdash \psi$}
\RightLabel{ $\to I$}
\UnaryInfC{$\Gamma \vdash \varphi \to \psi$}
\DisplayProof

\item
\AxiomC{$\Gamma \vdash f : \varphi \to \psi$}
\AxiomC{$\Gamma \vdash a : \varphi$}
\BinaryInfC{$\Gamma \vdash f\,a : \psi$}
\DisplayProof
\qquad
$\eqsim$
\qquad
\AxiomC{$\Gamma \vdash \varphi \to \psi$}
\AxiomC{$\Gamma \vdash \varphi$}
\RightLabel{ $\to E$}
\BinaryInfC{$\Gamma \vdash \psi$}
\DisplayProof

\item
\AxiomC{$\Gamma \vdash a : \varphi$}
\AxiomC{$\Gamma \vdash b : \psi$}
\BinaryInfC{$\Gamma \vdash (a,b) : \varphi \times \psi$}
\DisplayProof
\qquad
$\eqsim$
\qquad
\AxiomC{$\Gamma \vdash \varphi$}
\AxiomC{$\Gamma \vdash \psi$}
\RightLabel{ $\land I$}
\BinaryInfC{$\Gamma \vdash \varphi \land \psi$}
\DisplayProof

\item
\AxiomC{$\Gamma \vdash p : \varphi \times \psi$}
\UnaryInfC{$\Gamma \vdash fst\,p : \varphi$}
\DisplayProof
\qquad
$\eqsim$
\qquad
\AxiomC{$\Gamma \vdash \varphi \land \psi$}
\RightLabel{ $\land E_1$}
\UnaryInfC{$\Gamma \vdash \varphi$}
\DisplayProof

\item
\AxiomC{$\Gamma \vdash p : \varphi \times \psi$}
\UnaryInfC{$\Gamma \vdash snd\,t : \psi$}
\DisplayProof
\qquad
$\eqsim$
\qquad
\AxiomC{$\Gamma \vdash \varphi \land \psi$}
\RightLabel{ $\land E_2$}
\UnaryInfC{$\Gamma \vdash \psi$}
\DisplayProof

\item
\AxiomC{$\Gamma \vdash a : \varphi$}
\UnaryInfC{$\Gamma \vdash Left\,a : \varphi \amalg \psi$}
\DisplayProof
\qquad
$\eqsim$
\qquad
\AxiomC{$\Gamma \vdash \varphi$}
\RightLabel{ $\lor I_1$}
\UnaryInfC{$\Gamma \vdash \varphi \lor \psi$}
\DisplayProof

\item
\AxiomC{$\Gamma \vdash b : \psi$}
\UnaryInfC{$\Gamma \vdash Right\,b : \varphi \amalg \psi$}
\DisplayProof
\qquad
$\eqsim$
\qquad
\AxiomC{$\Gamma \vdash \psi$}
\RightLabel{ $\lor I_2$}
\UnaryInfC{$\Gamma \vdash \varphi \lor \psi$}
\DisplayProof

\item
\AxiomC{$\Gamma \vdash e : \varphi \amalg \psi$}
\AxiomC{$\Gamma, x : \varphi \vdash a : \chi$}
\AxiomC{$\Gamma, y : \psi \vdash b : \chi$}
\TrinaryInfC{$\Gamma \vdash \mathbf{case}\,e\,\mathbf{of}\,\{\,Left(x) \to a; Right(y) \to b\,\} : \chi$}
\DisplayProof
$\eqsim$
\AxiomC{$\Gamma \vdash \varphi \lor \psi$}
\AxiomC{$\Gamma, \varphi \vdash \chi$}
\AxiomC{$\Gamma, \psi \vdash \chi$}
\RightLabel{ $\lor E$}
\TrinaryInfC{$\Gamma \vdash \chi$}
\DisplayProof
\end{enumerate}
\end{solution}

\begin{task}[2]
	Приведите для следующих теорем деревья вывода и термы, доказывающие их:
	\begin{enumerate}
		\item $P \to P$
		\item $P \to (P \to Q) \to Q$
		\item $P \land Q \to P \lor Q$
		\item $(P \lor Q) \land R \to (P \land R) \lor (Q \land R)$
	\end{enumerate}
\end{task}
\begin{solution}
\begin{enumerate}
\item
\AxiomC{}
\RightLabel{(var)}
\UnaryInfC{$P \vdash P$}
\RightLabel{ $\to I$}
\UnaryInfC{$\vdash P \to P$}
\DisplayProof

Терм: $\lambda x.x,\ x:P$
\vspace{20pt}
\item
\AxiomC{}
\RightLabel{(var)}
\UnaryInfC{$P, (P \to Q) \vdash P \to Q$}
\AxiomC{}
\RightLabel{(var)}
\UnaryInfC{$P, (P \to Q) \vdash P$}
\RightLabel{$\to E$}
\BinaryInfC{$P, (P \to Q) \vdash Q$}
\RightLabel{$\to I$}
\UnaryInfC{$P \vdash (P \to Q) \to Q$}
\RightLabel{$\to I$}
\UnaryInfC{$\vdash P \to (P \to Q) \to Q$}
\DisplayProof

Терм: 
\[ 
	\lambda x.\lambda f.f\ x,\ x:P, f:P \to Q
\]
\vspace{20pt}
\item
\AxiomC{}
\RightLabel{(var)}
\UnaryInfC{$P \land Q \vdash P \land Q$}
\RightLabel{$\land E_1$}
\UnaryInfC{$P \land Q \vdash P$}
\RightLabel{$\lor I_2$}
\UnaryInfC{$P \land Q \vdash P \lor Q$}
\RightLabel{ $\to I$}
\UnaryInfC{$\vdash P \land Q \to P \lor Q$}
\DisplayProof

Терм: 
\[
	\lambda p.Left(fst\ p),\ x:(P, Q)
\]
\vspace{20pt}

\item
\AxiomC{}
\RightLabel{\scriptsize (var)}
\UnaryInfC{$\scriptstyle \Gamma \vdash (P \lor Q) \land R$}
\RightLabel{$\scriptstyle \land E_1$}
\UnaryInfC{$\scriptstyle \Gamma \vdash P \lor Q$}
\AxiomC{}
\RightLabel{\scriptsize (var)}
\UnaryInfC{$\scriptstyle \Gamma, P \vdash P$}
\AxiomC{}
\RightLabel{\scriptsize (var)}
\UnaryInfC{$\scriptstyle \Gamma, P \vdash (P \lor Q) \land R$}
\RightLabel{$\scriptstyle \land E_2$}
\UnaryInfC{$\scriptstyle \Gamma, P \vdash R$}
\RightLabel{$\scriptstyle \land I$}
\BinaryInfC{$\scriptstyle \Gamma, P \vdash P \land R$}
\RightLabel{$\scriptstyle \lor I_1$}
\UnaryInfC{$\scriptstyle \Gamma, P \vdash (P \land R) \lor (Q \land R)$}
\AxiomC{}
\RightLabel{\scriptsize (var)}
\UnaryInfC{$\scriptstyle \Gamma, Q \vdash Q$}
\AxiomC{}
\RightLabel{\scriptsize (var)}
\UnaryInfC{$\scriptstyle \Gamma, Q \vdash (P \lor Q) \land R$}
\RightLabel{$\scriptstyle \land E_2$}
\UnaryInfC{$\scriptstyle \Gamma, Q \vdash R$}
\RightLabel{$\scriptstyle \land I$}
\BinaryInfC{$\scriptstyle \Gamma, Q \vdash Q \land R$}
\RightLabel{$\scriptstyle \lor I_2$}
\UnaryInfC{$\scriptstyle \Gamma, Q \vdash (P \land R) \lor (Q \land R)$}
\RightLabel{$\lor E$}
\TrinaryInfC{$(P \lor Q) \land R \vdash (P \land R) \lor (Q \land R)$}
\RightLabel{$\to I$}
\UnaryInfC{$(P \lor Q) \land R \to (P \land R) \lor (Q \land R)$}
\DisplayProof

Терм:
\[
	\lambda t.\mathbf{case}\ fst\ t\ \mathbf{of} \{ Left(p) \to Left\ (p, snd\ t); Right(q) \to Right\ (q,\ snd\ t)\}
\]
Где $t:(P \lor Q) \land R,\ p:P,\ q:Q$.
\end{enumerate}
\xqed
\end{solution}

\begin{task}[3]
Приведите для следующих теорем доказывающие их термы:
	\begin{enumerate}
		\item $(P \land R) \lor (Q \land R) \to (P \lor Q) \land R$
		\item $(P \lor Q) \lor R \to P \lor (Q \lor R)$
		\item $((((P \to Q) \to P) \to P) \to Q) \to Q$
	\end{enumerate}
\end{task}
\begin{solution}
\begin{enumerate}
\item $(P \land R) \lor (Q \land R) \to (P \lor Q) \land R$
\begin{align*}
	\lambda e.\mathbf{case}\ e\ \mathbf{of}\ \{ &Left(p) \to (Left\ (fst\ p), snd\ p); \\
	                                            &Right(q) \to (Right\ (fst\ p), snd\ q) \}
\end{align*}

\item $(P \lor Q) \lor R \to P \lor (Q \lor R)$
\begin{align*}
	\lambda e.\mathbf{case}\ e\ \mathbf{of}\ \{ &Left(s) \to \mathbf{case}\ s\ \mathbf{of}\ \{  \\
											   &\qquad Left(p) \to Left\ p; \\
	                                           &\qquad Right(q) \to Right\ (Left\ q) \}; \\
	                                            &Right(r) \to Right\ (Right\ r)  \}
\end{align*}

\item $((((P \to Q) \to P) \to P) \to Q) \to Q$
\begin{align*}
	\lambda f.f\ (\lambda g.g\ (\lambda h.f\ (\lambda z.h)))
\end{align*}
Типы получаются такие:
\[
\begin{array}{l}
	f : ((((P \to Q) \to P) \to P) \to Q)\\
	g : (P \to Q) \to P\\
    h : P\\
    z : (P \to Q) \to P
\end{array}
\]
\end{enumerate}
\xqed
\end{solution}

\begin{task}[4]
Добавим в нашей логике новую связку $\leftrightarrow$, удовлетворяющую следующим условиям:
	\begin{align*}
		\top \leftrightarrow \top & = \top \\
		\top \leftrightarrow \bot & = \bot \\
		\bot \leftrightarrow \top & = \bot \\
		\bot \leftrightarrow \bot & = \top
	\end{align*}
	\begin{enumerate}
		\item Опишите правила введения и элиминации для этой связки.
		    Они не должны использовать никакие другие связки.
		\item Опишите аналогичные конструкции и правила типизации для них в лямбда исчислении.
		\item Приведите терм, доказывающий формулу $(P \lor Q \to R) \leftrightarrow (P \to R) \land (Q \to R)$.
	\end{enumerate}
\end{task}
\begin{solution}
\begin{enumerate}
	\item
	Введение:
	\begin{center}
	\AxiomC{$\Gamma, \varphi \vdash \psi$}
	\AxiomC{$\Gamma, \psi \vdash \varphi$}
	\RightLabel{ $\leftrightarrow I$}
	\BinaryInfC{$\Gamma \vdash \varphi \leftrightarrow \psi$}
	\DisplayProof
	\end{center}
	Иллиминация:
	\begin{center}
	\AxiomC{$\Gamma \vdash \varphi \leftrightarrow \psi$}
	\AxiomC{$\Gamma \vdash \varphi$}
	\RightLabel{ $\leftrightarrow E_1$}
	\BinaryInfC{$\Gamma \vdash \psi$}
	\DisplayProof
	\qquad
	\AxiomC{$\Gamma \vdash \varphi \leftrightarrow \psi$}
	\AxiomC{$\Gamma \vdash \psi$}
	\RightLabel{ $\leftrightarrow E_2$}
	\BinaryInfC{$\Gamma \vdash \varphi$}
	\DisplayProof
	\end{center}
	\vspace{20pt}

	\item Кажется, что лямбда-аналогом может служить пара лямбда-абстракций. Введению соответствует правило типизации:
	\begin{center}
	\AxiomC{$\Gamma, a:\varphi \vdash b:\psi$}
	\AxiomC{$\Gamma, c:\psi \vdash d:\varphi$}
	\BinaryInfC{$\Gamma \vdash (\lambda a . b, \lambda c . d) : \varphi \to \psi \times \psi \to \varphi$}
	\DisplayProof
	\end{center}
	Иллиминациям соответствуют правила типизации:
	\begin{center}
	\AxiomC{$\Gamma\vdash p:\varphi \to \psi \times \psi \to \varphi$}
	\AxiomC{$\Gamma \vdash a:\varphi$}
	\BinaryInfC{$\Gamma \vdash (fst\ p)\ a : \psi$}
	\DisplayProof
	\qquad
	\AxiomC{$\Gamma\vdash p:\varphi \to \psi \times \psi \to \varphi$}
	\AxiomC{$\Gamma \vdash b:\psi$}
	\BinaryInfC{$\Gamma \vdash (snd\ p)\ b : \varphi$}
	\DisplayProof
	\end{center}
	\vspace{20pt}

	\item Терм, доказывающий $(P \lor Q \to R) \leftrightarrow (P \to R) \land (Q \to R)$:
	\begin{align*}
		( \\
			& \lambda f. (\lambda p . f\ (Left\ p), \lambda q. f\ (Right\ q))\\
			& , \\
			& \lambda p . \lambda e . \mathbf{case}\ e\ \mathbf{of} \{ Left(x) \to (fst\ p)\ x, Right(y) \to (snd\ p)\ y\}\\
		)
	\end{align*}


\end{enumerate}
\end{solution}
\end{document}
