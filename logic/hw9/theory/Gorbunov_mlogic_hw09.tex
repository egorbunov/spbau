%!TEX program = xelatex

\documentclass[12pt, a4paper]{article}
\usepackage[utf8]{inputenc}
\usepackage[russian]{babel}
\usepackage{pscyr}

\usepackage{xifthen}
\usepackage{parskip}
\usepackage{hyperref}
\usepackage[top=0.7in, bottom=1in, left=0.6in, right=0.6in]{geometry}
\usepackage{setspace}

\usepackage{amsmath}
\usepackage{MnSymbol}
\usepackage{amsthm}
\usepackage{mathtools}

\usepackage{algorithm}
\usepackage[noend]{algpseudocode}



\linespread{1.2}
\setlength{\parskip}{0pt}

\renewcommand\familydefault{\sfdefault}


% Stuff related to homework specific documents
\newcounter{MyTaskCounter}
\newcounter{MyTaskSectionCounter}
\newcommand{\tasksection}[1]{
	\stepcounter{MyTaskSectionCounter}
	\setcounter{MyTaskCounter}{0}
	\ifthenelse{\equal{#1}{}}{}{
	{\hfill\\[0.2in] \Large \textbf{\theMyTaskSectionCounter \enspace #1} \hfill\\[0.1in]}}
}

\newcommand{\task}[1]{
	\stepcounter{MyTaskCounter}
	\hfill\\[0.1in]
	\ifthenelse{\equal{\theMyTaskSectionCounter}{0}}{
	   \textbf{\large Задача №\theMyTaskCounter}
	}{
	   \textbf{\large Задача №\theMyTaskSectionCounter.\theMyTaskCounter}
	}
	\ifthenelse{\equal{#1}{}}{}{{\normalsize (#1)}}
	\hfill\\[0.05in]
}

% Math and algorithms

\makeatletter
\renewcommand{\ALG@name}{Алгоритм}
\renewcommand{\listalgorithmname}{Список алгроитмов}

\newenvironment{procedure}[1]
  {\renewcommand*{\ALG@name}{Процедура}
  \algorithm\renewcommand{\thealgorithm}{\thechapter.\arabic{algorithm} #1}}
  {\endalgorithm}

\makeatother

\algrenewcommand\algorithmicrequire{\textbf{Вход:}}
\algrenewcommand\algorithmicensure{\textbf{Выход:}}
\algnewcommand\True{\textbf{true}\space}
\algnewcommand\False{\textbf{false}\space}
\algnewcommand\And{\textbf{and}\space}

\newcommand{\xfor}[3]{#1 \textbf{from} #2 \textbf{to} #3}
\newcommand{\xassign}[2]{\State #1 $\leftarrow$ #2}
\newcommand{\xstate}[1]{\State #1}
\newcommand{\xreturn}[1]{\xstate{\textbf{return} #1}}

\DeclarePairedDelimiter\ceil{\lceil}{\rceil}
\DeclarePairedDelimiter\floor{\lfloor}{\rfloor}

\newcommand{\bigO}[1]{\mathcal{O}\left(#1\right)}


\title{Математическая логика. Домашнее задание №8}
\author{Горбунов Егор Алексеевич}

\begin{document}
\maketitle

\begin{task}[1]
	Определите формулу $\varphi(x,y)$, задающую график функции $pred$, удовлетворяющей следующим условиям:
	\begin{align*}
		pred(0) & = 0 \\
		pred(S(x)) & = x
	\end{align*}
	Докажите, что $\forall x \exists! y (\varphi(x,y))$
\end{task}
\begin{solution}
	Определим формулу:
	\[
		\varphi(x, y) = (x = 0 \land y = 0) \lor (x = S(y))
	\]
	Доказываем по индукции утверждение:
	\[
		\forall x \exists! y ((x = 0 \land y = 0) \lor (x = S(y)))
	\]
	База для $x = 0$ верна, т.к. у нас существует единственный $y = 0$ (по аксиоме $(i0)$). Пускай теперь верно, что существует единственный $y$, что $\varphi(n, y)$. Тогда (часть про $x = 0$ убрал сознательно):
	\[
		\varphi(S(n), y) = (S(n) = S(y))
	\]
	Но по аксиоме $(iS)$ тогда получаем, что $y = n$ и этот $y$ единтсвенный. Далее по аксиоме об индукции заканчиваем доказательство. \xqed
\end{solution}

\begin{task}[2]
Докажите, что аксиомы для сложения определяют его уникальным образом.
    То есть если мы добавим в сигнатуру новый функциональный символ $+'$ и новые аксиомы
\begin{align*}
\forall y\ & 0 +' y = y \tag{$+'0$} \\
\forall x \forall y\ & S(x) +' y = S(x +' y) \tag{$+'S$},
\end{align*}
то в ней будет доказуема формула $\forall x \forall y\ (x + y = x +' y)$.
\end{task}
\begin{solution}
Будем доказывать по индукции следующим образом: 
\begin{itemize}
	\item \textbf{База:} если $x = 0$ и $y = 0$, то $x + y = 0 + 0 = 0$ по аксиоме $(+0)$, аналогично по $(+'0)$ получаем $x +' y = 0$, а значит, т.к. все стандартные натуральные числа различны имеем $(x + y = x +'$
	\item \textbf{Переход 1.} Покажем, что если утверждение верно для $x = n$ и $y = m$, то оно верно для $x = S(n)$ и $y = m$:
	\begin{align*}
		x + y &= S(n) + m =^{(+S)} = S(n + m)\\
		x +' y &= S(n) +' m =^{(+'S)} = S(n +' m) =^{\text{предположение индукции}} = S(n + m)
	\end{align*}
	Видим, что $x + y = x +' y$ верно.
	\item \textbf{Переход 2.} Аналогично показывается, что $(n + m = n +' m)$ влечёт $(n + S(m) = n +' S(m))$ с использованием коммутативность, которую можно доказать.
\end{itemize}
Выше использовалась индукция, отличная от той, что используется в арифметике Пеано, но они эквивалентны. По сути мы провели индукцию по парам $(x, y)$. Мы можем однозначно и взаимно сопоставить паре натуральных чисел $(x, y)$ натуральное число (номер) так, что номер $(n, m)$ меньше номера $(S(n), m)$ и $(n, S(m))$. Таким образом мы свели индукцию выше к индукции, которая используется в арифметике Пеано. \xqed
\end{solution}

\begin{task}[3]
Добавим в сигнатуру функциональные символы $exp$ и $fac$ для функций возведения в степень и факториала соответственно.
    Напишите аксиомы для этих функциональных символов, определяющие их уникальным образом.
\end{task}
\begin{solution}
Возведение в степень:
\begin{align*}
\forall x\ & x\ exp\ 0 = S(0) \tag{$exp0$} \\
\forall x \forall y\ & x\ exp\ S(y) = x \cdot (x\ exp\ y) \tag{$expS$},
\end{align*}
Факториал
\begin{align*}
& fac\ 0 = S(0) \tag{$fac0$} \\
\forall x\ & fac\ S(x) = S(x) \cdot (fac\ x) \tag{$facS$},
\end{align*}
\end{solution}

\begin{task}[4]
	Докажите следующие свойства:
	\begin{enumerate}
		\item $\forall x \forall y\ (x + y = 0 \to x = 0 \land y = 0)$.
		\item $\forall x \forall y\ (x \cdot y = 0 \to x = 0 \lor y = 0)$.
	\end{enumerate}
\end{task}
\begin{solution}
\begin{enumerate}
	\item Нам просто нужно показать, что если при заданных $x$ и $y$ выводимо, что $x + y = 0$, то 
	тогда $x = 0 \land y = 0$ тоже выводимо. Будем перебирать все пары $x$ и $y$. Ясно, что при $x = 0$ и $y = 0$, $x + y = 0$ выводимо (через аксиомы сложения) и, как видим, $x = 0 \land y = 0$ тоже выводимо. Теперь, если $x = S(n)$, $y = m$, то $x + y = S(n) + m = S(n + m) \neq 0$, т. е. получили, что $x + y \neq 0$ и, т.к. $x = S(n) \neq 0$, то не верно, что $x = 0 \land y = 0$. Аналогично при $x = n, y = S(m)$...
	\item Тут нужно рассуждать, видимо, так же, только случай $x = S(n), y = m$ разбить на $x = S(n), y = 0$ и $x = S(x), y = S(m)$. 
\end{enumerate}
\xqed
\end{solution}

\begin{task}[5]
Докажите коммутативность сложения.
\end{task}
\begin{solution}
Покажем сначала, что $S(x) + y = x + S(y)$: (индукция по $x$ внутри которой индукция по $y$)
\begin{enumerate}
	\item База: $S(0) + y = 0 + S(y)$, т.к. $S(0) + y =^{(+S)} = S(0 + y) =^{(+0)} = S(y)$ (это верно для любого $y$)
	\item Пусть верно для $S(n) + m = n + S(m)$, покажем, что $S(S(n)) + m = S(n) + S(m)$: (так последовательно для каждого $m$)
	\begin{align*}
		S(S(n) + m) &=^{(+S)} = S(S(n + m))\\
		S(n) + S(m) &= S(n + S(m)) =^{\text{предположение}} = S(S(n) + m) = S(S(n + m))
	\end{align*}
\end{enumerate}
Показали. Обозначим это утверждение $(\alpha)$\\
Аналогичной индукцией покажем коммутативность:
\begin{enumerate}
	\item База: $0 + m = m + 0$. Базу тоже доказываем по индукции:
		\begin{enumerate}
			\item База: $0 + 0 = 0 + 0$ (это ясно по аксиомам)
			\item Пусть верно для $0 + n = n + 0$, тогда: 
			\begin{align*}
				0 + S(n) &=^{(\alpha)} S(0) + n =^{(+S)} S(0 + n) =^{\text{предп.}} \\
				         &= S(n + 0) = S(n) + 0
			\end{align*} 
			Показали.
		\end{enumerate}
	\item Пускай верно для $n + m = m + n$. Покажем, что $n + S(m) = S(m) + n$ (т.е. $S(m) + n = n + S(m)$:
	\begin{align*}
		S(m) + n = S(m + n) =^{\text{предположение}} = S(n + m) =^{(+S)} = S(n) + m =^{(\alpha)} = n + S(m)
	\end{align*}
	Доказали.
\end{enumerate}
\xqed
\end{solution}


\end{document}
