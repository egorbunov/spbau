\documentclass[12pt, a4paper]{article}
\usepackage[utf8]{inputenc}
\usepackage[russian]{babel}
\usepackage{pscyr}

\usepackage{xifthen}
\usepackage{parskip}
\usepackage{hyperref}
\usepackage[top=0.7in, bottom=1in, left=0.6in, right=0.6in]{geometry}
\usepackage{setspace}

\usepackage{amsmath}
\usepackage{MnSymbol}
\usepackage{amsthm}
\usepackage{mathtools}

\usepackage{algorithm}
\usepackage[noend]{algpseudocode}



\linespread{1.2}
\setlength{\parskip}{0pt}

\renewcommand\familydefault{\sfdefault}


% Stuff related to homework specific documents
\newcounter{MyTaskCounter}
\newcounter{MyTaskSectionCounter}
\newcommand{\tasksection}[1]{
	\stepcounter{MyTaskSectionCounter}
	\setcounter{MyTaskCounter}{0}
	\ifthenelse{\equal{#1}{}}{}{
	{\hfill\\[0.2in] \Large \textbf{\theMyTaskSectionCounter \enspace #1} \hfill\\[0.1in]}}
}

\newcommand{\task}[1]{
	\stepcounter{MyTaskCounter}
	\hfill\\[0.1in]
	\ifthenelse{\equal{\theMyTaskSectionCounter}{0}}{
	   \textbf{\large Задача №\theMyTaskCounter}
	}{
	   \textbf{\large Задача №\theMyTaskSectionCounter.\theMyTaskCounter}
	}
	\ifthenelse{\equal{#1}{}}{}{{\normalsize (#1)}}
	\hfill\\[0.05in]
}

% Math and algorithms

\makeatletter
\renewcommand{\ALG@name}{Алгоритм}
\renewcommand{\listalgorithmname}{Список алгроитмов}

\newenvironment{procedure}[1]
  {\renewcommand*{\ALG@name}{Процедура}
  \algorithm\renewcommand{\thealgorithm}{\thechapter.\arabic{algorithm} #1}}
  {\endalgorithm}

\makeatother

\algrenewcommand\algorithmicrequire{\textbf{Вход:}}
\algrenewcommand\algorithmicensure{\textbf{Выход:}}
\algnewcommand\True{\textbf{true}\space}
\algnewcommand\False{\textbf{false}\space}
\algnewcommand\And{\textbf{and}\space}

\newcommand{\xfor}[3]{#1 \textbf{from} #2 \textbf{to} #3}
\newcommand{\xassign}[2]{\State #1 $\leftarrow$ #2}
\newcommand{\xstate}[1]{\State #1}
\newcommand{\xreturn}[1]{\xstate{\textbf{return} #1}}

\DeclarePairedDelimiter\ceil{\lceil}{\rceil}
\DeclarePairedDelimiter\floor{\lfloor}{\rfloor}

\newcommand{\bigO}[1]{\mathcal{O}\left(#1\right)}

\title{Математическая логика. Домашнее задание №1}
\author{Горбунов Егор Алексеевич}

\begin{document}
\maketitle

\begin{task}[1]
Определите множество частичных функций через множество их графиков.
\end{task}
\begin{solution}
Множество частичный функций $A \rightarrow B$ --- это множество подмножеств $G \subseteq A \times B$ такое, что если $(a, b_1) \in G$ и $(a, b_2) \in G$, то $b_1 = b_2$. Т.е. множество частичных функций --- это просто множество графиков.
\end{solution}

\begin{task}[2]
Пусть $f:A\rightarrow B$ и $g:B \rightarrow C$. Задайте графк функции $g \circ f$ через графики функций $g$ и $f$.
\end{task}
\begin{solution} Пусть $G_f$ и $G_g$ --- графики функций $f$ и $g$.
$g\circ f: A \rightarrow C$ и её график: $G_{g\circ f}$ задаётся так:
\[
	(a, c) \in G_{g\circ f} \iff \exists b \in B : (b, c) \in G_g, (a, b) \in G_f
\]
\end{solution}

\begin{task}[3]
Докажите, что если $A$ вкладывается в $B$ и $B$ вкладывается в $C$, то $A$ вкладывается в $C$.
\end{task}
\begin{solution}
По определению, т.к. $A$ вкладывается в $B$ существует инъекция (вложение) $f:A \rightarrow B$.
Аналогично существует инъекция $g:B \rightarrow C$. Покажем, что $h = g \circ f: A \rightarrow C$ тоже является инъекцией. Пусть $h(a_1) = h(a_2)$, тогда:
\begin{equation*}
\begin{split}
g(f(a_1)) = g(f(a_2)) \Rightarrow f(a_1) = f(a_2) \Rightarrow a_1 = a_2
\end{split}
\end{equation*}
Оба перехода верны в силу тогоа, что $f$ и $g$ --- инъекции. Таким образом и $h$ --- инъекция. \xqed
\end{solution}

\begin{task}[4]
Доказать, что если $A$ накрывает $B$ и $B$ накрывает $C$, то $A$ накрывает $C$.
\end{task}
\begin{solution}
Аналогично предыдущей задаче определяем накрытие $f: A \rightarrow B$ и $g: B \rightarrow C$.
Рассмотрим любой $a$. Точно существует $b \in B: f(a) = b$, тогда точно существует $c \in C: g(b) = c$, т.е. точно существует $c \in C: g(f(a)) = c$, т.е. $g \circ f: A \rightarrow C$ --- накрытие, а значит $A$ накрывает $C$. \xqed
\end{solution}
\newpage
\begin{task}[5]
Доказать:
\begin{enumerate}[label=(\alph*)]
	\item $A$ равномощно $A$
	\item Если $A$ равномощно $B$, то $B$ равномощно $A$
	\item Есди $A$ равномощно $B$ и $B$ равномощно $C$, то $A$ равномощно $C$
\end{enumerate}
\end{task}
\begin{solution}
\begin{enumerate}[label=(\alph*)]
	\item Биекция $f:A \rightarrow A$: $f(a) = a$. \xqed
	\item Существует биекция $f: A \rightarrow B$. Рассмотрим функцию $f^{-1}: B \rightarrow A$ такую, что $f^{-1}(b) = a \iff f(a) = b$. Заметим, что $f(f^{-1}(b)) = b$ и $f^{-1}(f(a)) = a$ в силу определения (достаточно походить по стрелке влево и вправо). Но тогда, по доказанному на лекции (характеристики биекции): $f^{-1}$ --- биекция, а значит $B$ равномощно $A$. \xqed
	\item Существуют биекции: $f: A \rightarrow B$ и $g: B \rightarrow C$. Рассмотрим $g\circ f: A \rightarrow C$. Мы в заданиях $4$ и $3$ показали, что $g\circ f$ --- это накрытие и вложение, а значит это биекция, т.е. $A$ и $C$ равномощны \xqed
\end{enumerate}
\end{solution}

\begin{task}[6]
Доказать, что $A \amalg B \rightarrow C$ и $(A \rightarrow B) \times (B \rightarrow C)$ равномощны. 
\end{task}
\begin{solution} 
Посмтроим $f : (A \amalg B \rightarrow C) \rightarrow (A \rightarrow B) \times (B \rightarrow C) $:
\[
	f(f') = (x \mapsto f'(Left(x)), x \mapsto f'(Right(x)))
\]
Теперь построим $g : (A \rightarrow B) \times (B \rightarrow C) \rightarrow (A \amalg B \rightarrow C)$:
\[
g((g_1, g_2)) = fun, \text{где} 
\begin{cases}
	fun(Left(a)) = g_1(a)\\
	fun(Right(b)) = g_2(b) 
\end{cases}
\]
Построили взаимнообратные функции, т.е. $f$ и $g$ --- биекции, а значит доказано. \xqed
\end{solution}

\begin{task}[7]
Докажите, что $A \times B \rightarrow C$ и $A \rightarrow (B \rightarrow C)$ равномощны.
\end{task}
\begin{solution}
Построим взаимнообратные $f$ и $g$.
\begin{align*}
f &: (A \times B \rightarrow C) \rightarrow (A \rightarrow (B \rightarrow C)) \\
f&(f') = a \mapsto (b \mapsto f'((a, b))) \\
g &: (A \rightarrow (B \rightarrow C)) \rightarrow (A \times B \rightarrow C) \\
g&(g') = (a, b) \mapsto (g'(a))(b) \\
\end{align*}
\end{solution}

\begin{task}[8]
Докажите, что $|\mathcal{P}(A)| = 2^{|A|}$ и $|A \rightarrow B| = |B|^{|A|}$
\end{task}
\begin{solution}
В лекции мы показывали, что есть биекция между $\mathcal{P}$ и $A \rightarrow \Omega$, т.е. достаточно доказать $|A \rightarrow B| = |B|^{|A|}$. А для этого достаточно доказать, что
$|\overline{n} \rightarrow \overline{k}| = |\overline{k}|^|\overline{n}|$. Легко построить взаимнообратные функции для доказательства этого (код из задания в \texttt{car.hs}):
\begin{lstlisting}[]
exp_bij
    :: Int 
    -> Int
    -> ((Int -> Int) -> Int, Int -> (Int -> Int)) 
exp_bij n k = (\f -> sum $ zipWith (*) [k^i | i <- [0..n-1]] [f x | x <- [0..n]],
               \x -> (\i -> (x `div` (k^i)) `mod` k))
\end{lstlisting}
\end{solution}
\end{document}
