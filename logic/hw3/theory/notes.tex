\documentclass[12pt, a4paper]{article}
\usepackage[utf8]{inputenc}
\usepackage[russian]{babel}
\usepackage{pscyr}
\usepackage{amssymb}
\usepackage{xifthen}
\usepackage{parskip}
\usepackage{hyperref}
\usepackage{setspace}

\usepackage{graphicx}
\usepackage{xcolor}
\usepackage{amsmath}
\usepackage{MnSymbol}
\usepackage{amsthm}
\usepackage{mathtools}
\usepackage{algorithm}
\usepackage[noend]{algpseudocode}
\usepackage[shortlabels]{enumitem}
                    \setlist[enumerate, 1]{1\textsuperscript{o}}
\usepackage{subfig}
\usepackage{tikz}
\usepackage{tikz,fullpage}
\usetikzlibrary{shapes,snakes}
\usetikzlibrary{arrows,%
                petri,%
                topaths}%
\usepackage{tkz-berge}
\usepackage[top=0.5in, bottom=0.7in, left=0.6in, right=0.6in]{geometry}
\linespread{1.3}

% \renewcommand\familydefault{\sfdefault}


% Stuff related to homework specific documents
\newcounter{MyTaskCounter}
\newcounter{MyTaskSectionCounter}
\newcommand{\tasksection}[1]{
	\stepcounter{MyTaskSectionCounter}
	\setcounter{MyTaskCounter}{0}
	\ifthenelse{\equal{#1}{}}{}{
	{\hfill\\[0.2in] \Large \textbf{\theMyTaskSectionCounter \enspace #1} \hfill\\[0.1in]}}
}

\newcommand{\task}[1]{
	\stepcounter{MyTaskCounter}
	\hfill\\[0.1in]
	\ifthenelse{\equal{\theMyTaskSectionCounter}{0}}{
	   \textbf{\large Задача №\theMyTaskCounter}
	}{
	   \textbf{\large Задача №\theMyTaskSectionCounter.\theMyTaskCounter}
	}
	\ifthenelse{\equal{#1}{}}{}{{\normalsize (#1)}}
	\hfill\\[0.05in]
}

% Math and algorithms

\makeatletter
\renewcommand{\ALG@name}{Алгоритм}
\renewcommand{\listalgorithmname}{Список алгроитмов}

\newenvironment{procedure}[1]
  {\renewcommand*{\ALG@name}{Процедура}
  \algorithm\renewcommand{\thealgorithm}{\thechapter.\arabic{algorithm} #1}}
  {\endalgorithm}

\makeatother

\algrenewcommand\algorithmicrequire{\textbf{Вход:}}
\algrenewcommand\algorithmicensure{\textbf{Выход:}}
\algnewcommand\True{\textbf{true}\space}
\algnewcommand\False{\textbf{false}\space}
\algnewcommand\And{\textbf{and}\space}

\newcommand{\xfor}[3]{#1 \textbf{from} #2 \textbf{to} #3}
\newcommand{\xassign}[2]{\State #1 $\leftarrow$ #2}
\newcommand{\xstate}[1]{\State #1}
\newcommand{\xreturn}[1]{\xstate{\textbf{return} #1}}

\DeclarePairedDelimiter\ceil{\lceil}{\rceil}
\DeclarePairedDelimiter\floor{\lfloor}{\rfloor}

\newcommand{\bigO}[1]{\mathcal{O}\left(#1\right)}

\newcommand{\xqed}{\hfill $\blacksquare$}

\newcommand{\code}[1]{\colorbox{gray!15}{\footnotesize\texttt{#1}}}

\begin{document}

\section{Принципы индукции и рекурсии}

\begin{enumerate}
	\item \textbf{Принцип индукции.} Пусть есть множество $\mathbb{N}, 0 \in \mathbb{N}$ и задана функция $S(n): \mathbb{N} \rightarrow \mathbb{N}$, тогда оно (множество) удовлетворяет принципу индукции, если, имея: 
	\begin{itemize}
		\item Предикат $P$
		\item Доказательство того, что верно $P(0)$
		\item Доказательство для всех $n \in \mathbb{N}$ того, что если $P(n)$ верно, то верно и $P(S(n))$
	\end{itemize}
	Мы получаем, что $\forall n \in \mathbb{N}$ верно, что $P(n)$.

	\item \textbf{Принцип рекурсии.} Пусть есть множество $\mathbb{N}, 0 \in \mathbb{N}$ и задана функция $S(n): \mathbb{N} \rightarrow \mathbb{N}$, тогда оно (множество) удовлетворяет принципу рекурсии, если, имея:
	\begin{itemize}
		\item $B$ --- множество
		\item $b$ --- элемент из $B$
		\item $e$ --- выражение, которое может содержать $f(n)$ и задавать элемент из $B$
	\end{itemize} 
	Можно задать функцию $f: \mathbb{N} \rightarrow B$, удовлетворяющую этим свойствам.
\end{enumerate}

\section{Принцип Лейбница}

Для симметричности: $\phi(x) = (x = t_2)$
Дли конгруэнтности: $\phi(x) = \big( f(t_1, ..., t_i, ..., t_n) = f(t_1,...,x,...,t_n)\big)$ 
Для транзитивности: ...

\section{Принцип}
Пусть есть множество $T \subset Form$ --- набор теорем. Тогда принципом называем такую штуку $P$, что $P(T) \subset Form$. И для любого $T \subset P(T)$ верно $P(P(T)) = P(T)$ и $P(T) \subset P'(T)$. (замыкание).


\end{document}