\documentclass[12pt, a4paper]{article}
\usepackage[utf8]{inputenc}
\usepackage[russian]{babel}
\usepackage{pscyr}

\usepackage{xifthen}
\usepackage{parskip}
\usepackage{hyperref}
\usepackage[top=0.7in, bottom=1in, left=0.6in, right=0.6in]{geometry}
\usepackage{setspace}

\usepackage{amsmath}
\usepackage{MnSymbol}
\usepackage{amsthm}
\usepackage{mathtools}

\usepackage{algorithm}
\usepackage[noend]{algpseudocode}



\linespread{1.2}
\setlength{\parskip}{0pt}

\renewcommand\familydefault{\sfdefault}


% Stuff related to homework specific documents
\newcounter{MyTaskCounter}
\newcounter{MyTaskSectionCounter}
\newcommand{\tasksection}[1]{
	\stepcounter{MyTaskSectionCounter}
	\setcounter{MyTaskCounter}{0}
	\ifthenelse{\equal{#1}{}}{}{
	{\hfill\\[0.2in] \Large \textbf{\theMyTaskSectionCounter \enspace #1} \hfill\\[0.1in]}}
}

\newcommand{\task}[1]{
	\stepcounter{MyTaskCounter}
	\hfill\\[0.1in]
	\ifthenelse{\equal{\theMyTaskSectionCounter}{0}}{
	   \textbf{\large Задача №\theMyTaskCounter}
	}{
	   \textbf{\large Задача №\theMyTaskSectionCounter.\theMyTaskCounter}
	}
	\ifthenelse{\equal{#1}{}}{}{{\normalsize (#1)}}
	\hfill\\[0.05in]
}

% Math and algorithms

\makeatletter
\renewcommand{\ALG@name}{Алгоритм}
\renewcommand{\listalgorithmname}{Список алгроитмов}

\newenvironment{procedure}[1]
  {\renewcommand*{\ALG@name}{Процедура}
  \algorithm\renewcommand{\thealgorithm}{\thechapter.\arabic{algorithm} #1}}
  {\endalgorithm}

\makeatother

\algrenewcommand\algorithmicrequire{\textbf{Вход:}}
\algrenewcommand\algorithmicensure{\textbf{Выход:}}
\algnewcommand\True{\textbf{true}\space}
\algnewcommand\False{\textbf{false}\space}
\algnewcommand\And{\textbf{and}\space}

\newcommand{\xfor}[3]{#1 \textbf{from} #2 \textbf{to} #3}
\newcommand{\xassign}[2]{\State #1 $\leftarrow$ #2}
\newcommand{\xstate}[1]{\State #1}
\newcommand{\xreturn}[1]{\xstate{\textbf{return} #1}}

\DeclarePairedDelimiter\ceil{\lceil}{\rceil}
\DeclarePairedDelimiter\floor{\lfloor}{\rfloor}

\newcommand{\bigO}[1]{\mathcal{O}\left(#1\right)}

\title{Алгоритмы. Домашнее задание №3}
\author{Горбунов Егор Алексеевич}

\begin{document}
\maketitle

\task{про слегка перемешавшиеся патроны}
\begin{enumerate}[label=(\alph*)]
\item \small[ сортировка за $\bigO{nk}$ \small] Ясно, что т.к. $1$-ый по порядку патрон не мог оказаться дальше, чем на $k$-ой позиции, то
за $\bigO{k}$ операций легко найти его, пробежавшись по первым $k$ элементам и поменять местами с текущим патроном на первой позиции. Теперь на первой позиции нужный патрон. Тогда аналогично для $2$-ого по порядку патрона: он точно не дальше, чем на $k+1$ позиции. Тогда опять за $\bigO{k}$ операций находим его и сажаем на $2$-ую позицию. И т.д. мы после $i$-ого шага будем иметь первые $i$ патронов в отсортированном порядке, причём операций затрачено $\bigO{ki}$. Итого в конце будем иметь отсортированный массив, за $\bigO{kn}$ \xqed
\item \small[ сортирока за $\bigO{n+I}$ \small] Для $i=0$ число инверсий в массиве $\leq k-1$. Аналогично для остальных $i$. Но тогда инверсий в перестановке патронов $\bigO{n(k-1)}$ = $\bigO{nk-n}$. Т.е. $I \leq (nk-n)$. У нас есть алгоритм из предыдущего пункта, который работает за $\bigO{nk}$. Но $\bigO{nk}=\bigO{n + (nk-n)}=\bigO{n+I}$ \xqed
\item Предположим, что можем отсортировать патроны быстрее, чем за $\Omega{n\log{k}}$, но тогда, если $k=n$, то можем отсортировать обычный массив, без доп. условий на элементы, быстрее, чем за $\bigO{n\log{n}}$, что невозможно \xqed
\item Рассмотрим алгоритм из пункта (a). На каждом $i$-ом шаге этого алгоритма мы ищем минимум среди элементов с номерами $i,\ldots,i+k$. Заметим, что это можно реализовать используя кучу: добавим первые $k$ элементов массива в кучу, с операцией $extractMin$, извлечём минимум, положим его на место первого элемента массива (тут куча не могла сломаться, если что), а потом добавим в кучу $k+1$ элемент и снова извлечём минимум. Таким образом всего операций с кучей $\bigO{n}$, а высота кучи всегда $\bigO{k}$, а значит суммарная сложность алгоритма $\bigO{n\log{k}}$ \xqed
\end{enumerate}

\task{$p_i$-ые порядковые статистики}

\task{про перестановку $p$ максимизирующую сумму $a_{p(i)}b_i$}
Очевидно, что максимальная такая сумма равна:
\[\sum_{i=1}^{n}{a_{sort_a(i)}b_{sort_b(i)}}\]
Действительно, пускай в сумме $\sum_{i=1}^{n}{a_ib_i}$ нет слагаемого $a_{max}b_{max}$ ($a_{max}$ --- максимальный элемент в $a$, $b_{max}$ аналогично, но вместо него есть слагаемые $a_{max}b_i + b_{max}a_j$. Покажем тогда, что $a_{max}b_{max}+a_jb_i > a_{max}b_i + b_{max}a_j$, т.е. произведение максимальных брать выгоднее:
\[
	(a_{max}b_i+b_{max}a_j)-(a_{max}b_{max}+a_jb_i) = a_{max}(b_i-b_{max}) + a_j(b_{max}-b_i) = 
	  (a_{max}-a_j)(b_i-b_{max}) \leq 0 
\]
Тут $sort_a$ - перестановка, соответствующая сортировке элементов массива $a$ по убыванию (или по возрастанию). Аналогично $sort_b(i)$
Тогда перестановку $p$ нужно задать таким образом, чтобы, $a_{p(i)}$ был элементом, стоящим на той же позиции в $a$ после его сортировки, что и $b_i$ в массиве $b$ после сортировки. Ясно, что такая перестановка это:
\[p = sort_a^{-1}\circ sort_b\]
Тогда соответственно:
\[p(i) = sort_a^{-1}\circ sort_b(i) = \overbrace{sort_a^{-1}(\overbrace{sort_b(i)}^{\text{индекс } b_i\text{ в отсортированном массиве } b})}^{\text{индекс элемента в массиве } a \text{, что после сортировки попал бы на позицию } i}\]
тут $sort_a^{-1}$ --- обратная перестановка. \xqed

\task{про кучу}

\end{document}