\documentclass[12pt, a4paper]{article}
\usepackage[utf8]{inputenc}
\usepackage[russian]{babel}
\usepackage{pscyr}

\usepackage{xifthen}
\usepackage{parskip}
\usepackage{hyperref}
\usepackage[top=0.7in, bottom=1in, left=0.6in, right=0.6in]{geometry}
\usepackage{setspace}

\usepackage{amsmath}
\usepackage{MnSymbol}
\usepackage{amsthm}
\usepackage{mathtools}

\usepackage{algorithm}
\usepackage[noend]{algpseudocode}



\linespread{1.2}
\setlength{\parskip}{0pt}

\renewcommand\familydefault{\sfdefault}


% Stuff related to homework specific documents
\newcounter{MyTaskCounter}
\newcounter{MyTaskSectionCounter}
\newcommand{\tasksection}[1]{
	\stepcounter{MyTaskSectionCounter}
	\setcounter{MyTaskCounter}{0}
	\ifthenelse{\equal{#1}{}}{}{
	{\hfill\\[0.2in] \Large \textbf{\theMyTaskSectionCounter \enspace #1} \hfill\\[0.1in]}}
}

\newcommand{\task}[1]{
	\stepcounter{MyTaskCounter}
	\hfill\\[0.1in]
	\ifthenelse{\equal{\theMyTaskSectionCounter}{0}}{
	   \textbf{\large Задача №\theMyTaskCounter}
	}{
	   \textbf{\large Задача №\theMyTaskSectionCounter.\theMyTaskCounter}
	}
	\ifthenelse{\equal{#1}{}}{}{{\normalsize (#1)}}
	\hfill\\[0.05in]
}

% Math and algorithms

\makeatletter
\renewcommand{\ALG@name}{Алгоритм}
\renewcommand{\listalgorithmname}{Список алгроитмов}

\newenvironment{procedure}[1]
  {\renewcommand*{\ALG@name}{Процедура}
  \algorithm\renewcommand{\thealgorithm}{\thechapter.\arabic{algorithm} #1}}
  {\endalgorithm}

\makeatother

\algrenewcommand\algorithmicrequire{\textbf{Вход:}}
\algrenewcommand\algorithmicensure{\textbf{Выход:}}
\algnewcommand\True{\textbf{true}\space}
\algnewcommand\False{\textbf{false}\space}
\algnewcommand\And{\textbf{and}\space}

\newcommand{\xfor}[3]{#1 \textbf{from} #2 \textbf{to} #3}
\newcommand{\xassign}[2]{\State #1 $\leftarrow$ #2}
\newcommand{\xstate}[1]{\State #1}
\newcommand{\xreturn}[1]{\xstate{\textbf{return} #1}}

\DeclarePairedDelimiter\ceil{\lceil}{\rceil}
\DeclarePairedDelimiter\floor{\lfloor}{\rfloor}

\newcommand{\bigO}[1]{\mathcal{O}\left(#1\right)}

\title{Алгоритмы. Домашнее задание №2}
\author{Горбунов Егор Алексеевич}

\begin{document}
\maketitle
\task{Про вычёркивание $k$ цифр из числа длины $n$ за $\bigO{n}$}
$num[0..n-1]$ --- входное число длины $n$. Рассмотрим $n[0..k]$ --- первые $k+1$ цифру этого числа. Пусть $i_m$ --- это
индекс первого вхождения $max(n[0..k])$ в $n[0..k]$. Нам выгодно жадно вычеркнуть все цифры до $i_m$-ой, благо их не больше
$k$, т.к. $0 \leq i_m \leq k$. Выгодно нам это сделать потому, что положим, первая цифра, которую мы оставим в числе --- это 
цифра с индексом $j < i_m$, но по определению $i_m$: $num[j]<num[i_m]$, но числа с большей цифрой в старшем разряде всегда
больше чисел \textbf{той же длины} с меньшей цифрой в старшем разряде. Первой оставить цифру с индексом $>k$ мы не можем,
т.к. придётся вычеркнуть более $k$ цифр, что не позволительно по условию. Значит нам действительно выгодней всего 
вычеркнуть все цифры до $i_m$-ой! Таким образом мы вычеркнем $m$ цифр и нам останется вычеркнуть $k-m$ 
в числе длины $n-m-1$, что решается аналогично =).

\begin{algorithmic}[1]
\Require{
\begin{tabular}{p{0.9\textwidth}}
$num$ --- число длины $n$ \\
$k$ --- сколько цифр нужно вычеркнуть, $k <= n$
\end{tabular}
}
\xassign{$q$}{$uber\_queue()$}
\xassign{$from$}{$0$}
\xassign{$to$}{$k$}
\xstate{$start:$}
\For{\xfor{$i$}{$from$}{$to$}}
	\xstate{$q.push(num[i])$}
\EndFor
\xassign{$max$}{$q.max()$}
\While{$to-from \geq 0$}
	\xassign{$from$}{$from + 1$}
	\xassign{$x$}{$q.pop()$}
	\If{$x != max$}
		\xstate{$to\_cross.add(x)$}
	\Else
		\xassign{$to$}{$to + 1$}
		\xstate{$break;$}
	\EndIf
\EndWhile
\If{$to\_cross.size() < k$}
	\xstate{\textbf{goto} $start$}
\EndIf
\end{algorithmic}

Приведённый алгоритм будет работать за $\bigO{n}$, если очередь $q$ реализовать на $2$ стеках с поддержкой опреации
максимума за амортизированную константу. Ни один элемент массива $num$ не может побывать дважды в очереди $q$. $num$
обходится последовательно слева направо.

\task{Совершенное паросочетание в дереве за $\bigO{n}$}
Дано дерево $T$. Известно, что в любом дереве найдётся хотя бы $2$ вершины степени $1$. 
\begin{algorithmic}[1]
\Require{
\begin{tabular}{p{0.9\textwidth}}
$T$ --- дерево \\
\end{tabular}
}
\Ensure{$PM$ --- множество рёбер, входящий в совершенное паросочетание (если его нет, то $\emptyset$)}
\xassign{$PM$}{$\emptyset$}
\xassign{$S$}{$\emptyset$}
\xassign{$mcnt$}{$0$}
\For{$v \in V(T)$} 
	\If{$d(v) = 1$}
		\xstate{$S.add(v)$}
	\EndIf
\EndFor \label{pm:initend}
\While{$S \neq \emptyset$}
	\xassign{$v$}{$S.pop()$} \label{pm:delv}
	\xassign{$u$}{$\Gamma(v).first()$}
	\xstate{$T.delete\_edge((v, u))$}
	\xstate{$PM.add((v, u))$}
	\For{$v \in \Gamma(u)$} \label{pm:ssearch}
		\xstate{$T.delete\_edge((v, u))$}
		\If{$d(v) = 1$}
			\xstate{$S.add(v)$}
		\EndIf
	\EndFor					\label{pm:esearch}
	\xassign{$mcnt$}{$mcnt+2$}
\EndWhile
\If{$mcnt < |V(T)|$}
	\xassign{$PM$}{$\emptyset$}
\EndIf
\xreturn{$PM$}
\end{algorithmic}
$\Gamma(v)$ --- список вершин, смежных с $v$.\\
Если $d(v) = 1$, то из $v$ выходит единственное ребро $e \in E(T)$ и оно \textbf{обязано входить в совершенное 
паросочетание},
т.к. это единственная возможность покрыть $v$. После выполнения строки, в силу свойств дерева, \ref{pm:initend} в $S$ 
будет как минимум $2$ вершины. В строке \ref{pm:delv} происходит удаление висячей вершины $v$ из $s$, а на следующей 
строке мы удаляем единственное ребро $(v, u)$, выходящее из $v$. Теперь мы можем выкинуть их $T$ вершины $v$ и $u$ т.к.
они уже покрыты. Важно: после выкидывания этих вершин $T\backslash\lbrace v,u \rbrace$ останется деревом, а значит в нём
так же будет как минимум $2$ висячие вершины, некоторые из них уже могут лежать в $S$, а некоторые могли появиться после
удаления вершины $u$, поэтому мы и проделываем поиск таких вершин в строках \ref{pm:ssearch}---\ref{pm:esearch}.
$S$ --- может быть стеком, т.е. операции с ним реализуются за $\bigO{1}$. $PM$ так же может быть стеком, т.е. операции
с ним тоже за $\bigO{1}$. Удаление ребра из графа может быть реализовано за $\bigO{1}$ (помечать рёбра как удалённые).
На каждой итерации цикла $While$ мы покрываем $2$ вершинки, суммарное число покрытых вершин хранится в переменной $mcnt$.
Если к моменту выхода это число не равно $|V(T)|$, то совершенного паросочетания для $T$ не существует.


Таким образом на каждой итерации цикла $While$ мы делаем $\bigO{d(u)}$ операций, тело цикла $While$ отрабатывает столько
раз, сколько вершин в $T$, а таким образом весь алгоритм отрабатывает за $\bigO{2|E(T)|}=\bigO{n}$ (по лемме о 
рукопожатиях), где $n$ --- число вершин (в дереве $|E(T)| = n - 1$). \hfill $\blacksquare$
\task{-}
\task{Запросы о сумме на отрезке за $\bigO{1}$}
Делаем следующий предподсчёт, очевидно, работающий за линейное время от размера входного массива:
\begin{algorithm}
\begin{algorithmic}[1]
\caption{Предподсчёт}
\Require{
\begin{tabular}{p{0.9\textwidth}}
$A$ --- массив из $n$ целых чисел \\
\end{tabular}
}
\Ensure{$P$ --- массив из $n + 1$ целых чисел, что $P[0] = 0,\ P[k]=\sum_{i=0}^{k-1}{A[i]}, k > 0$
}
\xassign{$P$}{$fill(array[0..n], 0)$}
\For{\xfor{$i$}{$0$}{$n-1$}}
	\xassign{$P[i + 1]$}{$P[i]+A[i]$}
\EndFor
\end{algorithmic}
\end{algorithm}\\
Отвечаем на запрос за $\bigO{1}$ исходя из очевидного равенства: $\sum_{i = l}^{r}{A[i]} = \sum_{i=0}^{r}{A[i]}-\sum_{i=0}^{l-1}{A[i]}$
\begin{algorithm}
\begin{algorithmic}[1]
\caption{Ответ на запрос}
\Require{
\begin{tabular}{p{0.9\textwidth}}
$l, r$ --- натуральные, такие, что $0 \leq l \leq r \leq n-1$ \\
\end{tabular}
}
\Ensure{целое число: $s=\sum_{i=l}^{r}{A[i]}$
}
\xreturn{$P[r+1]-P[l]$}
\end{algorithmic}
\end{algorithm}

\task{Разложение числа в сумму за $\bigO{n}$}
\begin{algorithmic}[1]
\Require{
\begin{tabular}{p{0.9\textwidth}}
$s, A[0..n-1]$ --- целое положительное число и массив из целых положительных чисел \\
\end{tabular}
}
\Ensure{$l, r$ такие, что $\sum_{i = l}^{r}{A[i]} = s$; или $l = r = -1$, если сумму составить невозможно}

\xassign{l}{0}
\xassign{r}{0}
\xassign{sum}{0}
\While{$r < n$ AND $l < n$ AND $s != sum$}
	\If{$sum < s$}
		\xassign{$sum$}{$sum+A[r]$}
		\xassign{$r$}{$r + 1$};
	\ElsIf{$sum > s$}
		\xassign{$sum$}{$sum-A[l]$}
		\xassign{$l$}{$l + 1$}
	\EndIf
\EndWhile
\xassign{$r$}{$r+1$}
\end{algorithmic}
Этот алгоритм работает за $\bigO{n}$ т.к. $l \leq r$ на протяжении всей его работы, т.к. перед главным циклом это верно,
а далее $l$ увеличивается лишь тогда, когда $sum > s$, а это верно лишь тогда, когда $l < r$, т.к. $sum$ всегда хранит
сумму чисел массива с индексами в полуинтервале $[l, r)$.


Пускай для некоторого числа $s$ наименьший возможный по левой границе ответ на задачу --- полуинтервал $[a, b)$, 
алгоритм подбирает ответ, начиная с $[0, 0)$. На каждой итерации двигается \textbf{на 1} либо левая граница, либо правая.
Тогда рано или поздно либо $l$ дойдёт до $a$, либо $r$ дойдёт до $b$. Пусть $l$ дошёл до $a$ раньше, чем $r$ до $b$, тогда
$sum < s$, т.к. $l \leq r \leq b$ и числа в массиве положительны. Значит, исходя из алгоритма, далее мы будем двигать правую
границу до тех пор, пока $r$ не станет равным $b$ и условие внешнего цикла сломается (наконец-то)! Пускай $r$ раньше дошёл
до $b$, чем $l$ дошёл до $a$, т.е. $l < a < b = r$. Но тогда $sum > s$ в силу натуральности чисел массива. А значит, аналогично,
мы будем двигать границу $l$ вперёд до тех пор, пока $l$ не станет равным $a$ и условие цикла снова сломается. Значит
алгоритм корректен. \hfill $\blacksquare$ 
\task{Максимальная сумма в массиве над $\mathcal{Z}$ за $\bigO{n}$}

\begin{algorithmic}[1]
\Require{
\begin{tabular}{p{0.9\textwidth}}
$A$ --- массив из $n$ целых чисел \\
\end{tabular}
}
\Ensure{$l, r$ --- такие индексы массива, что: $\sum_{i=l}^{r}{A[i]}$ максимальна
}
\xassign{$P$}{$fill(array[0..n], 0)$}
\For{\xfor{$i$}{$0$}{$n-1$}}
	\xassign{$P[i + 1]$}{$P[i]+A[i]$}
\EndFor
\xassign{$(l, r)$}{$min\_max\_ind(P)$}
\xassign{$(l', r')$}{$max\_min\_ind(P)$}
\If{$P[r]-P[l] > P[r']-P[l']$}
	\xreturn{$(l+1, r)$}
\Else
	\xreturn{$(l'+1, r')$}
\EndIf
\end{algorithmic}
$min\_max\_ind(P)$ --- возвращает пару $(l, r)$, что $l$ --- индекс минимального элемента в массиве $P$, а $r$ --- индекс
максимального элемента в $P$, причём $r > l$ ($r$ ищется после $l$).\\
$max\_min\_ind(P)$ --- возвращает пару $(l, r)$, что $r$ --- индекс максимального элемента в массиве $P$, а $l$ --- индекс
минимального элемента в $P$, причём $l < r$ ($l$ ищется после $r$).\\


Функции $min\_max\_ind(P)$ и $max\_min\_ind(P)$ очевидно могут быть реализованы за $\bigO{n}$ операций. Подсчёт сумм на
префиксах так же занимает $\bigO{n}$. В конце мы получаем такую пару $(l, r)$, что $\sum_{i=0}^{l-1}{A[i]}$ минимальна,
а $\sum_{i=0}^{r}{A[i]}$ максимальна, причём из всех возможных $l \leq r$. Но это значит, что $\sum_{i=l}^{r}{A[i]}$
максимальна! \hfill $\blacksquare$
\end{document}