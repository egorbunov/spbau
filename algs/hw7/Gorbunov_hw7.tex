\documentclass[12pt, a4paper]{article}
\usepackage[utf8]{inputenc}
\usepackage[russian]{babel}
\usepackage{pscyr}
\usepackage{amssymb}
\usepackage{xifthen}
\usepackage{parskip}
\usepackage{hyperref}
\usepackage{setspace}

\usepackage{graphicx}
\usepackage{xcolor}
\usepackage{amsmath}
\usepackage{MnSymbol}
\usepackage{amsthm}
\usepackage{mathtools}
\usepackage{algorithm}
\usepackage[noend]{algpseudocode}
\usepackage[shortlabels]{enumitem}
                    \setlist[enumerate, 1]{1\textsuperscript{o}}
\usepackage{subfig}
\usepackage{tikz}
\usepackage{tikz,fullpage}
\usetikzlibrary{shapes,snakes}
\usetikzlibrary{arrows,%
                petri,%
                topaths}%
\usepackage{tkz-berge}
\usepackage[top=0.5in, bottom=0.7in, left=0.6in, right=0.6in]{geometry}
\linespread{1.3}

% \renewcommand\familydefault{\sfdefault}


% Stuff related to homework specific documents
\newcounter{MyTaskCounter}
\newcounter{MyTaskSectionCounter}
\newcommand{\tasksection}[1]{
	\stepcounter{MyTaskSectionCounter}
	\setcounter{MyTaskCounter}{0}
	\ifthenelse{\equal{#1}{}}{}{
	{\hfill\\[0.2in] \Large \textbf{\theMyTaskSectionCounter \enspace #1} \hfill\\[0.1in]}}
}

\newcommand{\task}[1]{
	\stepcounter{MyTaskCounter}
	\hfill\\[0.1in]
	\ifthenelse{\equal{\theMyTaskSectionCounter}{0}}{
	   \textbf{\large Задача №\theMyTaskCounter}
	}{
	   \textbf{\large Задача №\theMyTaskSectionCounter.\theMyTaskCounter}
	}
	\ifthenelse{\equal{#1}{}}{}{{\normalsize (#1)}}
	\hfill\\[0.05in]
}

% Math and algorithms

\makeatletter
\renewcommand{\ALG@name}{Алгоритм}
\renewcommand{\listalgorithmname}{Список алгроитмов}

\newenvironment{procedure}[1]
  {\renewcommand*{\ALG@name}{Процедура}
  \algorithm\renewcommand{\thealgorithm}{\thechapter.\arabic{algorithm} #1}}
  {\endalgorithm}

\makeatother

\algrenewcommand\algorithmicrequire{\textbf{Вход:}}
\algrenewcommand\algorithmicensure{\textbf{Выход:}}
\algnewcommand\True{\textbf{true}\space}
\algnewcommand\False{\textbf{false}\space}
\algnewcommand\And{\textbf{and}\space}

\newcommand{\xfor}[3]{#1 \textbf{from} #2 \textbf{to} #3}
\newcommand{\xassign}[2]{\State #1 $\leftarrow$ #2}
\newcommand{\xstate}[1]{\State #1}
\newcommand{\xreturn}[1]{\xstate{\textbf{return} #1}}

\DeclarePairedDelimiter\ceil{\lceil}{\rceil}
\DeclarePairedDelimiter\floor{\lfloor}{\rfloor}

\newcommand{\bigO}[1]{\mathcal{O}\left(#1\right)}

\newcommand{\xqed}{\hfill $\blacksquare$}

\newcommand{\code}[1]{\colorbox{gray!15}{\footnotesize\texttt{#1}}}
\title{Алгоритмы. Домашнее задание №7}
\author{Горбунов Егор Алексеевич}

\begin{document}
\maketitle

\task{Сильная ориентация графа}
\textit{Нужно ориентировать рёбра данного неориентированного графа $G(V,E)$ за $\bigO{|V|+|E|}$ так, чтобы получившийся граф был сильно связным.}\\
Из курса по дискретной математике нам известно, что неориентированный граф $G$ может быть сильно ориентирован тогда и
только тогда, когда он двусвязный, т.е. в $G$ нет точек сочленения. Искать точки сочленения в графе $G$ мы умеем за 
$\bigO{|V|+|E|}$ и первым делом запустим алгоритм поиска точек сочленения и в случае, если точки сочленения будут найдены, то сообщим о том, что данный граф $G$ не допускает сильной ориентации.\\
Далее будем считать, что граф $G$ не содержит точек сочленения, т.е. двусвязен, а значит допускает сильную ориентицаю
рёбре. Построим такую ориентацию рёбер:

\begin{algorithmic}
\Procedure{DFS}{$v$}
	\xstate{isUsed$[v]=true$}
	\For{$u \in \Gamma{(v)}$}
		\xstate{\Call{orientEdgeFromTo}{$v,u$}}
		\If{isUsed$[u]=false$}
			\xstate{\Call{DFS}{$u$}}
		\EndIf
	\EndFor
\EndProcedure
\end{algorithmic}
Вызов <<orientEdgeFromTo($v,u$)>> просто ориентирует ребро $\lbrace v, u \rbrace$ так: $(v,u)$, т.е. $v\rightarrow~u$.
Таким образом мы просто ориентируем все рёбра в порядке обхода в глубину от предка к сыну, а если встречаем обратное ребро, которое может вести из вершины $v$ к её предку, то ориентируем его от сына ($v$) к предку.\\
Этот алгоритм будет работать по следующим причинам: рассмотрим 2 любые вершины $v$ и $u$ графа $G'$, который 
есть ориентированный граф $G$ по процедуре, приведённой выше. Посмотрим на эти вершины в дереве, которое было построено поиском в глубину. Пускай $p$ --- это любой общий предок вершины $v$ и $u$ в этом дереве (он  может совпадать с $v$ или $u$).
Заметим теперь, что т.к. $G$ --- двусвязный, то нам известно (из того же курса дискретной математики), что \textbf{любые $2$ вершины $G$ лежат на одном цикле}, но 
это значит, что $p$ и $v$ лежат на некотором цикле. Ясно тогда, что найдётся такая вершина $x$, что в дереве, построенном обходом в глубину, $v$ будет предком $x$ ($x$ может совпадать с $v$) и из $x$ будет торчать обратное ребро
в $p$:

\begin{figure}[ht!]
\centering
\begin{tikzpicture}
\GraphInit[vstyle=Classic]
\tikzset{VertexStyle/.append style = {minimum size = 3pt}}
\tikzstyle{EdgeStyle}=[post]
\Vertex[Math,x=-0.5,y=-1.25]{v}
\Vertex[Math,x=2,y=-2.5]{u}
\Vertex[Math,x=0,y=0]{p}
\Vertex[Math,x=-1.5,y=-2]{x}
\Edge[style=dashed](p)(v)
\Edge[style=dashed](p)(u)
\Edge[style=dashed](v)(x)
\tikzset{EdgeStyle/.append style = {bend left}}
\Edge[color=red,style=dashed](x)(p)
\end{tikzpicture}
\end{figure}
Но тогда мы видим, что из $v$ можно пройти в $u$ так: $v \rightarrow \ldots \rightarrow x \rightarrow p \rightarrow \ldots \rightarrow u$. Таким образом мы показали, что для любых двух вершин $v$ и $u$ графа $G'$, который есть ориентация графа $G$ по приведённой процедуре, из $v$ существует путь по ориентированным рёбрам в $u$. А это значит, 
что граф $G'$ сильно ориентирован. \xqed

\task{xxx}
\end{document}