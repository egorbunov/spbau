\documentclass[12pt, a4paper]{article}
\usepackage[utf8]{inputenc}
\usepackage[russian]{babel}
\usepackage{pscyr}
\usepackage{amssymb}
\usepackage{xifthen}
\usepackage{parskip}
\usepackage{hyperref}
\usepackage{setspace}

\usepackage{graphicx}
\usepackage{xcolor}
\usepackage{amsmath}
\usepackage{MnSymbol}
\usepackage{amsthm}
\usepackage{mathtools}
\usepackage{algorithm}
\usepackage[noend]{algpseudocode}
\usepackage[shortlabels]{enumitem}
                    \setlist[enumerate, 1]{1\textsuperscript{o}}
\usepackage{subfig}
\usepackage{tikz}
\usepackage{tikz,fullpage}
\usetikzlibrary{shapes,snakes}
\usetikzlibrary{arrows,%
                petri,%
                topaths}%
\usepackage{tkz-berge}
\usepackage[top=0.5in, bottom=0.7in, left=0.6in, right=0.6in]{geometry}
\linespread{1.3}

% \renewcommand\familydefault{\sfdefault}


% Stuff related to homework specific documents
\newcounter{MyTaskCounter}
\newcounter{MyTaskSectionCounter}
\newcommand{\tasksection}[1]{
	\stepcounter{MyTaskSectionCounter}
	\setcounter{MyTaskCounter}{0}
	\ifthenelse{\equal{#1}{}}{}{
	{\hfill\\[0.2in] \Large \textbf{\theMyTaskSectionCounter \enspace #1} \hfill\\[0.1in]}}
}

\newcommand{\task}[1]{
	\stepcounter{MyTaskCounter}
	\hfill\\[0.1in]
	\ifthenelse{\equal{\theMyTaskSectionCounter}{0}}{
	   \textbf{\large Задача №\theMyTaskCounter}
	}{
	   \textbf{\large Задача №\theMyTaskSectionCounter.\theMyTaskCounter}
	}
	\ifthenelse{\equal{#1}{}}{}{{\normalsize (#1)}}
	\hfill\\[0.05in]
}

% Math and algorithms

\makeatletter
\renewcommand{\ALG@name}{Алгоритм}
\renewcommand{\listalgorithmname}{Список алгроитмов}

\newenvironment{procedure}[1]
  {\renewcommand*{\ALG@name}{Процедура}
  \algorithm\renewcommand{\thealgorithm}{\thechapter.\arabic{algorithm} #1}}
  {\endalgorithm}

\makeatother

\algrenewcommand\algorithmicrequire{\textbf{Вход:}}
\algrenewcommand\algorithmicensure{\textbf{Выход:}}
\algnewcommand\True{\textbf{true}\space}
\algnewcommand\False{\textbf{false}\space}
\algnewcommand\And{\textbf{and}\space}

\newcommand{\xfor}[3]{#1 \textbf{from} #2 \textbf{to} #3}
\newcommand{\xassign}[2]{\State #1 $\leftarrow$ #2}
\newcommand{\xstate}[1]{\State #1}
\newcommand{\xreturn}[1]{\xstate{\textbf{return} #1}}

\DeclarePairedDelimiter\ceil{\lceil}{\rceil}
\DeclarePairedDelimiter\floor{\lfloor}{\rfloor}

\newcommand{\bigO}[1]{\mathcal{O}\left(#1\right)}

\newcommand{\xqed}{\hfill $\blacksquare$}

\newcommand{\code}[1]{\colorbox{gray!15}{\footnotesize\texttt{#1}}}
\title{Алгоритмы. Домашнее задание №6}
\author{Горбунов Егор Алексеевич}

\begin{document}
\maketitle

\task{Число строк свободных от $s$}
\task{Взвешенный бинарный поиск}
Нужно определить стоимость поиска элемента в массиве $A[1..n]$ в худшем случае, даны стоимости обращения к элементу $i$: $c_i$. Введём:
\[
	D[i][j] \text{ --- стоимости поиска в худшем случае в подмассиве A[i..j]}
\]
Ясно, что $D[i][i]=c_i$. При поиске в $A[i..j]$ мы можем выбрать разделяющим любой элемент из $i..j$, т.е. подсчёт $D[i][j]$ таков:
\[
D[i][j] = \max_{i \leq k \leq j}{(\max{(D[i][k-1], D[k][j]) + c_k)}}
\]
Решение задачи после подсчёта находится в $D[1][n]$, для того, чтобы его найти нужно заполнить таблицу $D$ и для
каждой её клетки затратить $\bigO{n}$ операций. Отсюда получаем сложность $\bigO{n^3}$. Заполнение $D$ идёт по диагоналям, начиная с главной. \xqed
\task{Число способов расставить на доске...}
\task{Расстояние до палиндрома}
Дана строчка $s[1..n]$, нужно превратить строчку $s$ в палиндром за минимальную суммарную стоимость операций,
если возможны операции: удаление символа стоимостью в $A$ и замена символа стоимостью в $B$.\\
Введём следующую величину:
\begin{center}
\vspace{3mm}
\begin{tabular}[ht!]{lcp{13cm}}
	$D[i][j]$ & --- & наименьшая суммарная стоимость операций по приведению подстроки s[i..j] к палиндрому\\
\end{tabular}
\vspace{3mm}
\end{center}
$D[i][i] = 0$, т.к. $1$ символ --- палиндром. $D[i][j] = 0, j<i$.\\
\begin{enumerate}
\item Если в оптимальном приведении $s[1..n]$ к палиндрому первый и последний символы $s$ остались нетронутыми, 
т.е. $s[1]=s[n]$, то стоимость оптимального приведения $s[1..n]$ к палиндрому равно стоимости оптимального 
приведения к палиндрому $s[2..n-1]$.
\item Если в оптимальном приведении $s[1..n]$ к палиндрому $s[1]$ и $s[n]$ были удалены, то оптимальная стоимость 
приведения к палиндрому $s[1..n]$ равна $D[2][n-1]+2A$ (дважды удалили)
\item ... $s[1]$ был заменён на $s[n]$ (или наоборот), то $D[1][n] = D[2][n-1]+B$ (ясно, что $s[1]\neq s[n]$ в данной ситуации)
\item ... $s[1]$ был удалён (а $s[n]$ нет), то $D[1][n]=D[2][n]+A$ и аналогично, если $s[n]$ был удалён, а $s[1]$ нет.
\end{enumerate}
Других случаев быть не может и таким образом получили следующее решение методом динамического программирования:
\[
D[i][j] = \min
\begin{cases}
D[i+1][j-1] &, s[i]=s[j] \\
D[i+1][j-1]+B &, s[i]\neq s[j]\\
D[i+1][j-1]+2A\\
D[i+1][j]+A \\
D[i][j-1]+A
\end{cases}
\]
Чтобы вычислить ячейку $D[i][j]$ нужно знать значения в ячейках левее и ниже. Таким образом можно просчитывать $D$ 
по диагоналям. Ответ будет записан в правой верхней ячейке $D[1][n]$ и найдём его за время $\bigO{n^2}$. \xqed
\task{-}
\task{Про палиндромы...}
\begin{enumerate}[label=(\alph*)]
	\item Найти самую длинную подпоследовательность-палиндром за $\bigO{n^2}$\\
	Мы умеем за $\bigO{n}$ превращать удалениями символов и заменой символов строку $s[1..n]$ в палиндром (это
	задача №4). Заметим тогда, что если положить стоимость удаления символа $A=1$, а $B=(n+1)$ --- стоимость замены,то
	алгоритм построение преобразования строки $s$ в палиндром всегда будет пользоваться лишь операцией удаления, а т.к.
	он находит такое редактирование, что суммарная стоимость операций минимальна, т.е. число удалений минимально, то
	мы получим в результате наибольшую подпоследовательность-палиндром. \xqed
	\item Посчитать число подстрок палиндромов за $\bigO{n^2}$\\
	Будем идти по строке $s[1..n]$ слева направо. На каждой итерации рассматриваем $i$-ый элемент строки и делаем
	следующее:\\
	1) пытаемся построить палиндром нечётной длины с центром в $s[i]$:
	\begin{algorithmic}
	\xassign{$l$}{$i-1$}
	\xassign{$r$}{$i+1$}
	\While{$l \geq 1$ AND $r \leq n$ AND $s[l]=s[r]$ }
		\xstate{$l--;\ r++;$}
		\xstate{$polindromeCnt\ += 1$}
	\EndWhile
	\end{algorithmic}
	2) пытаемся построить палиндром чётной длины с центром <<между>> $s[i-1]$ и $s[i]$ ($i > 1$), псевдокод аналогичен
	тому, что приведён выше.\\
	Ясно, что т.к. мы перебираем центры палиндромов и все они различны, то ни один палиндром не будет посчитан дважды,
	а так же все палиндромы будут посчитаны, сложность $\bigO{n^2}$
	\item Разбить строку на минимальное число палиндромов за $\bigO{n^2}$\\
	За квадратичное время от длины строки, по пункту $(b)$ этой задачи мы можем найти все палиндромы. Их, очевидно,
	уж точно не более $n^2$. Тогда мы можем за квадратичное время построить массив $P[1..n]$, что в $P[i]$ хранятся
	все палиндромы, которые начинаются на позиции $i$ строки $s$. Будем считать, что мы за $\bigO{1}$ можем находить длину палиндрома (строки). Тогда ясно, что в любом разбиении строки $s$ на палиндромы (в том числе и минимальном),
	какой-то палиндром стоит в начале с позиции $1$ по какую-то позицию $k$. Будем находить тогда величину $D[i]$
	--- минимальное число палиндромов в разбиении $s[i..n]$. $D[n] = 1$.
	\[
		D[i] = \min_{s[i..k] \in P[i]}{\left( 1 + D[k + 1] \right)}
	\] 
	Ответ будет находиться в ячейке $D[1]$ и найден он будет за $\bigO{n^2}$ (включая нахождение всех палиндромов вначале). \xqed
\end{enumerate}
\end{document}