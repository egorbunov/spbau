\documentclass[12pt, a4paper]{article}
\usepackage[utf8]{inputenc}
\usepackage[russian]{babel}
\usepackage{pscyr}

\usepackage{xifthen}
\usepackage{parskip}
\usepackage{hyperref}
\usepackage[top=0.7in, bottom=1in, left=0.6in, right=0.6in]{geometry}
\usepackage{setspace}

\usepackage{amsmath}
\usepackage{MnSymbol}
\usepackage{amsthm}
\usepackage{mathtools}

\usepackage{algorithm}
\usepackage[noend]{algpseudocode}



\linespread{1.2}
\setlength{\parskip}{0pt}

\renewcommand\familydefault{\sfdefault}


% Stuff related to homework specific documents
\newcounter{MyTaskCounter}
\newcounter{MyTaskSectionCounter}
\newcommand{\tasksection}[1]{
	\stepcounter{MyTaskSectionCounter}
	\setcounter{MyTaskCounter}{0}
	\ifthenelse{\equal{#1}{}}{}{
	{\hfill\\[0.2in] \Large \textbf{\theMyTaskSectionCounter \enspace #1} \hfill\\[0.1in]}}
}

\newcommand{\task}[1]{
	\stepcounter{MyTaskCounter}
	\hfill\\[0.1in]
	\ifthenelse{\equal{\theMyTaskSectionCounter}{0}}{
	   \textbf{\large Задача №\theMyTaskCounter}
	}{
	   \textbf{\large Задача №\theMyTaskSectionCounter.\theMyTaskCounter}
	}
	\ifthenelse{\equal{#1}{}}{}{{\normalsize (#1)}}
	\hfill\\[0.05in]
}

% Math and algorithms

\makeatletter
\renewcommand{\ALG@name}{Алгоритм}
\renewcommand{\listalgorithmname}{Список алгроитмов}

\newenvironment{procedure}[1]
  {\renewcommand*{\ALG@name}{Процедура}
  \algorithm\renewcommand{\thealgorithm}{\thechapter.\arabic{algorithm} #1}}
  {\endalgorithm}

\makeatother

\algrenewcommand\algorithmicrequire{\textbf{Вход:}}
\algrenewcommand\algorithmicensure{\textbf{Выход:}}
\algnewcommand\True{\textbf{true}\space}
\algnewcommand\False{\textbf{false}\space}
\algnewcommand\And{\textbf{and}\space}

\newcommand{\xfor}[3]{#1 \textbf{from} #2 \textbf{to} #3}
\newcommand{\xassign}[2]{\State #1 $\leftarrow$ #2}
\newcommand{\xstate}[1]{\State #1}
\newcommand{\xreturn}[1]{\xstate{\textbf{return} #1}}

\DeclarePairedDelimiter\ceil{\lceil}{\rceil}
\DeclarePairedDelimiter\floor{\lfloor}{\rfloor}

\newcommand{\bigO}[1]{\mathcal{O}\left(#1\right)}

\title{Алгоритмы. Домашнее задание №4}
\author{Горбунов Егор Алексеевич}

\begin{document}
\maketitle

\task{унимодальный массив}
\begin{enumerate}[label={\textbf{(\alph*)}}]

\item \textit{Поиск пика за $\bigO{\log{n}}$ в унимодальном массиве $A[1..n]$}\\
Рассмотрим элементы $a = A[\frac{n}{2}]$ и $b = A[\frac{n}{2}+1]$. Если $a < b$, то
пик находится где-то справа от $b$ или в $b$, т.к. мы попали на <<подъём>>, иначе, если $a > b$, то
мы попали на <<спуск>> и пик точно находится левее $a$ или в $a$. По разные стороны от пика 2 соседних 
элемента $a$ и $b$ находиться не могут. Таким образом можно продолжить поиск пика в одной их выделенных частей
массива, размер которой $=\frac{n}{2}$. Поиск завершится тогда, когда будут выбраны последние возможные 2 элемента и из них будет выбран наибольший. Время работы такого поиска: $T(n)=T(\frac{n}{2})+\bigO{1}=\bigO{\log{n}}$\\
\textit{PS:} вообще говоря можно выбирать любые 2 элемента, меньшие $n$ в разы, и сравнивать их. Например 
$A[\frac{n}{3}]$ и $A[\frac{2n}{3}]$. Ясно, что если первый меньше второго, то пика уже точно нет левее $\frac{n}{3}$, иначе, если второй больше первого, то правее $\frac{2n}{3}$ пика уже точно нет. Таким образом размер задачи уменьшается на треть: $T(n)=T(\frac{2n}{3})+\bigO{1}=\bigO{\log{n}}$

\item \textit{Найти минимальный bounding box у выпуклого многоугольника за $\bigO{\log{n}}$}\\
Нам дан массив вершин $V[1..n]$ многоугольника в порядке обхода по часовой стрелке, причём никакие 3 
подряд идущие вершины не лежат на одной прямой. Ясно, что чтобы найти минимальный ограничивающий прямоугольник, нужно найти:
\[ x_{min}=min(V[i].x),\ x_{max}=max(V[i].x),\ y_{min}=min(V[i].y),\ y_{max}=max(V[i].y) \]
Тогда ответом на задачу будет прямоугольник, построенный на вершинах $(x_{min}, y_{min})$, 
$(x_{max},y_{max})$,
которые задают его диагональ (стороны прямоугольника параллельны осям координат).\\
Посмотрим на данный нам обход многоугольника по часовой стрелке, но будем рассматривать только
координату $x$ каждой точки обхода. Ясно, что по мере обхода, эта координата сначала увеличивается (уменьшается) до самой правой (самой левой) по оси абсцисс точки, а потом начинает уменьшаться (увеличиваться)
до самой левой (правой) по оси абсцисс точки, ну а далее опять может начать увеличиваться (уменьшаться), пока обход не приведёт нас к последней точке обхода. Для удобства добавим замкнём обход, поставив в конец первую точку. Удобно представлять обход, как показано на рисунку ниже: по оси ординат отложено значение $v.x$, а по 
оси абсцисс (влево) идут индексы точки обхода.
\begin{figure}[ht!]
\centering
\begin{tikzpicture}[yscale=0.3]
\draw[gray,very thin] (-1,0) -- (7,0);
\draw (0,0) -- (1,1) -- (2,3) -- (3,4) -- (4,2) -- (5,1) -- (6,0);
\end{tikzpicture}
\begin{tikzpicture}[yscale=0.3]
\draw[gray,very thin] (-1,0) -- (7,0);
\draw (0,0) -- (1,1) -- (2,3) -- (3,-1) -- (4,-3) -- (5,-2) -- (6,0);
\end{tikzpicture}
\caption{удобное представление массива обхода многоугольника по координате}
\end{figure}\\
Наша задача тогда --- найти моды на в таком массиве обхода. В обходе у первой и последней вершины
совпадают координаты (т.к. вершины одинаковы).\\
Делаем так: рассмотрим $3$ точки ($a=V[\frac{n}{2}],b=V[\frac{n}{2}-1],c=V[\frac{n}{2}+1]$) посередине массива обхода. В силу того, что никакие $3$ не лежат на одной прямой найдётся такая точка, что её 
$x$-координата отлична от $x$-координаты начальной (конечной) точки обхода. Пусть это точка $b$.
Если $b.x>V[1].x=V[n].x$. Пик может находиться справа от $b$ или слева от $b$. В свою очередь точка
$b$ находится на <<склоне>>, а значит, если от точки $b$ мы пойдём в сторону <<подъёма>>, то придём к пику
уж точно, т.к. где-то нужно будет начать спускаться опять к значению $V[1].x$. Сторону подъёма мы легко
можем определить за $\bigO{1}$ в силу того, что нет $3$ точек лежащий на одной прямой. Таким образом
мы будем делить задачу на двое и найдём в итоге $1$ пик за $\bigO{\log{n}}$. Второй пик найдётся элементарно за $\bigO{\log{n}}$, т.к. мы его поищем алгоритмом поиска пика в унимодальном массиве слева и справа от 
уже найденного пика на первом шаге.\\
Так мы найдём $x_{min}$ и $x_{max}$. Аналогично ищутся $y_{min}$ и $y_{max}$. Итоговая сложность $\bigO{\log{n}}$ \xqed

\item \textit{Принадлежность точки выпуклому многоугольнику за $\bigO{\log{n}}$}
\end{enumerate}

\task{Амортизационное время работы счётчика}
Будем обозначать счётчик: $cnt$\\
\begin{enumerate}
\item Увеличение счётчика на $1$\\
Введём функцию потенциала: $\phi{(cnt)}=\text{\textit{Число единиц в двоичной записи }} cnt$\\
Пусть длина двоичной записи $cnt$ равна $n$. И пусть в $cnt$ первые $k$ младших разрядов заполнены
единицами ($k+1$ разряд заполнен $0$, а остальные какие-то...). Ясно тогда, что число операций, которое нужно
затратить на инкремент равно $k+1$ --- обнулить первые $k$ младших разрядов и вставить в $k+1$ единицу.
Таким образом амортизированное время работы операции инкремента:
\[ T_{amortized}(n) = T_{real}(n)+\phi{(cnt+1)}-\phi{(cnt)} = k+1+\phi{(cnt)}-k+1-\phi{(cnt)}=2=\bigO{1}\]

\item Уменьшение счётчика на $1$\\
Аналогично увеличению, только потенциальная функция будет следующей:\\
\[ \phi{(cnt)}=\text{\textit{Число нулей в двоичной записи }} cnt \]

\item Сравнение счётчика с $0$\\
Т.к. на память ограничений в условии задачи я не вижу, то мы можем эффективно поддерживать индекс самой
левой единицы в двоичной записи числа $cnt$. Т.к. при операциях инкремента и декремента он как максимум
может сдвигаться на 1 влево или вправо. Таким образом для сравнения с $0$ счётчика достаточно посмотреть
на этот индекс и если он невалидный (напрмер, $-1$), То счётчик нулевой, иначе нет.
\end{enumerate}
\xqed

\task{Скошенная система счисления}
\begin{enumerate}[label={\textbf{(\alph*)}}]
\item \textit{неотрицательное целое число единственным образом записывается скошенной системе счисления}\\
Пусть это не так и $n \geq 0$ такого, что: $n = \sum_{i=1}^{l_1}{a_i(2^i-1)}$ и $n = \sum_{i=1}^{l_2}{b_i(2^i-1)}$, пусть $l_1>l_2$, тогда:
\[\sum_{i=1}^{l_2}{(a_i-b_i)(2^i-1)} + \sum_{i=l_2+1}^{l_1}{a_i(2^i-1)}=0 \]
Заметим, что $\sum_{i=l_2+1}^{l_1}{a_i(2^i-1)}\geq 2^{l_2+1}-1$ (только $a_{l_2+1} \neq 0$), а $\sum_{i=1}^{l_2}{(a_i-b_i)(2^i-1)} \leq \ \sum_{i=1}^{l_2}{(2^i-1)}=2^{l_2+1}-2-l_2$ (тут все $b_i=0$, а $a_i=1$, с поправкой на то, что лишь один элемент $a_i$ может быть = $2$ и хотя бы один $b_i$ не нулевой =) ).
Видим, что так нуля не может в сумме получиться, а значит $l_1=l_2$. Т.е. получили, что у любых 2-х
записей числа в скошенной системе счисления одинаковая длина! Докажем далее по индукции, что
такая запись единственна. База проверяется элементарно для $n=0,n=1$. Пускай верно для чисел вплоть до
$n-1$, докажем для $n$. Опять от противного: пусть есть 2 записи числа (уже одной длины):
\[ n = \sum_{i=1}^{l}{a_i(2^i-1)} = \sum_{i=1}^{l}{b_i(2^i-1)} \]
Если $a_{l}=0$, то и $b_{l}$ равно $0$, т.к. мы уже показали, что у чисел одинаковая длина.\\
Если $a_{l}=2$, то $\forall\ i\in \lbrace 1,\ldots,l_2-1\rbrace\ a_i = 0$, т.е. $n=2^{l+1}-2$. Но тогда
$b_{l}$ тоже равно $2$, т.к. самое больше число, которое можно получить не приравнивая $b_{l}$ к $2$ это:
$\sum_{i=1}^{l}{(2^i-1)}=2^{l+1}-2-l$, а это меньше $n$, что невозможно. Таким образом получили, что
$a_{l}=b_{l}$. Заметим тогда, что мы получили 2 различных записи числа $n-a_l(2^{l}-1) < n$, что противоречит
индукционному предположению. А значит, запись числа $n$ единственна. \xqed

\item \textit{Инкремент скошенного числа за $\bigO{1}$}\\
Пусть $n=\sum_{i=1}^{l}{a_i(2^i-1)}$, причём $a_k$ --- первый не равный $0$ элемент. Тогда как
найти $n+1$:

\begin{enumerate}

\item Если $a_k=1$ и $k>1$\\
\[n+1 = \sum_{i=k}^{l}{a_i(2^i-1)} + 1 = a_1'+\sum_{i=k}^{l}{a_i'(2^i-1)}\]
$a_1'=1$, а остальные $a_i'=a_i$
\item Если $a_k=1$ и $k=1$\\
\[ n+1 = a_1 + \sum_{i=2}^{l}{a_i(2^i-1)}+1=a_1'+\sum_{i=2}^{l}{a_i'(2^i-1)} \]
$a_1'=2$, а остальные $a_i'=a_i$. Свойства скошенных чисел не нарушены, т.к. это первая $2$
в записи и за нею ничего нет. (Другой двойки быть не могло, т.к. $a_k$ --- первый ненулевой знак).
\item Если $a_k=2$\\
Тогда 
\begin{equation*}
\begin{split}
n+1 &= 2(2^k-1)+a_{k+1}(2^{k+1}-1)+\sum_{i=k+2}^{l}{a_i(2^i-1)}+1 =\\
    &= 2^{k+1}-2+1+a_{k+1}(2^{k+1}-1)+\sum_{i=k+2}^{l}{a_i(2^i-1)} =\\
    &= (a_{k+1}+1)(2^{k+1}-1)+\sum_{i=k+2}^{l}{a_i(2^i-1)}
\end{split}
\end{equation*}
Получили корректную запись числа. $a_{k+1}\leq 1$ по свойству скошенного числа.
\end{enumerate}
Итого мы видим, что для операции инкремента нужно в первых двух случаях просто увеличить 
младший разряд, а в 3-ем случае (если первый ненулевой знак $= 2$) нужно обнулить бит,
где стоит двойка и увеличить следующий за ним бит.
Ясно, что это реализуемо за $\bigO{1}$, если хранить дополнительно индекс двойки в числе, 
а само число хранить в массиве. На каждый инкремент тогда мы точно знаем что менять и меняем
это за $\bigO{1}$. \xqed
\end{enumerate}

\task{-}
\task{-}

\task{Персистентный список символов с доступом, откатом и добавлением $\bigO{\log{n}}$}
Такую структуру данных можно реализовать используя любое дерево поиска, которое будет
обладать логарифмической сложностью на добавление и поиск элемента. Для каждого момента
времени $i$ храним в массиве корень дерева поиска $T_i$. В корне $T_i$ храним число элементов
в дереве $T_i.size$. Каждая вершина дерева $T_i$ представляет из себя пару $(char, index)$, причём
ключом дерева $T_i$, по которому производится поиск, является $index$ --- это и есть индекс такого
символа в массиве. Пусть вся система находится на моменте времени $i$. Добавление символа требует 
пройти по дереву поиска вглубь затратив $\bigO{\log{n}}$ операций и, соответственно, изменению подвергается
не более $\log{n}$ вершн. Те вершины, которые во время добавлению изменяются, копируются в новое дерево
$T_{i+1}$ с изменениями, а остальные вершины сохраняются и остаются, как в дереве $T_i$. Таким образом
мы затрачиваем на добавление $\bigO{\log{n}}$ операций. На вытаскивание $i$ элемента тратится столько
же, т.к. дерево поиска нам это обеспечивает, а откат производится за $\bigO{1}$, т.к. все корни хранятся
в массиве. \xqed

\end{document}