%!TEX program = xelatex


\documentclass[12pt, a4paper]{article}
\usepackage[utf8]{inputenc}
\usepackage[russian]{babel}
\usepackage{pscyr}

\usepackage{xifthen}
\usepackage{parskip}
\usepackage{hyperref}
\usepackage[top=0.7in, bottom=1in, left=0.6in, right=0.6in]{geometry}
\usepackage{setspace}

\usepackage{amsmath}
\usepackage{MnSymbol}
\usepackage{amsthm}
\usepackage{mathtools}

\usepackage{algorithm}
\usepackage[noend]{algpseudocode}



\linespread{1.2}
\setlength{\parskip}{0pt}

\renewcommand\familydefault{\sfdefault}


% Stuff related to homework specific documents
\newcounter{MyTaskCounter}
\newcounter{MyTaskSectionCounter}
\newcommand{\tasksection}[1]{
	\stepcounter{MyTaskSectionCounter}
	\setcounter{MyTaskCounter}{0}
	\ifthenelse{\equal{#1}{}}{}{
	{\hfill\\[0.2in] \Large \textbf{\theMyTaskSectionCounter \enspace #1} \hfill\\[0.1in]}}
}

\newcommand{\task}[1]{
	\stepcounter{MyTaskCounter}
	\hfill\\[0.1in]
	\ifthenelse{\equal{\theMyTaskSectionCounter}{0}}{
	   \textbf{\large Задача №\theMyTaskCounter}
	}{
	   \textbf{\large Задача №\theMyTaskSectionCounter.\theMyTaskCounter}
	}
	\ifthenelse{\equal{#1}{}}{}{{\normalsize (#1)}}
	\hfill\\[0.05in]
}

% Math and algorithms

\makeatletter
\renewcommand{\ALG@name}{Алгоритм}
\renewcommand{\listalgorithmname}{Список алгроитмов}

\newenvironment{procedure}[1]
  {\renewcommand*{\ALG@name}{Процедура}
  \algorithm\renewcommand{\thealgorithm}{\thechapter.\arabic{algorithm} #1}}
  {\endalgorithm}

\makeatother

\algrenewcommand\algorithmicrequire{\textbf{Вход:}}
\algrenewcommand\algorithmicensure{\textbf{Выход:}}
\algnewcommand\True{\textbf{true}\space}
\algnewcommand\False{\textbf{false}\space}
\algnewcommand\And{\textbf{and}\space}

\newcommand{\xfor}[3]{#1 \textbf{from} #2 \textbf{to} #3}
\newcommand{\xassign}[2]{\State #1 $\leftarrow$ #2}
\newcommand{\xstate}[1]{\State #1}
\newcommand{\xreturn}[1]{\xstate{\textbf{return} #1}}

\DeclarePairedDelimiter\ceil{\lceil}{\rceil}
\DeclarePairedDelimiter\floor{\lfloor}{\rfloor}

\newcommand{\bigO}[1]{\mathcal{O}\left(#1\right)}

\title{Домашнее задание №3 \\ Информационный поиск. 6 курс. Осенний семестр.}
\author{Горбунов Егор Алексеевич}
\date{11 ноября 2016 г.}
\begin{document}
\maketitle

\begin{task}[1]
Перед вами матрица смежности <<термин-документ>>, описывающая некую коллекцию (строки -- термы, столбцы -- документы):
\begin{equation*}
	C = 
	\begin{pmatrix}
	1 & 1 \\
	0 & 1 \\
	1 & 0
	\end{pmatrix}
\end{equation*}
\begin{enumerate}
	\item Вычислите матрицу совместной встречаемости $CC^T$. Что собой представляют диагональные элементы этой матрицы?
	\item Убедитесь, что сингулярное разложение матрицы $C$ выглядит следующим образом:
	\begin{equation*}
		U = 
		\begin{pmatrix}
		 -0.816 & 0.000 \\
		 -0.408 & -0.707 \\
		 -0.408 & 0.707
		\end{pmatrix}
		,
		\Sigma =
		\begin{pmatrix}
		1.732 & 0.000 \\
		0.000 & 1.000
		\end{pmatrix}
		,
		V^T = 
		\begin{pmatrix}
		-0.707 & -0.707 \\
		0.707 & -0.707
		\end{pmatrix}
	\end{equation*}
	\item Что собой представляют элементы матрицы $C^TC$? \label{t1:item3} 
\end{enumerate}
\end{task}
\begin{solution}
\begin{enumerate}
	\item Обозначим вектора слов:
	\begin{equation*}
		w_1 = \mmat{1\\1},
		w_2 = \mmat{0\\1},
		w_3 = \mmat{1\\0}
	\end{equation*}
	Элемент вектора $w_i[j]$ обозначает, встретилось ли слово $w_i$ в документе $D_j$.\\
	Тогда матрица совместной встречаемости ($\cdot$ -- скалярное произведене, dot product):
	\begin{equation*}
		CC^T = 
		\begin{pmatrix}
		w_1^T \\
		w_2^T \\
		w_3^T
		\end{pmatrix}
		\begin{pmatrix}
		w_1 & w_2 & w_3
		\end{pmatrix}^T
		=
		\begin{pmatrix}
		w_1 \\ w_2 \\ w_3
		\end{pmatrix}
		\begin{pmatrix}
		w_1 & w_2 & w_3
		\end{pmatrix}
		=
		\begin{pmatrix}
		w_1 \cdot w_1 && w_1 \cdot w_2 && w_1 \cdot w_3 \\
		w_2 \cdot w_1 && w_2 \cdot w_2 && w_2 \cdot w_3 \\
		w_3 \cdot w_1 && w_3 \cdot w_2 && w_3 \cdot w_3
		\end{pmatrix}
	\end{equation*}
	Откуда:
	\begin{equation*}
	CC^T=
	\begin{pmatrix}
	2 & 1 & 1 \\
	1 & 1 & 0 \\
	1 & 0 & 1 
	\end{pmatrix}
	\end{equation*}
	В силу того, что $w_i$ --- вектор над $\xbrace{1, 0}$, видно, что диагональный элемент матрицы $CC^T[i, i]$ равен числу документов коллекции, в которых встретилось слово $w_i$. Вообще: $CC^T[i, j]$ -- это скалярное произведение $w_i \cdot w_j$, которое
	представляет из себя сумму элементов вектора, в котором $1$ стоят на таких позициях $k$, что $w_i[k] = w_j[k] = 1$. Таким образом сумма элементов данного вектора будет равна число документов, в которых одновременно встречается как слово $w_i$ так и слово $w_j$.
	\item Перемножив данные матрицы получим:
	\begin{equation*}
	U\Sigma V^T = 
	\begin{pmatrix}
		0.9992 & 0.9992 \\
		-0.0002 & 0.9994 \\ 
        0.9994 & -0.0002
	\end{pmatrix}
	\end{equation*}
	Вообщем-то похоже, но с погрешность. Как минимум по-этому SVD стоит пересчитать руками =) Также, по определению сингулярного разложения исходная матрица раскладывается в произведение унитарной, диагональной (из ненулевых сингулярных чисел) и ещё одной унитарной матрицы. Матрица $U$, приведённая в задании не является унитарной, как минимум потому, что не является квадратной. Таким образом приведённое разложение нельзя называть сингулярным (как я понимаю), хотя это и не означает, что такое разложение неверно. Поэтому найдём сингулярное разложение матрицы $C$ своими силами.
	Будем искать матрицы $U$, $V$ и $\Sigma$, что $U$ имеет размер $3\times3$, $\Sigma$ $3\times2$, а $V$ $2 \times 2$. В силу унитарности матриц $U$ и $V$ можем записать следующее:
	\begin{align*}
		& U^T U = UU^T = V^TV = VV^T = I \\
		& CC^T = U\Sigma V^T V\Sigma^TU^T = U\Sigma\Sigma^TU^T \\
		& C^TC = V\Sigma^TU^TU\Sigma V^T = V\Sigma^T\Sigma V^T 
	\end{align*}
	Тут матрица $\Sigma^T \Sigma = \Sigma\Sigma^T = \Lambda$ --- это квадратная диагональная матрица $3\times3$ (на диагонали могут быть нули).\\
	Откуда мы получаем, домножая справа обе части на нужные $U$ в одном уравнении и на $V$ в другом:
	\begin{align*}
		& (CC^T)U = U\Lambda \\
		& (C^TC)V = V\Lambda
	\end{align*}
	Можно переписать это так:
	\begin{align*}
		(CC^T)\vec{u_i} &= \vec{u_i}\lambda_i,\ \text{для всех столбцов $u_i$ матрицы $U$} \\
		(C^TC)\vec{v_i} &= \vec{v_i}\lambda_i,\ \text{для всех столбцов $v_i$ матрицы $V$} \\
		\Lambda &= 
		\begin{pmatrix}
			\lambda_1 & 0 & 0 \\
			0 & \lambda_2 & 0 \\
			0 & 0 & \lambda_3
		\end{pmatrix},
		\ \Sigma[i][i] = \sqrt{\lambda_i}
	\end{align*}
	Видим, что собственные числа (ненулевые!) матриц $C^TC$ и $CC^T$ совпадают и равны квадратам искомых сингулярных чисел, составляющих $\Sigma$, а столбцы матрицы $V$ --- собственные вектора $C^TC$ и столбцы $U$ --- собственные вектора $CC^T$.
	Таким образом нам нужно отыскать собственные числа и вектора матриц:
	\begin{equation*}
	CC^T=
	\begin{pmatrix}
	2 & 1 & 1 \\
	1 & 1 & 0 \\
	1 & 0 & 1 
	\end{pmatrix},\
	C^TC=
	\begin{pmatrix}
	1 & 0 & 1 \\
	1 & 1 & 0 
	\end{pmatrix}
	\begin{pmatrix}
	1 & 1 \\
	0 & 1 \\
	1 & 0
	\end{pmatrix}
	=
	\begin{pmatrix}
	2 & 1 \\
	1 & 2 
	\end{pmatrix}
	\end{equation*}
	Начнём с собственных чисел. Ищем их через характеристические многочлены (приравнивая их к $0$, почему так делается я не поясняю, т.к. это за рамками курса):
	\begin{align*}
	&det
	\begin{pmatrix}
	2 - \lambda & 1 & 1 \\
	1 & 1 - \lambda & 0 \\
	1 & 0 & 1 - \lambda
	\end{pmatrix}
	= (2-\lambda)(1-\lambda)^2-(1-\lambda)-(1-\lambda)=\lambda(1-\lambda)(\lambda-3)
	 \\
	&\Lambda = 
	\begin{pmatrix}
	3 & 0 & 0 \\
	0 & 1 & 0 \\
	0 & 0 & 0
	\end{pmatrix}
	\end{align*}
	Чтобы получить $U$ и $V$ теперь нужно найти собственные векторы соответствующие собственным числам $3$ и $1$ матриц 
	$CC^T$ и $C^TC$. Теперь уже совсем очевидно, что столбец $\vec{u_3}$ может быть произвольным собственным вектором (соответствует с.ч. $0$) и роли он в разложении играть не будет. Для векторов получаем следующие системы (матрица системы та же, что под $det$ выше, но с уже подставленными $\lambda$) ($u_1, v_1$ соответствует $\lambda = 3$, $u_2, v_2$ соотв. $\lambda = 1$). Т.к. итоговые матрицы $U$ и $V$ должны быть унитарными, то вектора необходимо нормализовать.
	\begin{align*}
	\begin{cases}
	-u_{11} + u_{12} + u_{13} = 0 \\
	u_{11} - 2u_{12} = 0 \\
	u_{11} - 2u_{13} = 0
	\end{cases}
	\text{, откуда легко подобрать } \mmat{-2 \\ -1 \\ -1} \Rightarrow u_1 = \frac{1}{\sqrt{6}}\mmat{-2 \\ -1 \\ -1}\\
	\begin{cases}
	u_{21} + u_{22} + u_{23} = 0 \\
	u_{21} = 0 \\
	u_{21} = 0
	\end{cases}
	\text{, откуда легко подобрать } \mmat{0 \\ -1 \\ 1} \Rightarrow u_2 = \frac{1}{\sqrt{2}}\mmat{0 \\ -1 \\ 1}
	\end{align*}
	Аналогично можно подобрать собственный вектор для $\lambda_3 = 0$: $u_3 = \frac{1}{\sqrt{3}}\mmat{-1 \\ 1 \\ 1}$.
	Для $v_i$:
		\begin{align*}
	\begin{cases}
	-v_{11} + v_{12} = 0 \\
	v_{11} - v_{12} = 0 
	\end{cases}
	\text{, откуда легко подобрать } \mmat{-1 \\ -1} \Rightarrow v_1 = \frac{1}{\sqrt{2}}\mmat{-1 \\ -1}\\
	\begin{cases}
	v_{21} + v_{22} = 0 \\
	v_{21} + v_{22}= 0 
	\end{cases}
	\text{, откуда легко подобрать } \mmat{1 \\ -1} \Rightarrow v_2 = \frac{1}{\sqrt{2}}\mmat{1 \\ -1}
	\end{align*}
	Итого мы получили искомое разложение:
	\begin{align*}
	U &= 
	\begin{pmatrix}
	u_1 && u_2 && u_3 
	\end{pmatrix}
	=
	\begin{pmatrix}
	\frac{-2}{\sqrt{6}} & 0 & \frac{-1}{\sqrt{3}} \\
	\frac{-1}{\sqrt{6}} & \frac{-1}{\sqrt{2}} & \frac{1}{\sqrt{3}} \\
	\frac{-1}{\sqrt{6}} & \frac{1}{\sqrt{2}} & \frac{1}{\sqrt{3}}
	\end{pmatrix} \\
	V^T &=
	\begin{pmatrix}
	v_1^T \\
	v_2^T 
	\end{pmatrix}
	=
	\begin{pmatrix}
	\frac{-1}{\sqrt{2}} & \frac{-1}{\sqrt{2}} \\
	\frac{1}{\sqrt{2}} & \frac{-1}{\sqrt{2}}
	\end{pmatrix} \\
	\Sigma &= 
	\begin{pmatrix}
		\sqrt{3} & 0 \\
		0 & 1 \\
		0 & 0
	\end{pmatrix}
	\end{align*}
	Конечно, при вычислении собственных векторов я выбирал те знаки, которые бы соответствовали тому, что дано в задании =)
	Теперь, если перевести все значения в десятичные дроби легко увидеть, что полученные матрицы совпадают (если округлить до нужного числа знаков), кроме того, что вычисленная явно $U$ является честно унитарной, хоть на дальнейшие выкладки (например, при вычислении новых векторов для документов (LSA)) это не влияет.

	\item Аналогично пункту (a) введём вектора документов:
	\begin{align*}
	D_1 = \mmat{1 \\ 0 \\ 1}
	D_2 = \mmat{1 \\ 1 \\ 0}
	\end{align*}
	Тут если $D_i[k] = 1$, то терм $w_k$ встретился в документе $D_i$.\\
	Тогда получим:
	\begin{equation*}
	C^TC =
	\begin{pmatrix}
	D_1 \cdot D_1 & D_1 \cdot D_2 \\
	D_2 \cdot D_1 & D_2 \cdot D_2
	\end{pmatrix}
	=
	\begin{pmatrix}
	2 & 1 \\
	1 & 2 
	\end{pmatrix}
	\end{equation*}
	Откуда легко видеть, опять же аналогично первому пункту задания, что $C^TC[i, j]$ --- это число термов, которые \emph{одновременно} встречаются в документах $D_i$ и $D_j$, т.е. это мощность пересечения мешков слов для документов $D_i$ и $D_j$. Ещё можно написать, что $C^TC[i, j] = sim(D_i, D_j) ||D_i|| ||D_j||$.
\end{enumerate}
\end{solution}

\begin{task}[2]
Для чего используются распределения Дирихле $Dir(\vec{\alpha})$ и $Dir(\vec{\beta})$ в тематических
моделях? Что контролируют параметры $\vec{\alpha}$ и $\vec{\beta}$? Какие значения этих параметром имеет смысл использовать и почему?
\end{task}
\begin{solution}
Начнём с того, что распределение Дирихле -- это \emph{сопряжённое априорное распределение} для мультиномиального распределения. Функция плотности вероятности распределения Дирихле:
\begin{align*}
Dir(\vec{p} | \vec{\alpha}) &= \frac{1}{B(\vec{\alpha})}\prod_{i = 1}^{m}{p_i^{\alpha_i - 1}}, \text{ где } \vec{p} = \xbrace{p_1, \ldots, p_m}, \vec{\alpha} = \xbrace{\alpha_1, \ldots, \alpha_m} \\
B(\vec{\alpha}) &= \frac{\prod_{i = 1}^m{\Gamma(\alpha_i)}}{\Gamma(\sum_{i=1}^m{\alpha_i})} = \xbracket{\text{при натуральных } \alpha_i} = \frac{\prod_{i=1}^m{(\alpha_i-1)!}}{(\sum_{i=1}^m{\alpha_i}-1)!}
\end{align*}
$B(\vec{\alpha})$ --- бета-функция, при раскрытии которой выше было использовано, что $\Gamma(x)$ (гамма-функция) --- это обобщение факториала.
\end{solution}
\end{document}