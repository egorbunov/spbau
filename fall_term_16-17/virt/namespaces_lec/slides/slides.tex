%!TEX program = xelatex
\documentclass[12pt, a4paper]{article}
\usepackage[utf8]{inputenc}
\usepackage[russian]{babel}
\usepackage{pscyr}
\usepackage{amssymb}
\usepackage{xifthen}
\usepackage{parskip}
\usepackage{hyperref}
\usepackage{setspace}

\usepackage{graphicx}
\usepackage{xcolor}
\usepackage{amsmath}
\usepackage{MnSymbol}
\usepackage{amsthm}
\usepackage{mathtools}
\usepackage{algorithm}
\usepackage[noend]{algpseudocode}
\usepackage[shortlabels]{enumitem}
                    \setlist[enumerate, 1]{1\textsuperscript{o}}
\usepackage{subfig}
\usepackage{tikz}
\usepackage{tikz,fullpage}
\usetikzlibrary{shapes,snakes}
\usetikzlibrary{arrows,%
                petri,%
                topaths}%
\usepackage{tkz-berge}
\usepackage[top=0.5in, bottom=0.7in, left=0.6in, right=0.6in]{geometry}
\linespread{1.3}

% \renewcommand\familydefault{\sfdefault}


% Stuff related to homework specific documents
\newcounter{MyTaskCounter}
\newcounter{MyTaskSectionCounter}
\newcommand{\tasksection}[1]{
	\stepcounter{MyTaskSectionCounter}
	\setcounter{MyTaskCounter}{0}
	\ifthenelse{\equal{#1}{}}{}{
	{\hfill\\[0.2in] \Large \textbf{\theMyTaskSectionCounter \enspace #1} \hfill\\[0.1in]}}
}

\newcommand{\task}[1]{
	\stepcounter{MyTaskCounter}
	\hfill\\[0.1in]
	\ifthenelse{\equal{\theMyTaskSectionCounter}{0}}{
	   \textbf{\large Задача №\theMyTaskCounter}
	}{
	   \textbf{\large Задача №\theMyTaskSectionCounter.\theMyTaskCounter}
	}
	\ifthenelse{\equal{#1}{}}{}{{\normalsize (#1)}}
	\hfill\\[0.05in]
}

% Math and algorithms

\makeatletter
\renewcommand{\ALG@name}{Алгоритм}
\renewcommand{\listalgorithmname}{Список алгроитмов}

\newenvironment{procedure}[1]
  {\renewcommand*{\ALG@name}{Процедура}
  \algorithm\renewcommand{\thealgorithm}{\thechapter.\arabic{algorithm} #1}}
  {\endalgorithm}

\makeatother

\algrenewcommand\algorithmicrequire{\textbf{Вход:}}
\algrenewcommand\algorithmicensure{\textbf{Выход:}}
\algnewcommand\True{\textbf{true}\space}
\algnewcommand\False{\textbf{false}\space}
\algnewcommand\And{\textbf{and}\space}

\newcommand{\xfor}[3]{#1 \textbf{from} #2 \textbf{to} #3}
\newcommand{\xassign}[2]{\State #1 $\leftarrow$ #2}
\newcommand{\xstate}[1]{\State #1}
\newcommand{\xreturn}[1]{\xstate{\textbf{return} #1}}

\DeclarePairedDelimiter\ceil{\lceil}{\rceil}
\DeclarePairedDelimiter\floor{\lfloor}{\rfloor}

\newcommand{\bigO}[1]{\mathcal{O}\left(#1\right)}

\newcommand{\xqed}{\hfill $\blacksquare$}

\newcommand{\code}[1]{\colorbox{gray!15}{\footnotesize\texttt{#1}}}
\makeatletter
\def\input@path{{./fig/}}
\makeatother
\graphicspath{{./fig/}}

\title{Linux namespaces}
\author[Егор Горбунов]{
	Егор Горбунов
}
\institute{CПбАУ}
% \date{20 мая 2016 г.}

\begin{document}
\maketitle

\begin{frame}{Содержание}
\begin{large}
\begin{enumerate}
	\item \hyperlink{namespaces_overview}{Напоминание о namespaces}
	\item \hyperlink{uts_ns_usage}{Использование UTS NS}
	\item \hyperlink{pid_ns_usage}{Использование PID NS}
	\item \hyperlink{ns_internals}{Namespaces внутри}
	\item \hyperlink{uts_ns_inter}{Устройство UTS NS}
	\item \hyperlink{pid_ns_inter}{Устройство PID NS}
\end{enumerate}
\end{large}
\end{frame}

\begin{frame}[label={namespaces_overview}]
\frametitle{Namespaces}
\begin{tabular}{lp{9cm}}
	\alert{User} & Изоляция ID пользователя, ID группы \\
	\alert{\textbf{PID}} & \textbf{Изоляция ID процесса} \\
	\alert{\textbf{UTS}} & \textbf{Изоляция Hostname и доменного имени NIS} \\
	\alert{Network} & Изоляция сетевого стека: таблица маршрутизации, firewall rules, сетевые устройства, порты \\
	\alert{IPC} & Изоляция объектов межпроцессного взаимодействия: семафоры (sem), очереди сообщений (msg), общая память (shm) \\
	\alert{Mount} & Изоляция точек монтирования \\
	\alert{Cgroup} & Изоляция иерархии cgroup \\
	... & 
\end{tabular}
\end{frame}

\begin{frame}[label={uts_ns_usage}]
\frametitle{Что позволяет UTS NS?}
\begin{enumerate}
	\item \alert{Изменять \emph{hostname}} для процесса (контейнера)
	\begin{itemize}
		\item Идентификация контейнера
		\item Используется для администрирования (логи, ...)
		% (hostname используется многими приложениями как идентификатор; записи в логах могут классифицироваться по hostname)
		% работа в терминале с разными контейнерами ===> легче различать по именам
	\end{itemize}
	\item \alert{Изменять \emph{NIS domain name}} для процесса (контейнера)
	\begin{itemize}
		\item NIS --- Network Information Service --- устаревшая технология, используется для централизованного хранения файлов конфигураций (на сервере) и распространения их по клиентам
		\item Разные NIS domain name $\Rightarrow$ разные конфигурации
	\end{itemize}
\end{enumerate}
\end{frame}

\begin{frame}[fragile]
\frametitle{Как работать с UTS NS?}
\begin{lstlisting}[language=bash, basicstyle={\normalsize\ttfamily}]
$ sudo unshare --uts
$ hostname leonid
$ ypdomainname some-nis-domain
$ ping leonid
\end{lstlisting}
\end{frame}

\begin{frame}[fragile, label={pid_ns_usage}]
\frametitle{Что позволяет PID NS?}
\begin{itemize}
	\item Создавать свежее дерево процессов (набор PIDs)
	\item Важно для контейнеров
\end{itemize}

\begin{block}{}
\begin{lstlisting}[language=bash, basicstyle={\normalsize\ttfamily}]
$ unshare --fork --pid --user
$ echo $$
1
\end{lstlisting}
\end{block}

Многие приложения рассчитывают на то, что PID константный, \texttt{getpid()} кэширует PID $\Longrightarrow$ \texttt{unshare(), setns()} вступает в силу лишь для первого ребёнка (после \texttt{fork()})

\end{frame}

\begin{frame}[fragile, label={ns_internals}]
\frametitle{Namespaces внутри}
\begin{lstlisting}[title={\href{http://lxr.free-electrons.com/source/include/linux/sched.h\#L1658}{sched.h}}]
struct task_struct {
    ...
/* namespaces */
    struct nsproxy *nsproxy;
    ...
}
\end{lstlisting}
\begin{lstlisting}[title={\href{http://lxr.free-electrons.com/source/include/linux/nsproxy.h\#L30}{nsproxy.h}}]
struct nsproxy {
     atomic_t count;
     struct uts_namespace *uts_ns;
     struct ipc_namespace *ipc_ns;
     struct mnt_namespace *mnt_ns;
     struct pid_namespace *pid_ns_for_children;
     struct net           *net_ns;
     struct cgroup_namespace *cgroup_ns;
};
\end{lstlisting}
\end{frame}

\begin{frame}[fragile]
\frametitle{\texttt{nsproxy}}
\begin{small}
\begin{itemize}
	\item copy on write
	\item \alert{изменять} \textbf{\texttt{task->nsproxy}} можно только если \textbf{\texttt{task}} --- текущая задача
\end{itemize}
\end{small}
\begin{lstlisting}[basicstyle={\tiny\ttfamily}, title=\href{http://lxr.free-electrons.com/source/include/linux/nsproxy.h\#L67}{nsproxy methods}]
int copy_namespaces(unsigned long flags, struct task_struct *tsk);
void exit_task_namespaces(struct task_struct *tsk);
void switch_task_namespaces(struct task_struct *tsk, struct nsproxy *new);
void free_nsproxy(struct nsproxy *ns);
int unshare_nsproxy_namespaces(unsigned long, struct nsproxy **, struct cred *, struct fs_struct *);
int __init nsproxy_cache_init(void);

static inline void put_nsproxy(struct nsproxy *ns){
        if (atomic_dec_and_test(&ns->count)) {
                free_nsproxy(ns);
        }
}

static inline void get_nsproxy(struct nsproxy *ns){
        atomic_inc(&ns->count);
}
\end{lstlisting}
\end{frame}

\begin{frame}[fragile]
\frametitle{Структура одного namespace}
\begin{lstlisting}[title=\href{http://lxr.free-electrons.com/source/include/linux/ns_common.h\#L6}{ns\_common}]
struct ns_common {
    atomic_long_t stashed;
    const struct proc_ns_operations *ops;
    unsigned int inum;
};
\end{lstlisting}
\begin{lstlisting}[title=\href{http://lxr.free-electrons.com/source/include/linux/proc_ns.h\#L15}{proc\_ns\_operations}]
struct proc_ns_operations {
    const char *name;
    int type;
    struct ns_common *(*get)(struct task_struct *task);
    void (*put)(struct ns_common *ns);
    int (*install)(struct nsproxy *nsproxy, struct ns_common *ns);
};
\end{lstlisting}

\end{frame}

\begin{frame}[fragile]
\frametitle{namespaces в \texttt{fork()}, \texttt{clone()}, \texttt{unshare()}}

\begin{block}{\texttt{fork()}, \texttt{clone()}}
	\texttt{copy\_namespaces}
\end{block}

\begin{block}{\texttt{unshare()}}
	\texttt{unshare\_nsproxy\_namespaces} $\Rightarrow$ \texttt{switch\_task\_namespaces}
\end{block}

\begin{block}{\texttt{exit()}}
	\texttt{exit\_task\_namespaces}
\end{block}

\begin{block}{\texttt{setns()}}
	\texttt{create\_new\_namespaces} $\Rightarrow$ \texttt{ns->ops->install}
\end{block}

\end{frame}

\begin{frame}[fragile, label={uts_ns_inter}]
\frametitle{Устройство UTS NS}
\begin{lstlisting}[title=\href{http://lxr.free-electrons.com/source/include/linux/utsname.h\#L23}{uts\_namespace}]
struct uts_namespace {
    struct kref kref;
    struct new_utsname name;
    struct user_namespace *user_ns;
    struct ns_common ns;
};
\end{lstlisting}
\begin{lstlisting}[title=\href{http://lxr.free-electrons.com/source/include/uapi/linux/utsname.h\#L24}{utsname.h}]
struct new_utsname {
    char sysname[__NEW_UTS_LEN + 1];
    char nodename[__NEW_UTS_LEN + 1];
    char release[__NEW_UTS_LEN + 1];
    char version[__NEW_UTS_LEN + 1];
    char machine[__NEW_UTS_LEN + 1];
    char domainname[__NEW_UTS_LEN + 1];
};
\end{lstlisting}
\end{frame}

\begin{frame}[fragile, label={pid_ns_inter}]
\frametitle{Устройство PID NS}
\begin{lstlisting}[basicstyle={\tiny\ttfamily}, title=\href{http://lxr.free-electrons.com/source/include/linux/pid_namespace.h\#L24}{pid\_namespace.h}]
struct pid_namespace {
        ...
        struct pidmap pidmap[PIDMAP_ENTRIES];
        int last_pid;
        unsigned int nr_hashed;
        struct task_struct *child_reaper;
        struct kmem_cache *pid_cachep;
        unsigned int level;
        struct pid_namespace *parent;
        struct user_namespace *user_ns;
        struct work_struct proc_work;
        kgid_t pid_gid;
        int hide_pid;
        struct ns_common ns;
        ...
};
\end{lstlisting}

\end{frame}

\end{document}
