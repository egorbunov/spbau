%!TEX program = xelatex


\documentclass[12pt, a4paper]{article}
\usepackage[utf8]{inputenc}
\usepackage[russian]{babel}
\usepackage{pscyr}
\usepackage{amssymb}
\usepackage{xifthen}
\usepackage{parskip}
\usepackage{hyperref}
\usepackage{setspace}

\usepackage{graphicx}
\usepackage{xcolor}
\usepackage{amsmath}
\usepackage{MnSymbol}
\usepackage{amsthm}
\usepackage{mathtools}
\usepackage{algorithm}
\usepackage[noend]{algpseudocode}
\usepackage[shortlabels]{enumitem}
                    \setlist[enumerate, 1]{1\textsuperscript{o}}
\usepackage{subfig}
\usepackage{tikz}
\usepackage{tikz,fullpage}
\usetikzlibrary{shapes,snakes}
\usetikzlibrary{arrows,%
                petri,%
                topaths}%
\usepackage{tkz-berge}
\usepackage[top=0.5in, bottom=0.7in, left=0.6in, right=0.6in]{geometry}
\linespread{1.3}

% \renewcommand\familydefault{\sfdefault}


% Stuff related to homework specific documents
\newcounter{MyTaskCounter}
\newcounter{MyTaskSectionCounter}
\newcommand{\tasksection}[1]{
	\stepcounter{MyTaskSectionCounter}
	\setcounter{MyTaskCounter}{0}
	\ifthenelse{\equal{#1}{}}{}{
	{\hfill\\[0.2in] \Large \textbf{\theMyTaskSectionCounter \enspace #1} \hfill\\[0.1in]}}
}

\newcommand{\task}[1]{
	\stepcounter{MyTaskCounter}
	\hfill\\[0.1in]
	\ifthenelse{\equal{\theMyTaskSectionCounter}{0}}{
	   \textbf{\large Задача №\theMyTaskCounter}
	}{
	   \textbf{\large Задача №\theMyTaskSectionCounter.\theMyTaskCounter}
	}
	\ifthenelse{\equal{#1}{}}{}{{\normalsize (#1)}}
	\hfill\\[0.05in]
}

% Math and algorithms

\makeatletter
\renewcommand{\ALG@name}{Алгоритм}
\renewcommand{\listalgorithmname}{Список алгроитмов}

\newenvironment{procedure}[1]
  {\renewcommand*{\ALG@name}{Процедура}
  \algorithm\renewcommand{\thealgorithm}{\thechapter.\arabic{algorithm} #1}}
  {\endalgorithm}

\makeatother

\algrenewcommand\algorithmicrequire{\textbf{Вход:}}
\algrenewcommand\algorithmicensure{\textbf{Выход:}}
\algnewcommand\True{\textbf{true}\space}
\algnewcommand\False{\textbf{false}\space}
\algnewcommand\And{\textbf{and}\space}

\newcommand{\xfor}[3]{#1 \textbf{from} #2 \textbf{to} #3}
\newcommand{\xassign}[2]{\State #1 $\leftarrow$ #2}
\newcommand{\xstate}[1]{\State #1}
\newcommand{\xreturn}[1]{\xstate{\textbf{return} #1}}

\DeclarePairedDelimiter\ceil{\lceil}{\rceil}
\DeclarePairedDelimiter\floor{\lfloor}{\rfloor}

\newcommand{\bigO}[1]{\mathcal{O}\left(#1\right)}

\newcommand{\xqed}{\hfill $\blacksquare$}

\newcommand{\code}[1]{\colorbox{gray!15}{\footnotesize\texttt{#1}}}
\title{Домашнее задание №2 \\ Информационный поиск. 6 курс. Осенний семестр.}
\author{Горбунов Егор Алексеевич}
\date{11 ноября 2016 г.}
\begin{document}
\maketitle

\begin{task}[1]
Рассмотренные методы исправления опечаток не работают напрямую при
пропуске пробела (например, informationretrieval). Опишите, как исправлять такие
опечатки (не обязательно на основе рассмотренных методов).
\end{task}
\begin{solution}
\begin{enumerate}
	\item Если допустить, что пользователь редко по-случайности склеивает более 2-ух слов, то можно хранить, помимо словаря слов, словарь коллокаций длины 2. Таким образом теперь для исправления пропуска \emph{одного} пробела можно использовать обычное редакционное расстояния, которое должно теперь учитывать, что пробел --- это тоже символ. Минус --- сильно возрастает размер словаря и метод становится непрактичен (если считать, что в Английском языке около 400000 слов, что умещается в несколько мегабайт, то все коллокации займут около пары терабайт).
	\item Будем сканировать входное слово $w$ из запроса $q$ слева направо до тех пор, пока полученный префикс $p$ не будет словом из словаря. Далее отрезаем этот префикс от слова и повторяем те же действия с оставшейся строкой и таким образом получаем места, где нужно расставлять пробелы. У такого метода есть проблемы:
	\begin{itemize}
		\item Не учитывает опечаток помимо пробелов
		\item Что делать, если дошли до конца $w$, а набранный префикс не в словаре? Нужно добавлять откат назад и пытаться расширить уже существующее слово из словаря, равное какому-то префиксу.
		\item Имеет тенденцию разбивать слово на более короткие куски, т.е., если в словаре есть слова \emph{info} и \emph{rmation}, то \emph{informationretieval} превратится в \emph{info rmation retrieval}. В данном случае лучше предпочитать большие части, т.к. они, скорее всего, несут больше смысла.
	\end{itemize}
	\item Теперь придумаем что-нибудь более содержательное. Имея слово $w[1, n]$ (длины $n$) мы хотим найти самое ближайшее к нему исправление с возможными разбиениями (вставкой пробела). Рассмотрим такое оптимальное разбиение $w[1, n]$: $(w[1, i], w[i+1, k], w[k+1, n])$ мы точно можем заключить, что оптимальное разбиение, например $w[i+1, n]$ --- это $(w[i+1, k], w[k+1, n])$. Эту идею можно формализовать в виде задачи динамического программирования:
	\begin{align*}
		d(i, j)  \text{ --- стоимость редактирования куска } w[i, j] \\
		d(i, i) = \infty \text{ (запрет разбиения на отдельные буквы) } \\
		d(i, j) = \min\xbrace{\min_{i\leq k < k}\xparen{d(i, k) + d(k+1, j) + 1}, editDist(w[i, j])}
	\end{align*}
	Тут $editDist$ --- это обычное редакционное расстояние, которое находится по словарю (т.е. поиск самого близкого слова к данному). Ответ на задачу считается последовательно по возрастанию размера куска и будет записан в $d(1, |w|)$. Таким образом мы и находим оптимальное разбиение пробелами. Единичка в выражении $d(i, k) + d(k+1, j) + 1$ --- это штраф за разбиение. Заметим, что можно добавить запреты на очень короткие слова путём задания больших значений для $d(i, i + len)$. По сложности данный метод получается не сильно хуже, чем обычное редакционное расстояние (куб от длины слова (она не очень большая) на время вычисления editDistance). Вместо редакционного расстояния можно использовать и коэффициенты Жаккара.
\end{enumerate}
\end{solution}

\begin{task}[2]
Мы рассмотрели два типа методов для рекомендации запросов, аналогичных
заданному запросу:
\begin{enumerate}[label=\alph*)]
	\item Рекомендовать запросы, встречающиеся в одной сессии с заданным запросом.
	\item Рекомендовать те запросы, у которых множество кликнутых результатов сильно пересекается с аналогичным множеством для заданного запроса.
\end{enumerate}
Какие еще методы для рекомендации запросов вы можете предложить?
\end{task}
\begin{solution}
Приведённые методы рекомендации не привлекают глубокого анализа пользовательского запроса, а опираются не действие пользователей. Поэтому хочется предложить методы, которые учитывают контекст запроса. При помощи методов NLP можно попытаться извлечь из запроса сущности, такие как фильмы, музыку и места. Далее на помощь нам приходят графе сущностей и отношений между ними: по фильмам мы можем находить актёров и прочих личностей, по музыке мы можем находить композиторов и другие их произведения, по местам мы можем находить похожие по типу и по расположению на карте места. Все эти данные могут служить рекомендациями.
\end{solution}

\begin{task}[3]
Вы планируете использовать следующие методы поиска: модель векторного
пространства с весами TF-IDF, BM25, языковую модель. Какую минимальную
информацию должен содержать индекс, чтобы поддерживать эффективное использование
этих методов? Какую информацию нет смысла хранить в индексе? Как ее нужно
хранить? Дайте развернутый ответ.
\end{task}
\begin{solution}
\begin{itemize}
	\item[TF-IDF] Минимальная информация для эффективного использования:
	\begin{itemize}
		\item Векторные представления каждого документа (считать векторные представления на лету было бы очень затратно)
		\item Значения TF-IDF для всех возможных слов (термов), чтобы считать векторное представление запроса быстро
		\item Обратный индекс из термов в документы, ясное дело, чтобы быстро формировать множества документов, в которых будет присутствовать слово из запроса. Перебирать все документы не нужно, т.к. косинусное расстояние у документов не содержащих ни одного слова из запроса с запросом будет заведомо равным 0
	\end{itemize}
	Не нужно хранить, например, векторные представления для всевозможных запросов.
	Хранить просто: всем документам раздать id и устроить мапу из id в векторное представление. В обратном индексе хранить map из терма в список id. И так же хранить мап из слов в tfifd для них.
	Сами документы, конечно, нужно уметь доставать по id.
	\item[BM25] Минимальная информация для эффективного использования (и её хранение):
	\begin{itemize}
		\item Храним $tf(t, d)$ для каждого терма и документа. Храним в двухуровневой мапе (на первом уровне может быть массив): $tf[d][t] := tf(t, d)$ Разумеется тут используем разреженные структуры данных.
		\item Храним значения $k_1 [(1-b) + b \frac{dl(d)}{dl_{ave}}]$ для каждого документа в массиве $arr[doc_{id}]$, чтобы не выполнять арифметические операции каждый раз
		\item Храним коэффициент $k = k_1 + 1$ (это и пункт выше позволяют упростить множитель в формуле BM25)
		\item Для терма храним $df(t)$ в мапе.
		\item Обратный индекс как в прошлом пункте
	\end{itemize}
	Не нужно хранить векторное представление документа.
	\item[LM] Что и как храним:
		\begin{itemize}
			\item В разреженных структурах храним вероятности $P(t | M_d)$ для каждого документа.
			\item Обратный индекс аналогично пунктам выше
		\end{itemize}
		Сами языковые модели мы не храним, т.к. они используются исключительно для получения вероятностей $P(t | M_d)$
\end{itemize}
\end{solution}

\begin{task}[4]
Отранжируйте документы из таблицы~\ref{table:task4} по запросу “car insurance” с использованием
модели векторного пространства и весов TF-IDF.
\begin{table}[ht!]
\centering
\begin{tabular}{llrrr}
    \multirow{2}{*}{Сорво} & \multirow{2}{*}{idf} & \multicolumn{3}{c}{tf} \\
    \cline{3-5}
    & & Документ 1 & Документ 2 & Документ 3 \\ 
    \hline
	car & $1.65$       & $27$ & $4$ & $24$ \\
	auto & $2.08$      & $3$ & $33$ & $0$ \\
	insurance & $1.62$ & $0$ & $33$ & $29$ \\
	best & $1.60$      & $14$ & $0$ & $17$
\end{tabular}
\caption{Частота слов в документах и обратная документная частота слов.}
\label{table:task4}
\end{table}
\end{task}
\begin{solution}
Вектора: (для запроса tf(car) = tf(insurance) = 1). Каждый элемент вектора получен умножением tf(word, doc) * idf(word)
\begin{table}[ht!]
\centering
\begin{tabular}{lccccc}
 & car & auto & insurance & best & норма вектора \\
 \hline
 Документ 1 & $44.55$ & $6.24$ & $0$ & $22.4$ & $50.25$\\
 Документ 2 & $6.6$ & $66.64$ & $53.46$ & $0$ & $85.68$ \\
 Документ 3 & $39.6$ & $0$ & $46.98$ & $27.2$ & $67.19$ \\
 Запрос     & $1.65$ & $0$ & $1.62$ & $0$ & $2.312$\\ 
\end{tabular}
\end{table}

Scores (косинусы):
\begin{itemize}
	\item $score(d1, q) = \frac{73.5}{50.25 * 2.312} = 0.633$
	\item $score(d2, q) = \frac{97.5}{85.68 * 2.312} = 0.492$
	\item $score(d3, q) = \frac{141.4}{67.19 * 2.312} = 0.91$
\end{itemize}
Ранжирование: D3, D1, D2
\end{solution}

\begin{task}[5]
Рассмотрим следующий запрос и три результата.
\begin{itemize}
	\item[Q] information retrieval course
	\item[D1] Information Retrieval and Web Search
	\item[D2] Introduction to Information Retrieval
	\item[D3] Text Retrieval and Search Engines
\end{itemize}
Результаты $1$ и $3$ – это страницы соответствующих курсов, поэтому пользователь
пометил их как релевантные. Результат $2$ – это страница с книгой, поэтому пользователь
пометил его как нерелевантный.
Примените алгоритм Роккио и выпишите вектор запроса после учета обратной
связи по релевантности. Элементы вектора перечислите в алфавитном порядке.
Считайте, что компоненты векторов содержат только частоты слов (без обратной
документной частоты и нормировки). Параметры алгоритма Роккио: $\alpha = 1$, $\beta =
0.75$, $\gamma = 0.25$.
\end{task}
\begin{solution}
TODO
\end{solution}

\begin{task}[6]
Выпишите формулу BM25 для длинных запросов. Опишите ее составляющие.
Каким образом каждая составляющая влияет на ранжирование (т.е. что происходит
с ранжированием результатов при изменении каждой из составляющих)?
\end{task}
\begin{solution}
TODO
\end{solution}

\begin{task}[7]
Пусть бинарная случайная величина $X_t$ – это индикатор того, что слово $t$
встречается в документе (т.е. $X_t = 1$, если слово $t$ есть в документе, и $X_t = 0$, если
слова $t$ нет в документе). $P_t = P(X_t = 1\ |\ d)$ – это вероятность того, что слово $t$
встречается в документе $d$.
Примените метод максимального правдоподобия (MLE) для формального вычисления
$P_t$ и покажите, что $P_t = \frac{tf(t, d)}{dl(d)}$, где $tf(t, d)$ – это частота слова $t$ в документе $d$, а
$dl(d)$ – это длина документа $d$.
\end{task}
\begin{solution}
TODO
\end{solution}

\begin{task}[8]
Рассмотрим коллекцию из двух документов.
\begin{itemize}
\item[D1] A language model is a probability distribution over words or sequences of words.
\item[D2] A language model is used in many natural language processing applications.
\end{itemize}
Выпишите сглаженную униграмную языковую модель для каждого документа. Используйте
сглаживание Jelinek-Mercer с параметром $\lambda = 0.5$. Отранжируйте эти документы
по запросу <<many words>>.
\end{task}
\begin{solution}
Сглаженная модель:
\[
P_s(t|M_d) = 0.5 (\frac{tf(t, d)}{dl(d)} + \frac{cf(t)}{cl})
\]
Разные величины:
\begin{itemize}
	\item $dl(D1) = 13$, $dl(D2) = 11$
	\item $cl = 13 + 11 = 24$
	\item Словарь ($cf$ в скобках): a(3), language(2), model(2), is(2), probability(1), distribution(1), over(1), words(2), or(1), sequences(1), of(1), used(1), in(1), many(1), natural(1), processing(1), applications(1)
\end{itemize}
\begin{center}
	\begin{tabular}{lcc}
			& D1 & D2 \\
		tf(a) & 2 & 1 \\
		tf(language) & 1 & 1 \\
		tf(model) & 1 & 1 \\
		tf(is) & 1 & 1 \\
		tf(probability) & 1 & 0 \\
		tf(distribution) & 1 & 0 \\
		tf(over) & 1 & 0 \\
		tf(words) & 2 & 0 \\
		tf(or) & 1 & 0 \\
		tf(sequences) & 1 & 0 \\
		tf(of) & 1 & 0 \\
		tf(used) & 0 & 1 \\
		tf(in) & 0 & 1 \\
		tf(many) & 0 & 1 \\
		tf(natural) & 0 & 1 \\
		tf(processing) & 0 & 1 \\
		tf(applications) & 0 & 1
	\end{tabular}
\end{center}
Модели документов:
\begin{center}
	\begin{tabular}{lcc}
			& D1 & D2 \\
		P(a) & $0.139$ & $0.108$ \\
		P(language) & $0.08$ & $0.087$ \\
		P(model) & $0.08$ & $0.087$ \\
		P(is) & $0.08$ & $0.087$ \\
		P(probability) & $0.059$ & $0.02$ \\
		P(distribution) & $0.059$ & $0.02$ \\
		P(over) & $0.059$ & $0.02$ \\
		P(words) & $0.118$ & $0.041$ \\
		P(or) & $0.059$ & $0.02$ \\
		P(sequences) & $0.059$ & $0.02$ \\
		P(of) & $0.059$ & $0.02$ \\
		P(used) & $0.02$ & $0.066$ \\
		P(in) & $0.02$ & $0.066$ \\
		P(many) & $0.02$ & $0.066$ \\
		P(natural) & $0.02$ & $0.066$ \\
		P(processing) & $0.02$ & $0.066$ \\
		P(applications) & $0.02$ & $0.066$
	\end{tabular}
\end{center}

Ранжирование: D1, D2 (это очевидно!)
\end{solution}

\end{document}